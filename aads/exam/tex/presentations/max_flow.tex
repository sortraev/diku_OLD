\subsection{Max flow}

\begin{enumerate}

  \item hello. today's topic is max flow.

  \item the most obvious applications of max flow are perhaps networking routing
    and bipartite matching, but max flow is also used in for example image
    segmentation.

  \item today we'll be looking at the Edmonds-Karp algorithm, a formalization
    and improvement on the Ford-Fulkerson method.

  \item for the sake of time conservation, I will assume you are familiar with
    the max flow problem. If not, please let me know now. In addition, I will
    omit a number of definitions such as flow conservation.

  \item I will also not go too much in depth with Ford-Fulkerson, except to say
    that the method is as follows: as long as there exists an augmenting path
    $p$ (ie. a path with positive capacity for each edge in the path): send the
    maximum amount of flow along each edge in this path. \emph{This maximum
    amount of flow is determined by the minimum residual edge capacity in the
    path, and this amount is added to the forward edges and subtracted from the
    residual edges.}

  \item Ford-Fulkerson does not specify how to find augmenting paths. This
    yields a worst-case complexity of $O(|E| * |f|)$.

  \item Edmonds-Karp is a formalization of Ford-Fulkerson which uses BFS to find
    augmenting paths. In this BFS, all edges have weight 1, but a vertex $v$
    is only reachable from $u$ if the edge $(u, v)$ has leftover capacity
    \emph{or} if $(u, v)$ is a backwards edge and there is positive flow on the
    forward edge $(v, u)$.

  \item let's see an example: *\emph{show example in
    \cref{fig:max_flow_example}}*. Please note that the state of the example
    graph is not obtainable through normal execution of Edmonds-Karp, since
    here, we have augmented using a path that is not a shortest path in the
    graph. This is simply chosen in order to illustrate flow cancellation.

  \item anyway, let's use Edmonds-Karp to finish computing this max flow (see
    \cref{sec:edmonds_karp_example}).

  \item the runtime of Edmonds-Karp is $O(|V| * |E|^2)$, since each BFS takes
    $O(|E|)$ time and there are $O(|V| * |E|)$ iterations. I will prove the
    latter statement using Christian's simplified proof.

  \item before proving the statement, I prove a necessary lemma.

  \item first, please note that I use $G_i = G_{f_i}$ and $\delta_i =
    \delta_{f_i}$ for convenience.

  \item let $G_0$ be the state of the residual graph at some point during
    iteration of Edmonds-Karp, and consider $f_1, \dots, f_k$, a sequence of
    flows for the next $k \geq 0$ iterations for which the shortest path length
    from source to target remains the same, ie. $d = \delta_0(s, t) =
    \delta_i(s, t)$ for all $i = 0 ... k - 1$.

  \item under these assumptions, the lemma states that flow is only pushed along
    edges that are forward edges in $G_0$, and that after the $k$'th iteration
    we have $\delta_k(s, t) \geq d$.

  \item the proof is by induction. The base case is $k = 0$. Trivially, flow is
    pushed through forward edges in $G_0$ for all $k = 0$ iterations. and here
    we trivially have $d = \delta_i(s, t)$ for all $i$ since the range is empty,
    and since we have $\delta_k(s, t) = \delta_0(s, t)$, we also have
    $\delta_k(s, t) \geq d$.

  \item assume that the claim holds for $k - 1$, ie. that flow has only been
    pushed through forward edges in $G_0$ for the last $k - 1$ iterations and
    that $\delta_{k - 1}(s, t) = d$.

  \item we obtain $G_k$ from $G_{k - 1}$ by augmenting along a shortest path $p$
    in $G_{k - 1}$. We know that for each edge $(u, v) \in p$, we have
    $\delta_0(s, v) \leq \delta_0(s, u) + 1$, since $(u, v)$ can be either a
    forward edge or a cancellation.\label{item:foo}

  \item however, since we know the shortest path to be $\delta_0(s, t) = d$, and
    any augmenting path is a shortest path, we must have $|p| = d$. Hence we
    must have $\delta_0(s, v) = \delta_0(s, u) + 1$.

  \item as for a shortest path $p$ in $G_k$, a similar argument to point
    \ref{item:foo} applies: for each edge $(u, v)$ in $p$ we must have
    $\delta_0(s, v) \leq \delta_0(s, u) + 1$, but we cannot say whether these
    are exclusively forward edges since we don't (necessarily) have $\delta_k(s,
    t) = \delta_0(s, t)$. Hence we can conclude $\delta_k(s, t) \geq d$, but
    nothing more.

  \item this concludes the lemma. The theorem I want to prove is then: the
    number of iterations in Edmonds-Karp is $O(|V| * |E|)$.

  \item consider $f_0, \dots, f_{k - 1}$ a maximum length sequence of flows for which
    the length of the augmenting path remains the same. In other words, we have
    $d = \delta_0(s, t) = \delta_i(s, t)$ for all $i = 0...k - 1$. Since the
    sequence was chosen to be maximal, we have $\delta_k \neq d$.

  \item by the previous lemma, we have $k \leq |E|$, since $G_0$ has at most
    $|E|$ forward edges and at least one edge is removed in each iteration of
    Edmonds-Karp (if zero edges are removed, then either we pushed flow along a
    backwards edge, which contradicts the lemma, \emph{or} we didn't fully
    satuate the new augmenting path, which is against the algorithm
    specification).

  \item By the lemma we also have $\delta_k(s, t) \geq d$, but by the previous
    assumption we also have $\delta_k(s, t) \neq d$, and hence we must have
    $\delta_k(s, t) > d$. In other words, the length of the augmenting path has
    increased by at least 1.

  \item Since the maximum possible length of an augmenting path is $|V| - 1$,
    the shortest path length will increase at most $O(|V|)$ times, and each
    increase takes at most $O(|E|)$ iterations, hence the number of iterations
    is bounded by $O(|V| * |E|)$.

  \item this concludes the proof. since each iteration involves a BFS of time
    $O(|V| + |E|) = O(|E|)$, the total runtime of Edmonds-Karp is $O(|V| *
    |E|^2)$.

\end{enumerate}

\subsubsection{Edmonds-Karp example}
\label{sec:edmonds_karp_example}

\begin{enumerate}

  \item the first iteration of Edmonds-Karp begins with a BFS from $s$.

  \item in the first iteration of this BFS, we discover nodes $a$ and $b$.

    in the second iteration, we discover no new neighbors to $a$, since $b$ is
    already found and $t$ is not reachable from $a$ because the edge $(a, t)$ is
    at full capacity. However, from $b$ we discover $t$ and the BFS terminates.
    the augmenting path is $p = \langle s, b, t\rangle$, and we can augment by
    1, since this is the minimum residual capacity on $p$.

  \item in the second iteration of Edmonds-Karp, we start a new BFS from $s$. we
    find only $l$, since the edge $(s, b)$ is at full capacity and thus $b$ is
    not reachable.

  \item in the first iteration of the second BFS, we again find $a$, but $b$ is
    not found since the edge $(s, b)$ is at full capacity and thus $b$ is not
    reachable.

    in the second iteration, we discover $b$ through the backwards edge from
    $a$. in the third iteration, we find $t$ from $b$ and the BFS terminates.

    the path is thus $p = \langle s, a, b, t \rangle$, and the minimum capacity
    on $p$ is equal to the backwards edge $(a, b)$, which has capacity 2 since
    that is the value of the flow from $b$ to $a$. we thus augment with 2 and
    cancel the 2 on edge $(b, a)$.

  \item in the third iteration of Edmonds-Karp, we once again start a new BFS
    from $s$.

  \item in the first iteration of this BFS, we find $a$, but $b$ is not
    reachable. in the second iteration, there are no reachable nodes from $a$
    and the BFS terminates unsuccessfully.

  \item since there are no more augmenting paths, Edmonds-Karp terminates.

\end{enumerate}

\subsubsection{Assumptions}

\paragraph{Value of a flow $f$} is equal to the flow leaving the source minus the flow
entering the source (we typically only send flow into the source in residual
networks):
\begin{align}
  |f| = \sum_{v \in V} f(s, v) - \sum_{v \in V} f(v, s).
\end{align}

\paragraph{Flow conservation:} for any $u$, the total flow going into $u$ is the same as the
total flow leaving $u$:
\begin{align}
  \sum_{v \in V} f(v, u) = \sum_{v \in V} f(u, v).
\end{align}

\paragraph{Capacity constraint:} for all $u,\, v \in V$ we require:

\begin{align}
  0 \leq f(u, v) \leq c(u, v).
\end{align}

For $(u, v) \not \in E$, we have $f(u, v) = c(u, v) = 0$.
