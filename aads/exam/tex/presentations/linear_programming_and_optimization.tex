\newpage
\subsection{Linear programming and optimization}

\begin{enumerate}

  \item hello. today's topic is linear programming and optimization.

  \item linear programming is a form of optimization that is used to find
    optimal solutions to systems of linear inequalities. (Many types of problems
    have equivalent linear programs, such as max flow)

  \item motivation: \TODO{?}

  \item a linear program consists of an objective function, which is to be
    minimized or maximized, as well as zero or more constraints under which the
    objective function should be optimized.


  \item here's an example:
    \begin{textred}
    \begin{alignat*}{3}
      \text{minimize } \quad& -2x_1 && +3 x_2  \quad&\\
      \text{s.t. }     \quad& x_1   && + x_2   \quad&=\quad 7\\
                            & x_1   && -2x_2   \quad&\leq\quad 4\\
                            & x_1   &&         \quad&\geq \quad0
    \end{alignat*}
    \end{textred}

  \item before we can apply an algorithm to solve the problem, we need to
    transform it to slack form, but before even that: standard form!

  \item \emph{*transform to standard form*} (see \cref{sec:to_standard_form})

  \item the general formula for the standard form LP is:
    \begin{textred}
    \begin{alignat*}{3}
      \text{maximize } \quad&\sum_{j = 1}^n c_j x_j&&\\
      \text{s.t. }     \quad&\sum_{j = 1}^n a_{ij}x_j \quad&& \leq \quad b_i   \text{ for } i = 1,\dots,m,\\
                       \quad& x_j                     \quad&& \geq \quad 0   \ \text{ for } j = 1,\dots,n.
    \end{alignat*}
    \end{textred}

  \item next, before applying the SIMPLEX algorithm, we also need to transform
    the standard form to slack form.

  \item \emph{*transform to slack form*} (see \cref{sec:to_slack_form})

  \item SIMPLEX algorithm (see \cref{sec:simplex}).

  \item the correctness of SIMPLEX is based on what we call the \emph{dual}
    problem.

  \item every LP problem has a dual. A dual to a maximization LP problem $A$ is
    a related minimization LP problem $B$ such that the two programs share the
    same optimal objective value.

  \item the dual is given by:

    \begin{textred}
    \begin{alignat*}{3}
      \text{minimize } & \sum_{i = 1}^m b_i y_i&&\\
      \text{s.t. }     & \sum_{i = 1}^m a_{ij}y_i \quad&& \geq \quad c_j   \text{ for } j = 1,\dots,n,\\
                       &                      y_i \quad&& \geq \quad 0   \ \text{ for } i = 1,\dots,m.
    \end{alignat*}
    \end{textred}

  \item \emph{on how to easily extract the dual from the primal, see below}.

  \item proof of weak duality:

    \newcommand{\x}{\bar x}
    \newcommand{\y}{\bar y}

  \item \textcolor{red}{let $\x$ and $\y$ be \underline{feasible} solutions to some primal LP
    and its dual. Then we have the following inequality:}

    \begin{textred}
    \begin{align}
      \sum_{i = 1}^m c_j \x_j \leq \sum_{j = 1}^n b_i\y_i
    \end{align}
    \end{textred}

  \item Proof: first, use the definition of the dual to rewrite $c_j$:

    \begin{textred}
    \begin{align}
      \sum_{i = 1}^m c_j  \x_j \leq \sum_{i = 1}^m \left(\sum_{j = 1}^n
      a_{ij}\y_i\right)  \x_j 
    \end{align}
    \end{textred}

    permute the summations and use the definition of the primal:
    \begin{textred}
    \begin{align}
      \sum_{i = 1}^m \left(\sum_{j = 1}^n a_{ij}\y_i\right) \x_j 
      \quad&=\quad
      \sum_{j = 1}^n \y_i \left(\sum_{i = 1}^m a_{ij} \x_j\right)\\
      \quad&\leq\quad
      \sum_{j = 1}^n \y_i b_i.
    \end{align}
    \end{textred}

\end{enumerate}

\subsubsection{Extra}

\paragraph{LP standard form requirements}

\label{sec:standard_form}
\begin{enumerate}
\item it is a maximization, not a minimization.

\item all variables have nonnegativity constraints.

\item all constraints are $\leq$-constraints (ie. no equalities or
  $\geq$-inequalities)
\end{enumerate}

\paragraph{Regular LP to standard form}
\label{sec:to_standard_form}
\begin{enumerate}
\item if the LP is a minimization: negate coefficients in the objective
  function to turn it into a maximization.

\item for each variable $x_j$ that does not have a nonnegativity constraint:
  replace each occurence of $x_j$ with $x_j' - x_j''$ and add constraints $x_j'
    \geq 0$ and $x_j'' \geq 0$.

\item for each equality constraint $f(X) = b$, replace this constraint with two
  inequality constraints $f(X) \leq b$ and $f(X) \geq b$

\item for each $\geq$-inequality, simply flip the inequality by negating the
  expression.

\end{enumerate}

\paragraph{Regular or standard form to slack form}
\label{sec:to_slack_form}

\begin{enumerate}
\item first, convert the LP to standard form if it is not already.

\item let $n$ and $m$ denote the number of variables and constraints, respectively.

\item we create $m$ new variables $x_{n + 1}, \dots, x_{n + m}$, one for each
  constraint in the standard form.

  Then, for the $i$'th constraint $f_i(x_1, \dots, x_n) \leq b$, the slack form
    has an equality constraint $x_{n + i} = b - f_i(x_1, \dots, x_n)$ and a
    nonnegativity constraint $x_{n + i} \geq 0$.

\item in addition, because we understand slack form LP's to always be
  maximizations, we typically omit the words ``maximize'' and ``subject to''; we
    use $z$ to denote the objective value; and we do not write out the implicit
    nonnegativity constraints.

\end{enumerate}

\paragraph{SIMPLEX algorithm}
\label{sec:simplex}

\begin{enumerate}

  \item given an LP in slack form, we want to repeatedly change the basic
    solution by ``pivoting'' variables to and from the basic variables (ie. the
    variables on the LHS of the constraints).

  \item to do so, we first pick any non-basic variable by which the objective
    function value \emph{increases}. Call this variable $x_e$, or the
    ``entering'' variable.

  \item next, we set all other non-basic variables to zeo and determine which
    basic variable puts the tighest bound on $x_e$. Call the selected basic
    variable $x_l$, or the ``leaving'' variable. (If some basic variable has a
    positive coefficient for $x_e$, then it will always grow with $x_e$, puts no
    bound on $x_e$, and thus is not interesting)

    Intuitively, we can think of this as examining the derivative wrt. $x_e$ --
    since essentially we are trying to find out how to make the objective
    function grow -- and seeing how far we can ``push'' $x_e$ while still
    respecting the constraints.

  \item we now want to swap $x_e$ for $x_l$. To do so, express $x_e$ in terms of
    $x_l$, make $x_e$ a basic variable, and remove $x_l$ as a basic variable.
    Replace all occurrences of $x_e$ in the constraints with the new expression
    for $x_e$.

  \item this creates a new but equivalent slack form in which the new, basic
    feasible solution is equal to or greater than the previous.

  \item eventually, if the LP has a bounded solution, all of the coefficients in
    the objective function will be negative, and by then there is no non-basic
    variable we can increase to make the objective function increase further.
    The algorithm terminates.

  \item the optimal values for the variables (both basic and non-basic) are then
    read out directly: the values for the basic variables are the slack in the
    constraints, while the values for the non-basic variables are zero.
\end{enumerate}


\paragraph{Easy dual construction}~\smallskip

I like to think of the construction of a dual $B$ from an LP $A$ as such:

Concatenate the coefficients of the objective function into a row vector and
append a dummy value to the end. Concatenate the constraint bounds $b_i$ as a
column vector to the right of the coefficient matrix. Then, append the row
vector of objective function coefficients to the bottom of this matrix.

Transpose this matrix, extract the objective function coefficients and the
constraint bounds, flip the inequalities, call it a minimzation program, and
then you have the dual. Boom!

For example, for the LP program

\begin{alignat*}{4}
  \text{maximize } \quad& 2x_1 && +3x_2     &+ 3x_3 &&   \\[4pt]
  \text{s.t. }     \quad& x_1  && + x_2    &- x_3  &&\leq 7\\
                        & -x_1 && -x_2     &+x_3   &&\leq -7\\
                        & x_1  && -2x_2    &+2x_3  &&\leq 0,
\end{alignat*}

we have the matrix:

\begin{align}
  \left[
  \begin{matrix}
    1 & 1 & -1 & 7\\
    -1 & -1 & 1 & -7\\
    1 & -2 & 2 & 0\\
    2 & -3 & 3 & \text{NIL}
  \end{matrix}
  \right]
\end{align}
where ``NIL'' is used for the dummy value. Transposing this, we get:
\begin{align}
  \left[
  \begin{matrix}
     1 & -1 &  1 &  2 \\
     1 & -1 & -2 & -3 \\
    -1 &  1 &  2 &  3 \\
     7 & -7 &  0 & \text{NIL}
  \end{matrix}
  \right]
\end{align}

and from this we extract the dual (remembering to flip the inequalities and to
change the LP to a minimization):

\begin{alignat*}{4}
  \text{minimize } \quad& 7y_1 && -7y_2  &       &&   \\[4pt]
  \text{s.t. }     \quad& y_1  && -y_2   &+y_3   &&\geq 2\\
                        & y_1  && -y_2   &-2y_3  &&\quad \geq -3\\
                        & -y_1 && +y_2   &+2y_3  &&\geq 0
\end{alignat*}
