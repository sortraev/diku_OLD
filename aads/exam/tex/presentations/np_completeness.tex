\newpage
\subsection{NP-completeness}

\begin{enumerate} \item what is a ``hard'' problem? answer: we think of ``hard''
      as something which cannot be solved in polynomial time.

  \item today we will be discussing the NP-complete class of problems. These are
    decision problems which: can be verified in polynomial time (NP) and which
    are at least as hard as all other problems in NP (NP-hard).

    \begin{textred}
      \begin{align}
        \text{NP-complete}
        \begin{cases}
          \text{NP}\\
          \text{NP-hard}
        \end{cases}
      \end{align}
    \end{textred}

  \item I am going to show NP-completeness of the k-vertex cover problem, and,
    if there is time, the decision variant of the traveling salesperson problem.

  \item first, I want to define three terms: languages, verifiability, and
    reducibility.

  \item in an effort to formalize the concept of a ``problem'', we use the
    framework of \emph{languages}, that is, strings of characters from some
    given alphabet. For our purposes, the language will be strings of zeroes and
    ones:

      \begin{textred}
      \begin{align}
        \Sigma^\ast = \{0, 1\}^\ast.
      \end{align}
      \end{textred}

  \item next is verifiability. For a lot of problems, we are able to verify a
    possible solution as fast as or faster than computing a new solution from
    scratch.

    Verifiability requires a ``certificate'' string, here called $y$, describing
    a solution to the given input string, which can be checked using a
    verification algorithm, here called $A$.

    If for a given language we know how to verify a string in time polynomial
    in the size of that string, then we say that the language belongs to NP:
      \begin{textred}
      \begin{align}
        L = \{x \in \Sigma^\ast\ :\ \exists y  \in \Sigma^\ast \ . \ |y| =
        O(|x|^c) \ .\  A(x, y) = 1\}.
      \end{align}
      \end{textred}
    For some languages, however, verifying a solution might take superpolynomial
    time, sometimes because it involves computing the solution string from
    scratch (example: TSP).

  \item the last is reducibility. Often when we analyze the complexity of a
    problem, or language, we ``reduce'' the language to be an instance of
    another language for which we know the complexity.

  \item we say that a language $L'$ is reducible to another language $L$ if an
    instance of $L'$ can be transformed into an instance of $L$. Ideally, we
    want this transformation to be possible in polynomial time, and we say that
    a language $L'$ is polynomial-time reducible to $L$ if there exists a
    polynomial-time computable function $f : \Sigma^\ast$ satisfying:

    \begin{textred}
    \begin{align}
       x \in L' \iff f(x) \in L.
    \end{align}
    \end{textred}

    if so, then we say that \textcolor{red}{$L' \leq_P L$}.

  \item \TODO{something missing here!!!!!!!!!!!!!!}

  \item example: k-vertex cover. If you are not familiar with the vertex cover
    problem \emph{and/or its decision variant}, please let me know now.

  \item I will show k-vertex cover is in NP-hard by reducing CLIQUE, which we know
    to be NP-complete, to k-vertex cover.

    What I propose is: $G$ has a clique of size $k$ \textbf{iff} its complement
    graph $\bbar G$ has a vertex cover of size $|V| - k$. In other words:
    CLIQUE($G$, $k$) has a solution \textbf{iff} VERTEX-COVER($\bbar G$, $|V| -
    k$) has a solution.

  \item assume CLIQUE($G$, $k$) has a solution $V'$ of size $k$.

  \item consider an edge \red{$e = (u, v) \in \bbar E$}. Then $e$ cannot be in
    $E$ and we have:

    \begin{textred}
    \begin{align}
      e \not \in E \implies u \not \in V' \text{ or } v \not \in V',
    \end{align}
    \end{textred}

    since $u$ and $v$ cannot both be in the clique if they do not share an edge.
    If any of them are not in the clique, then they are in the complement set:
    
    \begin{textred}
    \begin{align}
      u \in V \setminus V' \text{ or } v \in V \setminus V'
    \end{align}
    \end{textred}

    since at least one of $u$ or $v$ is in the complement set, we say that the
    edge $(u, v)$ is \emph{covered} by the complement set $V \setminus V'$.

  \item Since this holds for any $e \in \bbar E$, the complement graph $\bbar G$
    is a cover of size:

      \begin{textred}
      \begin{align}
        |V \setminus V'| = |V| - |V'| = |V| - k.
      \end{align}
      \end{textred}

  \item We will now show the backwards direction of the bi-implication. Assume
    that the complement graph $\bbar G$ has a vertex cover $V'$ of size $|V| -
    k$.

    For all $u, v \in V$, if $(u, v) \in \bbar E$ then at least one of its
    endpoints must be in $V'$:

    \begin{textred}
    \begin{align}
        \forall u, v \in V\ :\ (u, v) \in \bbar E \implies u \in V' \text{ or } v \in V'
    \end{align}
    \end{textred}

    since at least one of $u$ or $v$ must cover the edge.

    The contrapositive of this expression is:

    \begin{textred}
    \begin{align}
        \forall u, v \in V\ :\ u \not \in V' \text{ and } v \not \in V' \implies
        (u, v) \not \in \bbar E \iff (u, v) \in E,
    \end{align}
    \end{textred}

    and hence we have \red{$(u, v) \in E$}.

  \item constructing the complement graph takes \red{$O(|V|^2)$} time, hence the
    reduction takes polynomial time.\\[4pt]


  \item next up, if there is still time, is the traveling salesperson problem or
    TSP. Regular TSP is NP-hard but not NP, and hence we consider instead the
    decision variant of the problem, which is NP-complete, as we shall also see.

  \item if you are not familiar with TSP or its decision variant, please shout
    now! In any case, we define the language TSP to be:

    \begin{textred}
      \begin{align}
        \text{TSP} = \{(G, c, k)\ :\ &\ G \text{ complete},\\
                                   \ &\ c\, :\ V \times V \mapsto \mathbb R,\\
                                   \ &\ k \in \mathbb N \cup \{0\},\\
                                   \ &\ G \text{ has a tour of cost} \leq k\}.
    \end{align}
    \end{textred}

  \item let's show that TSP is NP-complete. First, we show that TSP is in NP by
    describing a possible certificate as such: the certificate is a sequence of
    $|V|$ vertices. The verification checks that the graph is complete, and if
    it is, then any sequence of distinct vertices form a simple cycle.
    Verification then checks that each vertex in $G$ is represented exactly once
    in the certificate, and that the weight of the cycle is $\leq k$.

  \item checking graph completeness takes \textcolor{red}{$O(|V|^2)$} time, and
    checking the cycle and computing its weight takes \textcolor{red}{$O(|V|)$}
    time.

  \item we will show NP-hardness by reducing HAM-CYCLE to TSP. The
    transformation of the input problem goes as such: let $G = (V, E)$. Then
    \textcolor{red}{HAM-CYCLE($G$) has a solution \textbf{iff} TSP($G'$, $c'$,
    0) has a solution}, where $G'$ is the completion of $G$ and
    \textcolor{red}{$c'(i, j) = [(v_i, v_j) \not \in E] * \infty$}. As such the
    weight of an edge in $E'$ is 0 if it is also an edge in $E$; else it is
    infinite. We could also assign a weight of 1 to edges not in $E$.

  \item assume HAM-CYCLE($G$) has a solution. Let one such hamiltonian cycle in
    $G$ be called $h$. Since $h$ contains only edges in $E$, it has weight 0.
    Since $E'$ is a superset of $E$, $h$ is also a hamiltonian cycle in $E'$.
    Hence there is a cycle of weight $\leq 0$ in $G'$, satisfying TSP($G'$,
    $c'$, 0).

    Assume now that TSP($G'$, $c'$, 0) has a solution. Let one such cycle of
    weight $\leq 0$ be called $h$. Since $h$ has weight $\leq 0$ and all edges
    are non-negative, all edges in $h$ must be 0. Then, since only edges in $E$
    have weight 0, all edges in $h$ must be in $E$ as well. Hence $G$ has a
    hamiltonian cycle.

  \item constructing the complete graph $G'$ takes \textcolor{red}{$O(|V|^2)$}
    time. Hence the reduction is polynomial time.

\end{enumerate}
