\newpage

\subsection{Randomized algorithms}


\paragraph{Disposition}

\begin{itemize}
  \item hello and intro.
  \item motivation.

  \item Las Vegas vs Monte Carlo.
  \item regular quicksort and its worst case complexity.

  \item LV example: RandQS.
  \item proof: \#comparisons made by RandQS.

  \item MC examle: RAND\_MIN\_CUT.
  \item proof: probabilistic error-bound for RAND\_MIN\_CUT.
\end{itemize}



\paragraph{Notes}

\begin{enumerate}
  \item hello.

  \item motivation for randomization.

  \item Las Vegas and Monte Carlo.

  \item regular quicksort: pivoting and complexities.

  \item LV example: RandQS -- strategy and complexity.

  \item proof: \#comparisons made by RandQS.

  \item MC example: RAND\_MIN\_CUT -- strategy.
  \item proof:probabilistic error-bound for RAND\_MIN\_CUT.
\end{enumerate}


\newpage
\subsubsection{Pseudocode}

\paragraph{RandQS}

\begin{minted}{text}
RandQS(S):
    if |S| <= 1:
        return S
    pivot = pick at uniform random an item in S;
            remove it from S and use it as pivot.
    S_l = {x in S : x <= pivot}
    S_r = {x in S : x >  pivot}
    return RandQS(S_l) ++ [pivot] ++ RandQS(S_r)
\end{minted}

\paragraph{Randomized min-cut}

\begin{minted}{text}
RAND_MIN_CUT(V, E):
    V' = {MAKE-SET(v) : v in V}

    C  = empty set
    E' = random permutation of E
    n = |V'|
    for (u, v) in E':
        p_u = find(u)
        p_v = find(v)
        if p_u != p_v:
            if n > 2 then:
                UNION(p_u, p_v)
                n -= 1
            else:
                C += (u, v)

    return C
\end{minted}
