\newpage
\section{The impact of dependence}

\subsection{Task 3.1}

We want to determine the probability of bounding $L(h)$ within some $\eps$ of
We want to bound the probability that $L(h)$ deviates from $\hat L(h, S')$ by
some $\eps$ or more.

I am eventually going to use corollary 2.5 of Hoeffding's inequality (from
Seldin's notes) to derive the bound, so to recap:
\begin{align*}
  \P\left(\E[Z_i] -\frac{1}{n} \sum_{i = 1}^n Z_i  \  \geq\   \eps\right)\
  \leq\ 
  e^{-2n\eps^2} \quad (\text{Corollary 2.5, Seldin}).
\end{align*}

Let us see how the problem fits the required form of corollary 2.5.

First, define new random variables $X'$ and $Y'$ such that:
\begin{align*}
  X'_i = X_{(i - 1)r + 1},\\
  Y'_i = Y_{(i - 1)r + 1}.
\end{align*}
Then $\hat L(h, S')$ is given by:
\begin{align*}
  \hat L(h, S') &= \frac{1}{(\sfrac{n}{r})}\sum_{i = 1}^{\sfrac{n}{r}} \ell(h(X'_i), Y'_i).
\end{align*}

Since the samples in $S'$ are iid, we also know that $\ell(h(X'_i), Y'_i)$ are
also iid. with $\ell(h(X'_i), Y'_i) \in [0, 1]$ for all $i$, (assuming zero-one
loss) and due to this independence we have $L(h) = \E[\hat{L}(h, S')]$.

With this we see that we can legally apply Corollary 2.5.

Apply a union bound and then Corollary 2.5 to obtain:
\begin{align*}
  \P\left(\exists\, h \in \mathcal H\; :\  L(h) - \hat{L}(h, S') \geq \eps
  \right)
  &\leq \sum_{h \in \mathcal H} \P\left(L(h) - \hat{L}(h, S') \geq \eps
  \right)\\
  &\leq \sum_{h \in \mathcal H} e^{-2\frac{n}{r}\eps^2}\\
  &= |\mathcal H| e^{-2\frac{n}{r}\eps^2}\ .
\end{align*}
Note that $\tfrac{n}{r}$ comes from the size of $S'$. Let now $M = |\mathcal
H|$; then the full expression of the bound is:
\begin{align}
  \P\left(\exists\, h \in \mathcal H\; :\  L(h) - \hat{L}(h, S') \geq \eps
  \right)
  \leq M e^{-2\sfrac{n}{r}\eps^2}.\label{eq:3.1}
\end{align}
Alternatively, we might want to express the bound in terms of a certainty
probability $\delta$. If we set $\delta := Me^{-2\sfrac{n}{r}\eps^2}$, then the
bound is:
\begin{align*}
  \PP{\exists\, h \in \mathcal H\; :\  L(h) - \hat{L}(h, S') \geq
  \sqrt{\frac{\text{ln}\frac{M}{\delta}}{2 \sfrac{n}{r}}}}
  \ \leq \ \delta.
\end{align*}

\subsection{Task 3.2}
\label{sec:3.2}

We now want to produce a bound based on all of the samples in $S$. However,
since the samples in $S$ are \textit{not} independent, we cannot apply
Hoeffding's directly. Instead, I shall attempt to express the bound on $\hat
L(h, S)$ in terms of the bounds on $\hat L(h, S_i)$, where all $S_i$ are subsets
of $S$ whose elements are independent.

First, let $S_i$ be the set consisting of the $i$'th sample of each 
of the $r$ households in $S$. In mathemtical terms, let $S_i$ be given by:
\begin{align*}
  S_i = \big\{(X_{(j - 1)r + i},\ Y_{(j - 1)r + i})\ |\ j \in \{1,\dots,
  \sfrac{n}{r}\}\big\}.
\end{align*}
Note that we have $\cup_{i = 1}^r S_i = S$ and $S_i \cap S_j$ for $i \neq j$
(every sample in $S$ is a member of exactly one subset  $S_i$).

In addition, please note that in Task 3.1, the result in \Cref{eq:3.1} is
derived using $S' = S_1$, but since all the samples in $S$ are drawn from the
same distribution, the result still holds if we substitute any $S_i$ for $S'$.
This will be useful later.

\bigskip

With this, I begin the derivation. The probability I want to bound is:
\begin{align*}
  \PP{\exists\, h \in \mathcal H\; :\ L(h) - \hat L(h, S) \geq \eps}.
\end{align*}

As stated, I want to transform the probability expression into something in
which I can apply the result from Task 3.1. First, note that since every sample
of $S$ is in exactly one $S_i$, the empirical loss over $S$ is equal to the mean
over empirical losses over all $S_i$, allowing us to write:
\begin{align*}
  = \PP{\exists\, h \in \mathcal H\; :\ L(h) - \frac{1}{r}\sum_{i = 1}^r \hat
  L(h, S_i) \geq \eps}.
\end{align*}

Multiplying all terms in the inequality with $r$ (this is legal since $r$ is
always positive):

\begin{align*}
  = \PP{\exists\, h \in \mathcal H\; :\ rL(h) - \sum_{i = 1}^r \hat
  L(h, S_i) \geq r \eps}.
\end{align*}

At this point I realize that if the sum over $r$ mean empirical losses $\hat
L(h, S_i)$ deviates from $rL(h)$ by at least $r\eps$, then one or more of these
mean empirical losses must deviate from $L(h)$ by at least $\eps$. In other
words, the former is a sub-event of the latter. Using $A \subseteq B \implies
P(A) \leq P(B)$, and subsequently the union bound:
\begin{align*}
  &\leq \PP{\exists\, i \in \{1, 2, \dots, r\},\, h \in \mathcal H\; :\  L(h) - \hat L(h,
  S_i) \geq \eps}\\
  &\leq \sum_{i = 1}^r \PP{\exists\, h \in \mathcal H\; :\ L(h)
  - \hat L(h, S_i) \geq \eps}
\end{align*}

Finally, the summation body resembles the RHS of the bound derived in
\Cref{eq:3.1}, but with $S_i$ substituted for $S'$. As discussed, the result in
Task 3.1 holds for all $S_i$, so I can swap it in here:
\begin{align*}
  &= \sum_{i = 1}^r M e^{-2\sfrac{n}{r}\eps^2} \\
  &= r M e^{-2\sfrac{n}{r}\eps^2}.
\end{align*}
Thus the full expression of the bound:
\begin{align*}
   \PP{\exists\, h \in \mathcal H\; :\  L(h) - \hat L(h, S) \geq \eps}\ \leq\ r M
   e^{-2\sfrac{n}{r}\eps^2},
\end{align*}
and, equivalently, in terms of $\delta$:
\begin{align*}
   \PP{\exists\, h \in \mathcal H\; :\  L(h) - \hat L(h, S) \geq
   \sqrt{\frac{\text{ln}\frac{M}{\delta}}{2 \sfrac{n}{r}}}}\ \leq\  r \delta.
\end{align*}

\subsection{Task 3.3}

The bound derived in Task 3.2 using $\hat L(h, S)$ is \textit{weaker} than the
bound found in Task 3.1 using $\hat L(h, S')$ (or rather, it is expected when
$S'$ grows large enough).

The reason we found a weaker bound in Task 3.2 is because here, we were not able
to apply Hoeffding's bound directly to $\hat L(h, S)$ -- instead, the dependence
in $S$ forced us to sum over known quantities.

If we had had independence in $S$, then we \textit{would} have been able to
apply Hoeffding's directly to $\hat L(h, S)$, and then we would have obtained
the same bound in the two tasks (again, due to the principle of empirical risk
minimzation, this is in expectation and assumes $S'$ is large enough).

\sectend
