\section{Make Your Own (10 points)}

\begin{enumerate}

  \item \textit{How do you define $\mathcal X$?}

    For each student I would collect current GPA, class attendance, and
    average hand-in page count. Assuming the Danish 7 point grading scale, that
    class attendance is described by a number between 0 and 1, and that page
    count is a natural number, we have $\mathcal X = [-3, 12] \times [0, 1]
    \times \mathbb N$.


  % \item for each prior course taken by the particular student, I would collect
  %   data on the student's lecture/class attendance, assignment point grading
  %
  %   \TODO{what?}. Thus the sample space would be \TODO{$\mathcal X = ...$}.

  \item \textit{How do you define $\mathcal Y$?}

    The label space $\mathcal Y$ would be the set of possible grades --
    assuming the Danish 7 point grading scale, we would have $\mathcal Y = \{-3,
    00, 02, 4, 7, 10, 12\}$.

  \item \textit{How do you define $\ell(y', y)$?}

    I would want incorrect predictions to be punished harder the greater the
    difference between $y'$ and $y$ -- however, to better reduce the number of
    incorrect predictions in the extremes, I would use a square loss function:
    $\ell(y', y) = (y' - y)^2$.

  \item \textit{How do you define $d(x, x')$?}


    Since the three dimensions in $\mathcal X$ have very different ranges, I
    can't use a simple Euclidean distance for $d$, since for one this would weigh
    average page count much higher than class attendance percentage. Instead, in
    order to weigh the differences in all three dimensions equally, I use
    a symmetric relative difference function, and $d$ is defined as:

    $$d(x, x') = \sqrt{\sum_{\substack{i \in \{GPA,\\ attendance,\\ pages\}}}
    \left(\frac{|x_i - x'_i|}{x_i + x'_i}\right)^2}$$

  \item \textit{How would you evaluate your algorithm in terms of $\ell(y', y)$}?

    I am not \textit{entirely} sure that I interpret the question correctly.
    However, I would probably evaluate model accuracy using MSE such that bad
    predictions weigh heavier. Since $\ell$ is a square loss function, the MSE
    is:

    $$
    \frac{1}{n}\sum_{i = 0}^n \ell(y'_i, y_i)
    $$


  \item \textit{Do you expect any issues after deployment?}

    The biggest problem I can imagine is the issue of new students for whom
    there exists no prior data. To alleviate this, I would generate random data
    based on other students (with added random noise, of course).

\end{enumerate}

\sectend
