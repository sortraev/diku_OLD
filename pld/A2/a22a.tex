\subsection{A.2.2.a)}
If we assume that that in the final statement, the two addition operations are
evaluated left to right, then the program prints "17" under all three parameter
passing methods.

\smallskip

\textbf{Why?}

\medskip

If the two additions are evaluated left to right, then the temporary result of
\ms{u + v} will be stored in a temporary variable, which is not going to be
modified by the call to \ms{f} (when call-by-reference or call-by-value-result
is used; under call-by-value all of this is irrelevant). \smallskip

\ms{f(u, v)} is, of course, always equal to \ms{f(4, -2) = 15} (since, as
explained, \ms{u} and \ms{v} are only modified inside \ms{f}). Thus the final
print statement is always equivalent to: \ms{print(4 + (-2) + f(4, -2))}, which
is equivalent to \ms{print(17)}.

\sectend

\subsection{A.2.2.b)}


\begin{itemize}
  \item \textit{i. What are the results of \ms{f(1)} and \ms{g(1)} as a function
    of \ms{n} under call by need and call by name?}
\end{itemize}

\begin{minted}{text}
f(1) under call-by-need: 2^(n + 1)
f(1) under call-by-name: 2^(n + 1)
g(1) under call-by-need: 2^(n + 1)
g(1) under call-by-name: 2^(n + 1)
\end{minted}


\begin{itemize}
  \item \textit{ii. In each of the four cases, how many additions are executed?}
\end{itemize}

\begin{minted}{text}
f(1) call by need: n + 1 additions
f(1) call by name: n + 1 additions
g(1) call by need: n + 1 additions
g(1) call by name: 2^(n + 1) - 1
\end{minted}

\Sectend
