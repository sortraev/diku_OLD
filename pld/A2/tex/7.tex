
\newpage

\section{A2.7}

\subsection{A2.7.a}

\begin{itemize}
  \item \emph{Give an example of a function that cannot be written because of
    the requirement of array types, but which could still be written in C\#.}
\end{itemize}

A good example is the ``dynamic tables'' examples that we all know and love from
CLRS, in which the size of an array is doubled whenever it is filled up. Fixed
upper bounds on statically declared Pascal arrays means the array can only be
doubled a certain number of times - however, Pascal also supports dynamic arrays
which can be used here.

\sectend

\newpage
\subsection{A2.7.b}

\begin{itemize}
  \item \emph{Suggest how to impose a bound on the length of arrays in C\# that
    would be checked by the type system.}
\end{itemize}

The easiest solution is to create a new class for size-bounded arrays. In the
snippet below I present a simple implementation for arrays of arbitrary number
of dimensions:

\begin{minted}[highlightlines={11-21}]{csharp}
public class MyBoundedArray<T> {
  private readonly int[] bounds;
  private Array array;
  
  public MyBoundedArray(int[] bounds) {
    this.bounds = bounds;
    array = null;
  }


  public void initialize(int[] dims) {
    if (dims.Rank != bounds.Rank) {
      throw new ArgumentException("invalid dimensions");
    }
    for (int i = 0; i < bounds.Rank; i++) {
      if (dims[i] > bounds[i]) {
        throw new ArgumentException("Dimension " + i + " exceeds bound");
      }
    }
    array = Array.CreateInstance(typeof(T), dims);
  }
  
  public void set(int[] idxs, T val) { array.SetValue(val, idxs);       }
  public T    get(int[] idxs)        { return (T) array.GetValue(idxs); }

  public int[] getBounds()  { return bounds;       }
  public int getRank()      { return array.Rank;   }

}
\end{minted}

Given an array of upper bounds for each dimension, a new array is
\emph{declared} (but not yet initialized) using the class constructor.

The important function here is then \ms{initialize(dims)} (highlighted above),
which (re-)initializes a new array in place of the old with the given dimensions
\emph{if} they are within the declared bounds; if they are not, or if an invalid
number of dimensions are given, an exception is thrown.

\sectend

\subsection{A2.7.c}

\begin{itemize}
  \item \emph{Are there good reasons why Pascal array types are the way they
    are?}
\end{itemize}

Yes! In Pascal, statically sized arrays allows the programmer to use a different indexing base.

\medskip

To illustrate how this works and why it is useful, consider the problem of
encoding a discrete two-dimensional real-valued function over a plane of
discrete points centered around the origin (ie. the point $(0, 0)$). For the
purposes of this example, consider the discrete plane of points $(i, j)$ which
satisfy $-50 <= i, j <= 50$ \footnote{More formally the set $\{(i, j)\ :\ -50
\leq i \leq 50 \land -50 \leq j \leq 50\}$.}.

\smallskip

In most programming languages, we would use an array of dimensions
\ms{[101][101]}. The origin would have index \ms{[50][50]} in this 2D array,
while an arbitrary point $(x, y)$ would have index \ms{[x + 50][y + 50]}.
Working with this 2D plane quickly becomes tedious!

\medskip

However, using static Pascal arrays, we can declare the plane as such:

\begin{minted}{pascal}
myPlane: array [-50 .. 50, -50 .. 50] of Real
\end{minted}

This specifies \ms{myPlane} to be a $101 \times 101$ array (since the intervals
are inclusive) of reals, and that its indices range from $-50$ to $+50$ in
either direction, rather than from $0$ to $100$ as discussed above. Now, any
given point $(x, y)$ in the plane (for which $-50 \leq x, y \leq 50$) has
location \ms{myPlane[x][y]} - even for negative coordinates.

\Sectend
