\newpage

\section{A2.5}

\resub{Here comes my resub of 2.5.b. New stuff is written in red.}

\subsection{A2.5.b \resub{- resubmission}}

\begin{itemize}
  \item \emph{Suggest how we should type check function declarations and
    function calls if we extend C with the previously discussed construct. The
    extend type system should handle cases such as the one presented in the
    assignment text.}
\end{itemize}

This question is a little ambiguous. It is not stated whether \ms{T1, ...,  T7}
are all known at compile-time (as is required in C), or if the type system
should support type inference. In the former case the answer is easy, because
the C type system already supports everything necessary to type check the
examples ;)

\smallskip

In the latter case, however, we would simply need to implement some sort of type
inference system, since if the type signature of each function in a program can
be inferred, then it must be well-typed.

\bigskip

In any case, we could type check functions as follows:

\smallskip

Given a function $f$ with $m$ parameters $(p_0,\ \dots,\ p_{m - 1})$ and function body
$f_{\text{body}}$, we first look up its type signature in the type environment.
If we find its type signature to be $f\; ::\; (t_0,\ ...,\ t_{n_1}) \to
t_{\text{ret}}$ for some $n$ (which may or may not be equal to $m$), then we type check
the function $f$ using the following steps (in order):

\begin{enumerate}
  \item type check function arity (ie. that the function takes as many
  parameters as the type signature states): assert that $n = m$.

  \item type check function parameters: assert that $p_i = t_i$ for $i \in \{0,
    \dots, n\}$.

  \item type check $f_\text{body}$: for each branch in the function body, assert
    that the value of this branch is $t_\text{ret}$.
\end{enumerate}

\resub{
\begin{itemize}
  \item In the feedback received I was asked to determine the types of T1, ...,
T7, and to explain what must hold for these in order for the program to be
well-typed.
\end{itemize}
}

\medskip

\resub{First, I write out the function types in Haskell-like type notation:}


\begin{minted}[linenos=false]{haskell}
f :: T2 -> T3 -> T1
h :: T5 -> T4
j :: T7 -> T6
\end{minted}

\resub{
First off, from the definition of \ms{f}, I can infer that \ms{T2} is a function
which takes a parameter of type \ms{T3} and returns a \ms{T1}, so I add the
binding \ms{T2 = T3 -> T1}.
}

\medskip

\resub{
Secondly, from the definition of \ms{j}, I know that \ms{f(h, v) :: T6}, which
gives me the type binding \ms{T6 = T1} since I know \ms{f} to have return type
\ms{T1}.
}

\medskip

\resub{
Since \ms{h} and \ms{v} - which have types \ms{T5 -> T4} and \ms{T7},
respectively - are passed as parameters to \ms{f}, which has type \ms{T2 -> T3
-> T1}, I can also add the type bindings \ms{T2 = T5 -> T4} and \ms{T3 = T7}.
}

\medskip

\resub{
Lastly, since both \ms{T2 = T5 -> T4} and \ms{T2 = T3 -> T1}, I can infer that
\ms{T5 = T3} and \ms{T4 = T1}.
}

\medskip

\resub{
In conclusion, I have inferred the following type bindings:
}


\begin{minted}[linenos=false]{haskell}
T2 = T3 -> T1
T2 = T5 -> T4
T6 = T1
T7 = T3
T5 = T3
T1 = T4
\end{minted}

\resub{
which can be reduced to:
}

\begin{minted}[linenos=false]{haskell}
T5 = T7 = T3
T4 = T6 = T1
T2 = T3 -> T1
\end{minted}

\resub{
As such, for the program to be well-typed, these are the restrictions which must
hold. The function types given earlier can be redeclared as:
}

\begin{minted}[linenos=false]{haskell}
f :: T2 -> T3 -> T1
h :: T3 -> T1
j :: T3 -> T1
\end{minted}

\resub{
In a language with currying, the function types could of course be given as:
}

\begin{minted}[linenos=false]{haskell}
f :: T2 -> T2
h :: T2
j :: T2
\end{minted}
