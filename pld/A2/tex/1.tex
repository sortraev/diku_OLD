\section{A2.1)}


\subsection{A2.1.a)}

\begin{itemize}
    \item \emph{Find an S-expression containing a \ms{lambda} which
        evaluates to \ms{0}, but which, when changing that \ms{lambda} to
        \ms{lambdaD}, evaluates to \ms{1}.}
\end{itemize}


Below is a snippet of \ms{task1.le} of my code hand-in, showing the S-expression
I found, as well as the version with one \ms{lambda} changed to a \ms{lambdaD}:

\begin{minted}{lisp}
; the value of this S-expression is 0.
(if (define x 0) ((lambda (f x) (f)) (lambda y x) 1))
    
; this S-expression is identical to the expression above, except the innermost
; 'lambda' has been changed to 'lambdaD'. the value of this S-expression is 1.
(if (define x 0) ((lambda (f x) (f)) (lambdaD y x) 1))
\end{minted}


\subsection{A2.1.b)}
\begin{itemize}
    \item \emph{Is it possible to find an S-expression as above but where the
        \ms{lambda} in question is a closed lambda expression? If possible, give
        an example; if not, argue why not.}
\end{itemize}

I would argue that it is impossible. Here is why:

\smallskip

If a particular lambda -- let's call it \ms{foo} -- is a closed lambda
expression, then the value of any variable occuring in the \ms{foo}'s body
\textbf{must} be associated with the corresponding lambda parameter variable -
and thus a variable reference made within the body of \ms{foo} can never point
to a variable outside of the scope of \ms{foo}. 

\smallskip

Since there is no difference between static and dynamic scoping
when there is only one scope of reference, changing \ms{lambda} to \ms{lambdaD}
can never change the value of a closed S-expression.

\sectend

\subsection{A2.1.c)}

\begin{itemize}
    \item \emph{Do dynamically scoped functions in PLD-LISP use shallow or deep
        binding?}
\end{itemize}

Deep binding. Here is why:

\smallskip

When a pattern $p_i$ is matched to an expression $e_i$ in \ms{matchPattern}, each
value in $e_i$ is bound to a variable (or pattern) in $p_i$ and these are pushed
to the front of the environment (ie. variable stack). This is seen in the below
snippet of \ms{RunLISP.fsx} (relevant line highlighted):

\begin{minted}[highlightlines={5}]{fsharp}
and tryRules rules pars localEnv =
  match rules with
  | Cons (p, Cons (e, rules1)) ->
      match matchPattern p pars with
      | Some env -> eval e (env @ localEnv)
      ...
  ...
\end{minted}

\newpage

\subsection{A2.1.d)}

\begin{itemize}
    \item \emph{Does shallow vs. deep binding give any observable difference in
        PLD-LISP?}
\end{itemize}

No, it does not. 

\smallskip

There are very rarely observable differences between using shallow and deep
binding - according to PLDI, this can only occur when it is possible to take the
address of a variable, and therefore also when using call-by-reference. However,
this is irrelevant, since PLD-LISP uses pass-by-value as evidenced by the
highlighted lines of code in the snippet below:

\begin{minted}[highlightlines={5, 10}]{fsharp}
let rec eval s localEnv =
  match s with
  ...
  | Cons (e1, args) -> // function application
      applyFun (eval e1 localEnv, evalList args localEnv, localEnv)
...
and evalList es localEnv =
  match es with
  ...
  | Cons (e1, es1) -> Cons (eval e1 localEnv, evalList es1 localEnv)
\end{minted}


\Sectend
