\newpage

\section{A2.2}

% \subsection{A2.2.a}
%
% \begin{itemize}
%   \item \emph{What does the program print under call-by-reference,
%     call-by-value, and call-by-value-result?}
% \end{itemize}
%
% \emph{In my answer, I will assume that that in the final \ms{print} statement, the
% operands to the two additions are evaluated \textbf{left to right}, and that the
% additions are also evaluated left-to-right.}
%
% \smallskip
%
% Under this assumption, the program prints ``17'' for \textbf{all three}
% parameter passing methods. Here is why:
%
% \medskip
%
% If the two additions are evaluated left to right, then the temporary result of
% \ms{u + v} will be stored in a temporary variable, which is not going to be
% modified by the call to \ms{f} (when call-by-reference or call-by-value-result
% is used; under call-by-value all of this is irrelevant). \smallskip
%
% \ms{f(u, v)} is, of course, always equal to \ms{f(4, -2) = 15} (since, as
% explained, \ms{u} and \ms{v} are only modified inside \ms{f}). Thus the final
% print statement is always equivalent to: \ms{print(4 + (-2) + f(4, -2))}, which
% is equivalent to \ms{print(17)}.
%
% \sectend

\subsection{A.2.2.b.ii) \resub{- resubmission}}


% \begin{itemize}
%   \item \textit{i. What are the results of \ms{f(1)} and \ms{g(1)} as a function
%     of \ms{n} under call by need and call by name?}
% \end{itemize}
%
% All four combinations of functions and parameter passing methods return the same
% value of $2^{n + 1}$:
%
% \begin{minted}[linenos=false]{text}
% f(1) under call-by-need: 2^(n + 1)
% f(1) under call-by-name: 2^(n + 1)
% g(1) under call-by-need: 2^(n + 1)
% g(1) under call-by-name: 2^(n + 1)
% \end{minted}
%
% This is as expected, since there can be no observable (ie. semantic) difference
% between call by name/need in a functional language without side effects.

\begin{itemize}
  \item \textit{ii. In each of the four cases, how many additions are executed?}
\end{itemize}

\resub{In my feedback I was told that under call-by-name, the let bindings in
\ms{f} are evaluated like function calls. Here is my resubmission with the
answers changed accordingly. New stuff is in written in red.}

\medskip

In the program there are {(n + 1)} distinct additions - \resub{and this is the
number of additions executed under call-by-need, since here, the result of let
bindings are memoized and not reevaluated, whereas with call-by-name, a binary
recursion tree is built in evaluating the let bindings. The below table is
changed accordingly.}

\medskip


At the same time, the function \ms{g} builds a binary recursion tree of height
\ms{(n + 1)} - not simply \ms{n} because we also have the last layer of \ms{y =
gn() + gn()}. However, we must subtract one to account for the root of the tree.
Therefore, the number of additions

\begin{minted}[highlightcolor=red, highlightlines={2}, escapeinside=||, linenos=false]{text}
f(1) call by need: n + 1 additions
f(1) call by name: 2^(n + 1) - 1 additions
g(1) call by need: 2^(n + 1) - 1 additions
g(1) call by name: 2^(n + 1) - 1 additions
\end{minted}

\medskip

\noindent \emph{Why does call by need not decrease the number of additions to
$O(n)$ for} \ms{g}?

\smallskip

Answer: call-by-need only applies to function parameters, and it is wrong to
think of call-by-need as a sort of memoization. However, if we were to rewrite
\ms{g} as such:

\begin{minted}{haskell}
double x = x + x

g(x0) = let g1()  = x0 + x0         in
        let g2()  = double (g1 ())  in
        ...
        let y     = double (gn ())
        in  y
\end{minted}

Then the \ms{x} in \ms{double} would only be evaluated once per call to
\ms{double}, even though it appears twice in the function body, and thus we
would have the linear recursion tree and $O(n)$ complexity (whereas with
call-by-name we would still have the binray recursion tree and exponential
complexity).

\Sectend
