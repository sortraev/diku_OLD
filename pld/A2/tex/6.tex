\newpage
\section{A2.6}

\begin{itemize}
  \item \emph{Give (at least) one example of slack in your favorite statically
    typed programming language that \emph{does not} involve conditional
    statements or conditional expressions.}
\end{itemize}

My favorite statically typed programming language is, of course, C, and the kind
of slack in the type system which I would like to discuss is that of dealing
with different pointer types.

\smallskip

As we all know, C represents pointers as unsigned integers - when doing pointer
arithmetic, the compiler uses the types of pointers to determine how much to add
to a particular pointer (eg. if \ms{xp} has type \ms{int32_t*}, then the
statement \ms{xp += 1} will actually increment \ms{xp} by 4 byte).

\medskip


Below snippet shows a dummy program computing the size of an allocation by
subtracting the distance between two pointers created by \ms{malloc()}:


\begin{minted}[highlightlines={15}]{c}
#include <stdlib.h>
#include <stdint.h>

#define ABS(x) (((long) (x)) < 0 ? -(x) : (x))
#define MAX(x, y) ((x) > (y) ? (x) : (y))

int main() {

  // first, let's allocate some memory ...
  size_t   request_size = 256 * sizeof(int);
  int      *my_int_arr  = malloc(request_size);
  unsigned *my_uint_arr = malloc(request_size);

  // compute the distance between the two allocations.
  int difference = my_int_arr - my_uint_arr;
  int distance   = ABS(difference);

  // to verify success of one of the allocations, assert that the
  // distance between the two arrays is at least the size of the request.
  int allocation_successful = distance >= request_size;

  return allocation_successful;
}
\end{minted}

The program is \textbf{incorrect} for a number of reasons, but that is not the
matter of this analysis, because the program actually fails to compile due to a
type checking error caused by line 15 (highlighted above). The type checker
complains about invalid operands to binary minus since \ms{my_int_arr} is an
a signed int pointer, and \ms{my_uint_arr} is an unsigned int pointer.

\medskip

However, since both \ms{int} and \ms{unsigned} are both 32 bit large
\footnote{Actually, \ms{int} and \ms{unsigned} are not guaranteed to be 32 bit,
but they \textit{are} guaranteed to be of the same size on a given machine.},
the subtraction cannot produce errors, and type checker should be able to
discern this since it knows the sizes of both types.

\medskip

In any case, it is probably a good idea to be just a little conservative about
pointer arithmetic - even for a language like C whose compilers does no take
much responsibility for the programs they compile.

\Sectend
