\newpage

\section{A2.5}

\subsection{A2.5.a}

\begin{itemize}
  \item \emph{What should T be in the example given in the assignment text?}
\end{itemize}

\ms{T} should be a function type from int to int, ie. \ms{int -> int}, since the
input array is an int array and the function operates on each element and writes
back to the original array.

\begin{itemize}
  \item \emph{Suggest new type constructs that we should add to the type system
    for C in order to achieve higher order functions.}
\end{itemize}

This task is \emph{highly} ambiguous because everything that is discussed in the
assignment text is \textbf{already supported} by the C type system, and any
(reasonable) C compiler (eg. gcc or clang) would, in fact, reject the program if
\ms{myarray} was anything but an integer array.

\medskip

All this aside, there is undoubtedly \textbf{a lot} of room for improvement in
the C type system for function types. In the following, I will discuss some
features/constructs that would benefit the C type system.

\paragraph{Safety}~\smallskip

For one, a lot of unsafety arises from the fact that C only supports function
\textit{pointers} and the fact that in C, any pointer can be typecast to a
pointer of any different type. For this reason, it would be important to
implement a distinct function type construct.


\paragraph{Facilitating optimization}~\smallskip

Secondly, since one of the main benefits of programming in C is efficient memory
utilization, it might be beneficial for the function type construct to include
information about whether a particular function accesses global memory or not,
and whether it both reads and writes or simply reads - in other words, whether
or not the function can have side effects outside its own scope - as this would
\emph{significantly} increase the possibilities for automatic (memory)
optimizations.


\paragraph{Ease of expression}~\smallskip

As is, the syntactic side of the C function (pointer) type construct is baffling
to say the least. As an example, consider the below Haskell program:

\begin{minted}{haskell}
h :: Int -> Int -> Int -> Int
h a b c = a + b * c

g :: Int -> Int -> Int -> (Int -> Int)
g a b c = h a (b + c)

f :: Int -> Int -> (Int -> Int -> (Int -> Int))
f a b = g (a + b)
\end{minted}

Here, \ms{f} has the type \ms{Int -> Int -> (Int -> Int -> (Int -> Int))}.
%\footnote{\emph{A function which takes two ints and returns a function which
%takes two ints and returns a function which takes one int and returns an int.}}
This is a rather obscure function type, but nevertheless it is not a
particular complex type signature to discern for a human programmer.

\medskip

In C function type notation, however, \ms{f} has the type signature:

\begin{minted}[linenos=false, frame=none]{c}
int (* (* (*)(int, int))(int, int))(int)
\end{minted}

These kinds of ridiculous function type signatures make it very hard to
implement and reason about higher-order functions in C. It would be a very good
idea to design a more concise, expressive function type construct for the C type
system, perhaps similar to that of Haskell. 

\smallskip

However, unlike Haskell, C does not support partial application (or currying),
and we should make this distinction clear in the syntax for the new function
types. In general, the type signature for some function \ms{g} of $n$ parameters
should in the source code be declared by:

\begin{minted}[linenos=false, frame=none]{c}
  foo :: (t_0, t_1, ..., t_n) -> t_return
\end{minted}

In the case of the function \ms{f} as described above, the type signature would
in this system be:

\begin{minted}[linenos=false, frame=none]{c}
  f :: (int, int) -> ((int, int) -> ((int) -> int))
\end{minted}

As such, the new syntax has some redundant parenthesization, but it is arguably
a more concise syntax than the current syntax for function pointer types.

\sectend

\newpage

\subsection{A2.5.b}

\begin{itemize}
  \item \emph{Suggest how we should type check function declarations and
    function calls if we extend C with the previously discussed construct. The
    extend type system should handle cases such as the one presented in the
    assignment text.}
\end{itemize}

This question is a little ambiguous. It is not stated whether \ms{T1, ...,  T7}
are all known at compile-time (as is required in C), or if the type system
should support type inference. In the former case the answer is easy, because
the C type system already supports everything necessary to type check the
examples ;)

\smallskip

In the latter case, however, we would simply need to implement some sort of type
inference system, since if the type signature of each function in a program can
be inferred, then it must be well-typed.

\bigskip

In any case, we could type check functions as follows:

\smallskip

Given a function $f$ with $m$ parameters $(p_0,\ \dots,\ p_{m - 1})$ and function body
$f_{\text{body}}$, we first look up its type signature in the type environment.
If we find its type signature to be $f\; ::\; (t_0,\ ...,\ t_{n_1}) \to
t_{\text{ret}}$ for some $n$ (which may or may not be equal to $m$), then we type check
the function $f$ using the following steps (in order):

\begin{enumerate}
  \item type check function arity (ie. that the function takes as many
  parameters as the type signature states): assert that $n = m$.

  \item type check function parameters: assert that $p_i = t_i$ for $i \in \{0,
    \dots, n\}$.

  \item type check $f_\text{body}$: for each branch in the function body, assert
    that the value of this branch is $t_\text{ret}$.
\end{enumerate}


\Sectend
