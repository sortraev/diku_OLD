\newpage

\section{A2.3}

\begin{itemize}
  \item \textit{For each of the three functions \ms{concat}, \ms{subtract}, and
    \ms{altSum}, determine whether this function is a map homomorphism, a
    homormorphism, or not a homomorphism.}
\end{itemize}


In classifying the functions, I will first check for map homomorphisms, then
regular homomorphisms, and if both tests fail, then simply declare that the
function is neither.

\subsection{A2.3.a - \texttt{concat}}

\ms{concat} is a \textbf{map homomorphism}. To show this, I need to show the
following three properties:

\begin{align*}
  \text{concat}\ (x \concat y) &= \text{concat} x  \concat \text{concat} y \\[4pt]
  \text{concat}\ [x]      &= g(x)\\[4pt]
  \text{concat}\ []       &= e
\end{align*}

where $g(x)$ is some function and $e$ is the neutral element for list
concatenation. The latter is easy, since I know from PLDI that the neutral
element for list concatenation is the empty list, ie. $e = []$.

\medskip

Since $\text{concat}\ [x] = x$, the second property is satisfied by setting $g =
\text{id}$ (the identity function).

\medskip

Thirdly, I need to show the first property. This one is a little more tricky.
I intend to show the property by intuition and argue that it is sufficient.

\smallskip

Let \ms{x} and \ms{y} be 2D lists of some element type (which may or may not in
itself be a composite type); then \ms{(x ++ y)} is a list containing \ms{x}'s
sublists followed by \ms{y}'s sublists, and then \ms{concat (x ++ y)} is a
list containing the elements of \ms{x}'s sublists followed by the elements of
\ms{y}'s sublists.

\medskip

On the left hand side, \ms{concat x} and \ms{concat y} are two lists containing
the elements of \ms{x}'s sublists, and \ms{y}'s sublists, respectively.
Concatenating these with \ms{(++)} yields a list containing the elements of
\ms{x}'s followed by the elements of \ms{y}'s sublists.

\medskip

We see that the two are equal. All three properties are satisfied, and thus
\ms{concat} is a map homomorphism.

\sectend


\subsection{A2.3.b - \texttt{subtract}}

\ms{subtract} is \emph{not} a map homomorphism, because it has type \ms{[int] ->
int} and is thus not a function from lists to lists.

\medskip

It is also disqualified from being a regular list homomorphism as there exists
no binary operator $\odot$ which satisfies:

\begin{align*}
\text{subtract}\,(x \concat y) = \text{subtract}\,(x) \ \odot \ \text{subtract}\, (y)
\end{align*}


In conclusion, \ms{subtract} is neither a map homomorphism nor a regular
homomorphism.

\sectend

\subsection{A2.3.c - \texttt{altSum}}

\ms{altSum} is also \emph{not} a map homomorphism, since it has type \ms{[int]
-> (int -> int)} (ie. takes a list of ints and returns a function which takes an
int and returns an int).

\medskip

It is, however, a regular list homomorphism. If we set $g$ to be \ms{altSum}
itself, $\odot$ to be function composition, and $e$ to be the identity function
(since that is the neutral element for function composition), then the
properties for a regular homomorphism are satisfied. Let's see how!

\bigskip

Consider the case where $n$ is even. Then $\text{altSum} [x_1, \dots, x_n]$
returns the function:

$$
f(y) = x_1 - x_2 + x_3 - ... - x_n + y
$$

Let then $f_1$ and $f_2$ be two functions given by:

\begin{align*}
  f_1(y) &= x_1 - x_2 + ... + x_{i - 1} - x_i + y\\[4pt]
  f_2(y) &= x_{i + 1} - x_{i + 2} + ... + x_{n-1} - x_n + y
\end{align*}

where $i$ is some even integer between $1$ and $n$. Then the composition of $f_1$
and $f_2$ is precisely $f$, as seen below:

\begin{align*}
  (f_1 \circ f_2)(y) &= x_1 - x_2 + \dots + x_{i - 1} - x_i + (x_{i + 1} - x_{i + 2}
  + \dots + x_{n-1} - x_n + y)\\[4pt]
                   &= f(y)
\end{align*}

Here I have only showed it for the case where both $n$ and $i$ are even, but the
same also holds when both numbers are odd and when one is odd and one is even.

\medskip

In conclusion, by setting $\odot = \circ$ (function composition), $g =
\ms{altSum}$, and $e = \text{id}$ (identity function; neutral element for
function composition), the properties for a regular list homomorphism are
satisfied.

\Sectend
