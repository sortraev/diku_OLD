\newpage
\subsection{A4.3.c}

\begin{itemize}
  \item \emph{Write a short guide to usage of your implementation. Show example
    rolls by generating at least 10 rolls of each example Troll expression given
    in the assignment text.}
\end{itemize}

\subsubsection{User guide}

To boot up my embedded Troll interpreter implementation, use \ms{make run} from
within my code hand-in directory; alternatively, load the module manually by
evaluating \ms{(load Troll)} from the PLD-LISP interpreter.

\medskip

The API of my evaluator is simply the function \ms{troll}.
To evaluate some Troll expression \ms e (where \ms e is given in the embedded
syntax), use \ms{(troll e)}. To roll \ms e multiple times - say, \ms n times,
each with different random results; use instead \ms{(troll n e)}.

\subsubsection{Example rolls}

In below snippet, I evaluate the example rolls given in assignment text,
followed by 7 rolls of the the example Troll expression discussed earlier:

\begin{minted}[linenos=false]{lisp}
> ; 10 rolls of `d12+d8`
  (troll 10 '((d 12) + (d 8)))
> = ((10) (13) (13) (8) (14) (12) (2) (9) (17) (7))

> ; 10 rolls of `min 2d20`
  (troll '(min (2 d 20)))
> = ((5) (3) (7) (13) (9) (12) (3) (4) (13) (8))

> ; 10 rolls of `max 2d20`
  (troll 10 (max (2 d 20)))
> = ((11) (20) (17) (8) (10) (20) (17) (20) (19) (18))

> ; 10 rolls of `sum largest 3 4d6`
  (troll 10 '(sum (largest 3 (4 d 6))))
> = ((12) (14) (14) (9) (4) (15) (11) (10) (13) (8))

> ; 10 rolls of `count 7 < 10d10`
  (troll 10 '(count (7 < (10 d 10))))
> = ((3) (5) (3) (3) (1) (3) (3) (3) (1) (4))

> ; 10 rolls of: `choose {1, 3, 5}`
  (troll 10 '(choose (1 , 3 , 5)))
> = ((5) (5) (3) (5) (3) (5) (3) (1) (3) (1))

> ; 10 rolls of `accumulate x := d6 while x > 2`
  (troll 10 '(accumulate x := (d 6) while (x > 2)))
> = ((3 4 4 1) (2) (6 4 5 3 2) (4 1) (6 4 4 1)
      (5 6 5 5 4 4 5 3 6 6 3 3 3 6 3 1) (3 6 4 1) (4 2)
      (3 5 2) (4 3 3 4 6 4 5 6 2))

> ; 10 rolls of `x := 3d6 ; min x + max x`
  (troll 10 '(x := (3 d 6) ; ((min x) + (max x))))
> = ((3) (11) (9) (3) (6) (9) (8) (6) (7) (5))

> ; 7 rolls of the example discussed in the report wrt. parenthesization
  (troll 7 '(sum (least (19 + 1) (accumulate x := ((d 42) d (d (d 1337)))
  while (800 < (sum (largest 19 x)))))))
= ((782) (1250) (175) (1212) (418) (170) (277))
\end{minted}

To reproduce these example rolls, use \ms{make example} from within the code
hand-in. The output will not include the same annotations as above snippet.


\newpage
\subsubsection{Testing}

In addition to the example rolls, I want to test my program. Ideally, I would
write automated and reproducible unit tests, but this is only feasible for Troll
expressions with no randomness involved; for dice rolling (ie. \ms d $e$ and
$e_1$ \ms d $e_2$) and the \ms{choose} operator, manual testing will have to
suffice \footnote{In a different setting or scope we might have used eg.
quickcheck testing.}. I also perform manual testing of the \ms{accumulate}
expression, since if there is no randomness involved then we can only have one
accumulation \emph{or} infinite looping.

\paragraph{Manual testing of random operators}~\smallskip

Below are the test cases used in my manual testing of random operators. I hope
that the prompt print-out speaks for itself.

\begin{minted}{lisp}
> (troll '(d 400))    ; single die roll.
= (276)               
> (troll '(d 1))      ; single one-sided die roll.
= (1)                 
> (troll '(d -12))    ; invalid single die roll.
= ()

> (troll '(23 d 400)) ; multi dice roll.
= (197 341 318 240 79 387 121 318 121 113 287 86 131 211 12 216 116 53 166 120 
   202 11 112)
> (troll '(1 d 19))   ; multi dice roll, but just one die.
= (8)
> (troll '(0 d 19))   ; no dice!
= ()

> (troll '(choose (1 , 2 , 3 , 4 , 5))) ; simple choose expression.
= (2)
> (troll '(choose (1843 d 5000000)))    ; choose among many.
= (4533089)                             
> (troll '(choose (d 1)))               ; choose from singleton collection.
= (1)                                   
> (troll '(choose (50 < (4 d 19))))     ; choose from empty collection.
= ()

> (troll '(accumulate x := (65 < (10 d 100))  ; simple accumulate expression
                      while (1 < (count x)))) ; which runs for multiple iterations.
= (91 76 81 98 82 66 100 75 94 78 99 96 75 91 100 77 87 90 83 80 97 79 87 87 86
   66 77 97 95 83 76 80 82 96 74 88 93 83 97 100 90 81 66 76 99 98 95 66 73 82
   97 97 74 90 93 76 88)
\end{minted}

The results seem reasonable - all values are within expected ranges, and the
sizes of all produced collections are as expected. To reproduce, use \ms{make
manual_tests} from within the code hand-in.

\paragraph{Unit testing}~\smallskip

I write small and simple unit tests for the rest of the constructors in the
subset of Troll. I attempt to also test various features of the Troll language,
such as correct flattening of collections in collections, correct variable
scope, and variable overshadowing, as well as a small number of edge case use
cases, such as empty collections, illegal dice rolls, and unknown variables.

\smallskip

I do \emph{not} write any tests of syntactically invalid programs.

\medskip

All my test cases run successfully. Test cases can be viewed in the
\ms{testing.le} file of my code hand-in, and test results can be reproduced with
\ms{make unit_tests} from inside the code hand-in directory.

\medskip

To reproduce all tests, use instead \ms{make test}.

\sectend

\subsection{A4.3.d}

\begin{itemize}
    \item \emph{If there is anything missing, discuss how this might be
      resolved.}
\end{itemize}

I am fairly certain that my implementation correctly handles all expressions in
the subset of Troll given in the assignment text. However, I for one cannot
guarantee that my implementation is bugfree.

\smallskip

More importantly, my implementation only works under the assumption of full
parenthesization of the embedded syntax. This is quite a huge assumption! To
remedy this, I would look into implementing an actual parser for the language.
In this regard, I again argue that PLD-LISP's head/tail pattern matching lends
itself best to a LL1-parser with stack-based AST generation. Given more time, I
would have liked to have given a better attempt at this - in hindsight, I would
have done better to have used an automatic parser table generator rather than
attempting (and quickly failing) at doing this by hand.

\Sectend
