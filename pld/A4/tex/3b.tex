\newpage
\subsection{A4.3.b}

\begin{itemize}
  \item \emph{Implement the subset of Troll as presented in the assignment text.
    Discuss non-trivial choices.}
\end{itemize}

\subsubsection{Implementation overview}

I have (to the best of my knowledge) successfully implemented evaluation of
every expression in the Troll subset as embedded syntax in the PLD-LISP
interpreter. As explaiend ealier, I have not implemented any sort of parser, so
the former only holds under the assumption of full parenthesization.

\smallskip

Aside from handling each constructor given in the grammar, my implementation
handles:

\begin{itemize}
  \item specifying a number of isolated rolls, which are then returned as a
    collection of collections;
  \item correct variable binding, including dynamic scoping and variable
    overshadowing;
  \item flattening of arbitrarily nested collections of collections (with the
    exception of specifying multiple rolls, where a single collection of flat
    collections is returned).
\end{itemize}

I claim neither efficiency nor exhaustive correctness (I will, however, make an
effort of testing later), but my implementation \emph{does} handle the following
cases of run-time errors:

\begin{itemize}
  \item non-singleton collection used where singleton collection expected
  \item (singleton collection containing a) non-positive value used where (singleton
    collection containing a) positive value expected
  \item unknown variable names
\end{itemize}

The implementation consists of one large evaluator function, \ms{eval}, and a
number of helper functions. In the following subsubsection, I go into detail
with most of the features mentioned above, as well as other interesting features
of the code.

\smallskip

The code can be viewed in its entirety in the attached \ms{Troll.le} of the code
hand-in directory.

\subsubsection{Non-trivial implementation details}

\paragraph{Randomness}~\smallskip

My handling of randomness is entirely trivial, but I see it fit to mention that
for the sampling and accumulation operators (which both evaluate some of their
operand expressions multiple times, with different, random results each time),
randomness is achieved by passing expressions \emph{unevaluated}, such that they
can be re-evaluated with new random results.

\paragraph{Variable environment}~\smallskip

To implement dynamic scoping, I let my main evaluator function take an
additional parameter \ms{env} holding the local variable environment.
Unfortunately, this parameter must be passed explicitly, which has lead to
cluttered code and, during implementation, a great deal of errors.

\smallskip

Troll variable binding and lookup expressions are handled as such:


\begin{minted}[highlightlines={6, 10}]{lisp}
(define eval
  (lambda ...

    ; variable binding. v must be a symbol; e1 and e2 are arbitrary exps.
    ((v ':= e1 '; e2) env) (if (symbol? v)
                               (eval e2 (bind v (eval e1 env) env)))
    ...
    ; scalar expression. symbols are assumed to be variables, while
    ; singleton collections are created from scalar numeric values.
    (x env)  (if (symbol? x) (lookup x env) (if (number? x) (list x)))
  )
)
\end{minted}

In line 6, the current \ms{env} is extended with a binding of \ms{v} to
whichever value \ms{e1} evaluates to, before \ms{e2} is evaluated under this new
environment. Line 10 shows how symbols are looked up as variables. Note that
this is the bottom-most production of the \ms{eval} function.

\medskip

The actual variable environment \ms{env} is implemented as a list of two-element
PLD-LISP lists whose first element is a symbol and whose second element is a
fully evaluated collection. I omit the code for binding and lookup as it is
entirely trivial, except to say that variable overshadowing is implemented by
always prepending variables to the head of the environment list.

\paragraph{Singleton collections}~\smallskip

A number of Troll operators expect one or more singleton collections as
operands; for example, the arithmetic operators and the multiple dice roll
expects that \emph{both} operands are singleton, while the sample operator
expects its first operand to be.
Furthermore, we also have operators which expect singleton collections
containing positive values - examples of this are dice rolls and
\ms{least/largest}.

\smallskip

This is implemented using two wrappers for the main \ms{eval} function, called
\ms{evalST} and \ms{evalSTPos} (for ``evaluate singleton [positive]''), which
both evaluate its first argument expression and return empty lists (to signal
failure) when the result is not a singleton collection or a singleton collection
containing one positive value, respectively.

\paragraph{Flattening collections}~\smallskip

The Troll manual specifies that nested collections of collections are always
flattened. This is relevant for the sampling, union, and accumulate
expressions, which produce multiple, isolated collections before flattening into
a single collection.

Since collections are implemented as PLD-LISP lists, I write a generic function
for the flattening of arbitrarily nested and arbitrarily irregular PLD-LISP
lists (the function is called \ms{flatten} in the code hand-in), which can then
be used to wrap the result of a sampling, accumulation, or union expression.

\paragraph{Union expressions}~\smallskip

An example of collection flattening is in the union expression $\{\ms e_1,
\dots, \ms e_n\}$. 

As explianed earlier, I use simply PLD-LISP lists to represent Troll unions. To
evaluate arbitrarily long union expressions, I give \ms{eval} a head/tail
pattern as seen below:

\begin{minted}[highlightlines={6}]{lisp}
(define eval
  (lambda
    ...
    ; {e1, ..., en}: this production assumes n > 0; to create an empty collection,
    ; simply give an empty list. for example, '(count ()) evaluates to (0).
    ((e ', . es) env)  (flatten (cons (eval e env) (eval es env)))
    ...
\end{minted}

Notice how the same \ms{env} is used for each expression, and how the resulting
union is created using \ms{flatten}. The code comment explains the case of empty
unions.


% \paragraph{Collection (ie. list) operators}~\smallskip
%
% The implementation of \ms{sum}, \ms{min}, \ms{max}, \ms{count}, and the
% filtering
%
% To implement the collection operators \ms{sum}, \ms{min}, \ms{max}, and
% \ms{count}, I first implement generic list \ms{map} and \ms{reduce} functions.
% The first three operators are then implemented as reductions using addition, and
% binary min and max and their respective neutral elements. \ms{count} is
% simply implemented using the built-in \ms{length}.
%
% \medskip
%
% Similarly, list filtering is implemented using a generic \ms{filter} function,
% the code of which I shall also omit as it is trivial.


\paragraph{\texttt{least} and \texttt{largest} expressions}~\smallskip

\ms{least} and \ms{largest} are both implemented using a simple (ie. not
in-place) quicksort. To pick the \ms n least elements of some collection
\ms{xs}, the collection is first fully evaluated and sorted before the first
$\ms{min(n, count xs)}$ elements are extracted; for the \ms n largest, the list
is reversed after sorting.


\paragraph{\texttt{accumulate} expression}~\smallskip

Below snippet shows how \ms{accumulate} expressions are evaluated (interesting
lines highlighted):

\begin{minted}[highlightlines={4, 5, 12-14}]{lisp}
(define eval
  (lambda
    ...
    (('accumulate v ':= e1 'while e2) env) 
      (if (symbol? v) (flatten (accumulate v e1 e2 env)))
    ...
  ...
...

(define accumulate
  (lambda
    (v e1 e2 env) (if (define x (eval e1 env))
                      (cons x (if (eval e2 (bind v x env))
                              (accumulate v e1 e2 env))))
  )
)
\end{minted}

Lines 4-5 show how \ms{accumulate} expressions are matched in the main evaluator
function; if \ms{v} is a symbol, expressions \ms{e1} and \ms{e2} are passed
\emph{unevaluated} to a helper function for accumulation expressions.

\medskip

Given a symbol \ms{v}, two \emph{unevaluated} expressions, and the current
environment \ms{env}, the \ms{accumulate} helper function works as such:

\begin{itemize}
  \item evaluate \ms{e1} under \ms{env} and bind result to \ms{x} (PLD-LISP
    variable);
  \item evaluate \ms{e2} under the environment \ms{env[v -> x]};
  \item accumulate \ms x, and, if \ms{e2} evaluates to a non-empty collection;
    recurse with \ms{e1} and \ms{e2} again unevaluated, and using the original
    \ms{env} (ie. without the binding of \ms{v -> x}).
\end{itemize}

Note the \ms{if}-expression beginning on line 12 of the snippet. This is to
sequence the binding of \ms{eval e1 env} to the PLD-LISP variable \ms{x} before
recursing, such that \ms{e1} is only evaluated once. This works because
\ms{define} is always successful (which it is because \ms x is not a reserved
name in PLD-LISP).

\sectend

