
\newpage
\subsection{A4.1.c}
%
\begin{itemize}
  \item \emph{Is the problem of whether a SIS program terminates (with or
    without error) regardless of input a trivial, decidable, or undecidable
    property?}
\end{itemize}

\emph{Unfortunately, I have not managed to work out a foolproof answer to this
task - instead, I share some of my thoughts.}

\bigskip

Neither the SIS grammar nor the language specification restricts the
number of statements in a SIS program, but I shall restrict my view to those SIS
programs which can be expressed finitely. Under this assumption, a SIS program
is nonterminating only if it contains an infinite loop.

\medskip

Because the computational range of SIS is limited to whatever can be expressed
in three 32-bit variables, SIS is obviously \emph{not} Turing-complete, so a
cheap proof of undecidability by Rice's theorem is unfortunately out of the
question ;)

\smallskip

On the contrary, because there is no limit to the number of nested loops in a
SIS program, and thus no limit to the number of live loop variables at any given
time, the amount of memory required to run a SIS program is unbounded - for this
reason, it is not possible to simulate the program, checking for repeated memory
states on a realizable machine with bounded memory.

\paragraph{Termination for a single loop with no nested loops}~\smallskip

Termination is decidable for a program if it only contains singularly nested
loops. This is because if a loop does not contain nested loops, then it can only
contain increment and decrement statements, and because we assume SIS programs
to be finitely expressed, there can only be a finite number of such statements
in any loop body. 

\medskip

A finite sequence of statements can be counted in finite time, and hence the
change in value of any variable between the start and end of the loop can be
determined in finite time. As an example, for a loop with loop variable \ms{i},
lower bound $\ms a$ and upper bound \ms b, a loop terminates if there exists an
integer $k \geq 0$ such that:


\begin{align}
\ms a_0 + k \ms i_\Delta \;\equiv\; \ms b_0 + k  \ms
  b_\Delta \quad(\text{mod } 2^{32})\label{termination_single_loop}
\end{align}

where $\ms x_0$ is the value of variable \ms x \emph{before} the loop and $\ms
x_\Delta$ is the change in value of variable \ms x across a single loop
iteration. If such a $k$ exists, then $k$ is also the number of iterations of
the loop. Equation (\ref{termination_single_loop}) can be solved in finite time.
Note that the equation does not consider $\ms a_\Delta$; this is because under
the restrictions on loops, we know that the loop will terminate if $\ms i = \ms
a$ at the end of a loop iteration.

\medskip

Under the assumption that the program contains only singularly nested lists, we
can determine the values of $\ms a_0$ and $\ms b_0$ in similar fashion by
counting the number of decrements and increments leading up to the beginning of
the loop, and given $k$, we can compute the values of $\ms a_n$ and $\ms b_n$,
where $\ms x_n$ is the value of some global variable $\ms x$ after a
particular loop.

\bigskip

However, this method does not generalize to nested loops, since we cannot infer
$\ms x_0$ nor  $\ms x_n$ in a nest of loops; we might abstract these away in
variables in a system of equations whilst keeping track of net change in value
of any global variable at any point in the program, but these equations become
messy and, by my intuition, unsolvable once we consider the fact that a nested
loop can have a variable amount of iterations across iterations of its enclosing
loop.

\smallskip

By this reasoning - however lackluster - I posit that SIS is undecidable.

\Sectend
