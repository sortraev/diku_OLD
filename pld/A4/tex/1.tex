\newpage
\section{\underline{A4.1}}

\subsection{A4.1.a}


\begin{itemize}
  \item \emph{Write a \textbf{SIS} program that starts with $a = n$, $b = 0$,
    and $c = 0$, and which terminates with $a = n$, $b = 0$ and with $c$ equal
    to the sum of all numbers from 1 to $n$}.
\end{itemize}


I am looking for a program which computes:

\begin{align*}
  \sum_{i = 1}^n\ i
\end{align*}

This sum can be rewritten as a nested sum in which the body of the inner
summation is a constant 1, since the sum of $i$ ones is precisely $i$:

\begin{align*}
  \sum_{i = 1}^n\ i = \sum_{i = 1}^n\ \sum_{j = 1}^{i}\ 1
\end{align*}

This is helpful since we can accumulate constants using a finite sequence of
variable increments - in this case, accumulating the number $1$ corresponds to a
single SIS variable increment. However, before translating the math to SIS code,
we shift the bounds of the inner loop down by one - this is legal because the
summation maintains the same number of iterations, and we do not have to modify
the body because it is constant in the value of $i$ (this shift will be
convenient later):

\begin{align}
  \sum_{i = 1}^n\ \sum_{j = 1}^{i}\ 1 = \sum_{i = 1}^n\ \sum_{\blue{j =
  0}}^{\blue{i - 1}}\ 1 \label{sum}
\end{align}

We, of course, cannot shift the outer loop since \emph{its} body is \emph{not}
constant in $i$.

At this point I want to formulate the mathematical expression in the SIS
language. I use \ms{c} as accumulator for the result of the summation; then, for
each $1$ in the body of the summation corresponds I want to increment \ms{c} by
one. 

\newpage

The SIS code then becomes:

\begin{minted}[frame=none, linenos=false]{c}
for i = b to a {
  i++;
  for j = b to i {
    c++;
    j++;
  };
}
\end{minted}

As explained it was convenient to shift the inner summation bounds down by one -
this was done so we could use \ms{b = 0} for the lower bound. Notice then that
\ms{i} is incremented \emph{before} the inner loop of index \ms{j} rather than
after; this is to achieve the outer loop going from $1$ to and \emph{including}
$n$, rather than from $0$ to and including $(n - 1)$. As such we get $(k + 1)$
iterations of the inner loop on the $k$'th iteration of the outer loop.

\sectend

\subsection{A4.1.b}

\begin{itemize}
  \item \emph{Show that SIS is a reversible language.}
\end{itemize}

The course notes state that ``\emph{A programming language is reversible if
every operation it performs is reversible}'', and so I intend to show that SIS
is a reversible language by showing how each of its operations is reversible. I
do so by constructing reversal rules for each. First, here are my reversal
rules:
\begin{align*}
  \textit{R}\, (\ms{A++})    \quad &= \quad \ms{A--}\\[4pt]
  \textit{R}\, (\ms{A--})    \quad &= \quad \ms{A++}\\[4pt]
  \textit{R}\, (\ms{S1; S2}) \quad &=
      \quad \textit{R}\, (\ms{S2}) ;\ \textit{R}\, (\ms{S1})\\[4pt]
  \textit{R}\, (\ms{for A = B to C { S }}) \quad &=
      \quad \texttt{for A = C to B \{}\textit{R}\, (\ms{S})\texttt{\}}
\end{align*}

where $R(\ms{S})$ is the reversal of some statement \ms{S}, and where \ms{A},
\ms{B}, and \ms{C} are program variables.

\medskip

The first two rules state that an increment is inversed by a decrement and vice
versa.

\medskip

The third rule states that to reverse a sequence of two statements, we must
reverse each statement individually and then execute the resulting statements in
reverse order. Since reversing the order of statements is an associative
operation, this reversal rule applies to arbitrary sequences of statements.

\bigskip

The fourth reversal rule is the interesting one. It states that to reverse a
\ms{for} loop, we must swap the upper and lower bounds of the loop and apply
reversal to the loop body \ms{S}. \emph{Why?}

\smallskip

If a program \emph{p} terminates without error, then any loop in \emph{p} must
have executed a finite number of iterations \footnote{A program does not
terminate if it executes indefinitely!}. A loop whose body is \ms{S} and which
has terminated after $i$ iterations will have executed \ms{S} $i$ times, and is
thus equivalent to the unrolled sequence of statements:

$$
\ms{S}_1 \ ;\ \ms{S}_2\ ;\ \dots\ ;\ \ms{S}_i
$$

\noindent We can apply reversal to this sequence of statements using rule three - of
course here, all statements are identical, so we don't technically need to
reverse the order of the sequence, but we \emph{do} recursively reverse each
statement in the sequence. %

\medskip

\emph{Why swap the loop bounds?} Consider a loop in some program $p$ given by
\ms{for i = a to b { S }} which executes $n$ times (where $n$ is finite such
that the loop terminates). If we unroll the executed statements of the loop into
the sequence $\ms{S}_1\ ;\  \dots \ ;\ \ms{S}_n$, then \ms{i} will have had
value \ms{a} before executing $\ms{S}_1$ and value \ms{b} after executing
$\ms{S}_n$.

\smallskip

In $p'$, the loop is reversed to a sequence of statements $\textit{R}(\ms{S}_n)\ ;\  \dots \
;\ \textit{R}(\ms{S}_1)$. Here, the loop variable \ms{i} will have value \ms{b} before
executing \textit{R}(\ms{S}$_n$), and we expect it to have value \ms{a} after executing
\textit{R}(\ms{S}$_1$) - hence we swap the loop bounds.

\sectend

\newpage

\begin{itemize}
  \item \emph{Explain for the loop construct why (or why not) the restriction
    described in the assignment text is required for reversibility.}
\end{itemize}

Yes, the restriction \emph{is} required for reversibility. To show why, let us
\emph{lift the restriction on loops} - but first, let's introduce some
assumptions to ease understanding. Let:


\begin{itemize}
  \item $p'$ = \textit{R}($p$) = \texttt{for i = b to a
    \{}\textit{R}(\ms{S})\texttt{\}};
  \item $\hat{\texttt{x}}$ denote the value of some global variable
    \texttt{x} at the beginning of $p$ and before $p'$ terminates;
  \item $\bar {\texttt{x}}$ denote the value of global variable \texttt{x} before
    $p$ terminates and at the beginning of $p'$;
  \item \ms{S'} = \textit{R}(\ms{S}), for brevity;
  \item $\ms{S}_j$ and $\ms{S'}_j$ be the $j$'th execution of \ms{S} in $p$ and
    \ms{S'} in $p'$, respectively - note that $\ms{S}_j = \textit{R}(\ms{S'}_j)$;
  \item and $\ms{v}_j$ be the value of some program variable \ms{v} (global or
    loop variable) after executing $\ms{S}_j$ in $p$; similarly, let $\ms{v'}_j$
    be the value of \ms{v} after executing $\ms{S'}_{j}$ in $p'$.
\end{itemize}

Using this, I reformulate the property given in the assignment text:

\begin{enumerate}

  \item[(1)] if for any SIS program $p$ which starts with $\ms x =
    \hat{\texttt{x}}$ and ends with $\ms x = \bar{\texttt{x}}$ for all global
    variables \ms x, there exists a SIS program $p'$ which starts with $\ms x =
    \bar{\texttt{x}}$ and ends with $\ms x = \hat{\texttt{x}}$ for all global
    variables \ms x, then SIS is a reversible language.

\end{enumerate}

Let now $p$ be the program $\ms{for i = a to b { S }}$, and assume that $p$
terminates after $n$ iterations of the loop.
Since reversibility must hold for every subsequence of the program, I derive the
following invariant for $p$:

\begin{enumerate}
  \item[(2)] $\ms{v}_j = \ms{v'}_{j + 1}$ for all $j < n$.
\end{enumerate}

Where \ms v is any SIS variable (global or loop variable). Notice that the case
where $j = n$ is covered by property (1), since when $j = n$ the loop will have
exited and only global variables are live.

\bigskip

Assume now that at some point in $p$ we have $\ms i_k = \ms a_k$ after executing
a $k$'th iteration of the loop (with $1 \leq k < n$). Since we have lifted the
restriction on loops, an error is \emph{not} thrown; the loop continues on
executing for the remaining $n - k$ iterations, at which point it
terminates with $\ms{i}_n = \hat{\texttt{b}}$, after which $p$ terminates
\emph{without error} with $\hat{\texttt{x}}$ for all global variables $\ms{x}$.

\medskip

Now, we execute $p'$. Since $\ms i_k = \ms{i'}_{k + 1}$ and $\ms a_k =
\ms{a'}_{k + 1}$ (by the invariant), and because we assumed $\ms i_k =
a_k$, we will after iteration $(k + 1)$ of $p'$ have $\ms{i'}_{k + 1} =
\ms{a'}_{k + 1}$. At this point the loop terminates, meaning $p'$ \emph{without
error} with $\ms x = \ms{x'}_{k + 1}$ for all global variables \ms x. However,
we cannot assume $\ms {x'}_{k + 1} = \hat{\texttt{x}}$ is true for any \ms x or
$k$, and so invariant (2) is broken.

\sectend

