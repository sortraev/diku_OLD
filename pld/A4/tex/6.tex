
\newpage
\section{A3.6}

% Consider below Prolog program
%
% \begin{minted}{prolog}
% friend(a, b).
% friend(b, c).
% friend(c, d).
%
% knows(X, Y) :- knows(X, Z), knows(Z, Y).
% knows(X, Y) :- friend(X, Y).
% \end{minted}


\subsection{A3.6.a and A3.6.b}

\begin{itemize}
  \item \emph{Explain by means of the query resolution algorithm why the query
    \emph{\texttt{knows(a, d)}} fails to terminate.}
\end{itemize}

When \ms{knows(a, d)} is queried, the resolution algorithm will match the first
\ms{knows/2} rule: and try to unify \ms{X} with \ms a, and \ms Y with \ms d.
This is successful, since neither \ms{X} nor \ms{Y} is bound yet so we now have
the list of substitutions $\Theta = \{\ms{X01 = a},\ \ms{Y01 = d}\}$, where
\ms{X01} and \ms{Y01} are temporary variables.

\medskip

To solve this goal under $\Theta$, the algorithm will then attempt to solve
\ms{knows(a, Z)}, which again matches the first \ms{knows/2} rule - the
algorithm will unify \ms Y with \ms Z, which is successful, and the list of
substitutions is now $\Theta = \{\ms{X01 = a},\ \ms{Y01 = d},\ \ms{Y02 =
Z01}\}$. The algorithm recurses once again on solving \ms{knows(A, Z01)} (where
\ms{Z01} is a newly bound temporary variable), and in fact goes into an infinite
loop here.

\medskip

The infinite recursion happens because the resolution algorithm is always free
to instantiate a new temporary variable for \ms{Z} and attempt solving the first
\ms{knows/2} rule. The resolution tree looks like this:

\begin{minted}[linenos=false, frame=none]{text}
knows(a, d)
   |
   | X = a
   |
knows(a, Z01)
   |
   |
knows(a, Z02)
   |
   |
knows(a, Z03)
   |
  ...
\end{minted}

\newpage
\begin{itemize}
   \item \emph{Modify the program such that \ms{knows(a, d)} terminates
      successfully and explain why the modification works.}
\end{itemize}

To solve the problem, we need to eliminate left recursion while still encoding
the transitive closure of \ms{knows/2}. This is done by replacing the culprit
left recursing \ms{knows(X, Z)} call with a call to \ms{friend(X, Z)}, like in
the below snippet:

\begin{minted}[linenos=false, highlightlines={5}]{prolog}
friend(a, b).
friend(b, c).
friend(c, d).

knows(X, Y) :- friend(X, Z), knows(Z, Y).
knows(X, Y) :- friend(X, Y).
\end{minted}

Now, in solving the goal \ms{knows(X, Y)}, the resolution algorithm is forced to
satisfy the goal \ms{friend(X, Z)}, and since our program has no recursive rules
for \ms{friend/2} (in fact, it only has facts), we can guarantee progression
towards solving the supergoal.

\medskip

Now, the resolution tree for \ms{knows(a, d)} looks like this (the algorithm
makes a pre-order traversal):

\begin{minted}[frame=none, linenos=false]{text}
         knows(a, d)
            |
         friend(a, Z1)
            |
            | Z1 = b
            |
         knows(b, d)
            |
         friend(b, Z2)
            |
            | Z2 = c
            |
         knows(c, d)
        /           \
       /             \
friend(c, Z3)        friend(c, d)
    |                    |
    | Z3 = d       this is a known fact.
    |                  return true
knows(d, d)
    |
friend(d, Z4)
    |
no possible instantiation
  for Z4. return false
\end{minted}

\Sectend
