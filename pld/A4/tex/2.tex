\newpage
\section{\underline{A4.2}}

\subsection{A4.2.a}

\begin{itemize}
  \item \emph{Write the sequence of lines executed in the program given in the
    assignment text. Also, what is the value of \emph{\texttt{n}} printed?}
\end{itemize}


\emph{Note: it is ambiguous whether \emph{``CBreak''} should be printed whenever
the \emph{\texttt{if}}-condition is evaluated \emph{\textbf{or}} only when the
\emph{\texttt{break}} statement is actually reached.} For this reason, I answer
for both cases.

\smallskip

First, assuming ``CBreak'' is printed each time the  \ms{if}-condition if
evaluated:

\paragraph{Assuming ``CBreak''}
\begin{minted}[linenos=false]{text}
Init
Test
CBreak
IDecr
NIncr
Test
CBreak
IDecr
NIncr
Test
CBreak
Print
\end{minted}

Second, assuming ``CBreak'' is only printed when the \ms{break} statement is
reached:

\begin{minted}[linenos=false]{text}
Init
Test
IDecr
NIncr
Test
IDecr
NIncr
Test
CBreak
Print
\end{minted}

In any case, the value of \ms{n} printed is 2, since \ms{i} becomes 8 after two
iterations and, and on the third iteration the loop is broken since \texttt{i \%
j == 8 \% 4 == 0}.

\sectend

\subsection{A4.2.b}

\begin{itemize}
  \item \emph{Eliminate the \emph{\ms{break}} statement in the while-loop
    without using \emph{\ms{goto}} or \emph{\ms{continue}} statements.}
\end{itemize}

Since the conditional \ms{break} statement is nested directly inside the
\ms{while} loop, we can move the break condition up into the loop condition.
This is because looping until \ms{(i > j)} and then breaking the loop if
\texttt{(i \% j == 0)} is equivalent to simply looping until \texttt{(i > j \&\&
i \% j != 0)}.

\begin{minted}[highlightlines={2},linenos=false]{c}
int i = 10, j = 4, n = 0;
while (i > j && i % j != 0) {
  i--;
  n++;
}
printf("%d\n", n);
\end{minted}


\newpage
\subsection{A4.2.c}

\begin{itemize}
  \item \emph{Describe a general method for eliminating \emph{\texttt{break}}
    statements without introducing \emph{\texttt{goto}} statements.}
\end{itemize}

Consider the below C-like pseudocode containing a while loop with a \ms{break}
statement at some arbitrary position in the loop body:

\begin{minted}[linenos=false]{c}
// arbitrary code preceding the loop.
while(loop_cond()) { // loop_cond() is any function and may have side effects.

  ... // arbitrary code preceding the break.
    ...
      ...

      // the break statement may be arbitrarily nested.
      break;

      ...
    ... 
  ... // arbitrary code following the break.
}
\end{minted}

We wish to eliminate the \ms{break} statement but preserve semantics.

If we are disallowed from using gotos and break statements, then the only way to
manipulate control flow out of a loop is for the loop condition to somehow
become zero, causing the loop to end. This can be done like so:

\begin{itemize}

  \item declare a new variable \ms{foo} and initialize to a non-zero value
    before entering the loop;

  \item if the original loop condition is \ms{loop_cond()}, then replace this
    with \ms{foo && loop_cond()}. To preserve semantics, it is important that we
    place \ms{foo} as the first operand to the logical \ms{AND}, since
    \ms{loop_cond()} might have side effects;

  \item finally, replace all occurrences of ``\ms{break}'' in the loop body
    with:\\ ``\texttt{foo = 0; continue;}''

\end{itemize}

\newpage
\noindent The resulting code looks like so:

\begin{minted}[linenos=false]{c}
// arbitrary code preceding the loop.
int foo = 1;
while(foo &&          // important that we short-circuit on foo
      loop_cond()) {  // in case loop_cond() has side effects!
  ...
    ... // arbitrary code preceding the break.
      ...

      // set foo to zero and jump to loop head.
      foo = 0;
      continue;

      ...
    ... // arbitrary code following the break.
  ...
}
\end{minted}

In summary, whenever control reaches the point in the program where previously
there was a \ms{break} statement, the code will now set \ms{foo = 0}; jump to
the head of the loop; evaluate the logical \ms{AND} expression and, because
\ms{foo} is zero, short-circuit and exit the loop without evaluating
\ms{loop_cond()} that extra time. Thus semantics of the loop is preserved.

\Sectend
