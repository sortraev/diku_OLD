\newpage
\section{\underline{A4.5}}

\begin{itemize}
  \item \emph{Formulate big-step semantic rules for the new \emph{\ms{for}} loop
    construct.}
\end{itemize}


\subsection{Disambiguating the specifications}

The assignment text is ambiguous in a number of ways. First off, it states that
``\emph{[...]the loop body s is executed, if $n_2 > n_1$}''. However, I argue
that it is more correct to say that the loop body is executed if $\ms x < \ms
n_2$. \footnote{Hans agrees on this, as per his reply to this post on the
course's Absalon forum:
\url{https://absalon.ku.dk/courses/47084/discussion_topics/351077}}

Second, it is not entirely clear whether \ms n$_1$ and \ms n$_2$ are constants
or variables (ie. modifiable in \ms s), or whether \ms x is modifiable in \ms s
(not counting the increment immediately following \ms s).

\medskip

\noindent I resolve the ambiguities by making the following assumptions:

\begin{enumerate}

  \item the loop condition is $\ms x < \ms n_2$, not $\ms n_1 < \ms n_2$;

  \item \ms n$_1$ and \ms n$_2$ are constants and are thus never modified in
    \ms s;

  \item \ms x is a variable as any other, and can thus be modified in \ms s (in
    addition to the increment immediately following \ms s).

\end{enumerate}

I also make the following assumption on the loop construct, since nothing was
stated explicitly in the text:

\begin{itemize}
  \item[4.] the assignment \ms{x := n}$_1$ is made \emph{exactly} once - right
    before the first evaluation of the loop condition, and the state $\sigma$ is modified
    to reflect this assignment even if \ms s is executed zero times;
\end{itemize}


\subsection{Desugaring the \texttt{for} loop syntax}



From the lectures on loops and control flow, we know that \ms{for} loops are
(typically) syntactic sugar for equivalent \ms{while} loops. Using assumptions 4 and 1 as given above, a \ms{for} loop construct can be
desugared into a semantically equivalent \ms{while} loop as such:

\begin{align*}
  \ms{for x := n}_1 \ms{ to n}_2 \ms{ do s}
  \quad\equiv\quad
  \ms{x := n}_1\ms{; while (x < n}_2\ms{) {s; x++;}},
\end{align*}

where $\equiv$ is semantic equivalence of statements. \emph{But why desugar in
the first place?} Answer: In devising the new semantic rule(s), I suspect that I
will need to be able to distinguish between the first iteration of a \ms{for}
loop, which involves initialization of \ms x, and subsequent iterations, which
do not. I suspect this is better expressed by desugaring.


\subsection{New semantic rules}

I formulate the following three semantic rules for the new \ms{for} loop
construct:


\begin{align}
  \frac{ \begin{gathered}
    \sigma[\ms x \mapsto \ms n_1] \;\vdash\; \ms{while (x < n}_2\ms{) do { s; x++; }}
    \;\rightsquigarrow\; \sigma'
    \end{gathered}
    }
    {
      \sigma\;\vdash\;\ms{for x := to n}_1\ms{ to n}_2 \ms{ do s}\;
      \rightsquigarrow\; \sigma'
    }\label{desugaring_rule}\\[24pt]
  \frac{
      \sigma(\ms x) < \ms n \qquad
      \sigma  \;\vdash\; \ms s    \rightsquigarrow \sigma' \qquad
      \sigma' \;\vdash\; \ms{x++} \rightsquigarrow \sigma'[\ms x \mapsto
      \sigma(\ms x) + 1] = \sigma''
    }
    {
      \begin{aligned}
      \sigma\;\vdash\; \ms{while (x < n) do { s; x++; }}
        \;\rightsquigarrow\; \sigma''
      \end{aligned}
    }\label{while_true}\\[6pt]
  \frac{
    \begin{gathered}
      \sigma(\ms x) \geq \ms n\\
    \end{gathered}
    }
    {
      \sigma\;\vdash\; \ms{while (x < n) do { s; x++; }} \;\rightsquigarrow\; \sigma
    }\label{while_false}
\end{align}

% Consider below semantic rule (\ref{desugaring_rule}), which describes how a \ms{for}
% loop is desugared into an equivalent \ms{while}-loop:

Rule (\ref{desugaring_rule}) is the desugaring rule, states that the change in
the store of a \ms{for} loop is equivalent to that of the change in the store of
an equivalent \ms{while} loop under $\sigma$ extended with the initialized loop
variable. \emph{One could then define semantic rules for generic \ms{while}
loops and use these to derive \ms{for} loop semantics - but for the sake of
cementing the new \ms{for} loop, I define rules specifically for the desugared
loop.}

\bigskip

Rule (\ref{while_true}) describes the case where the loop condition is true, and
states that:

\begin{itemize}
  \item if \ms{x < n} under $\sigma$, and
  \item if under $\sigma$, executing the \ms{for} loop body \ms s provokes a
    change in state to some modified state $\sigma'$, and
  \item if under $\sigma'$ following an execution of \ms s, the final increment
    of \ms x further modifies the state from $\sigma'$ to some final state
    $\sigma'' = \sigma'[\ms x \mapsto \sigma'(\ms x) + 1]$,
  \item then the semantics of the \ms{while} loop is a change in state from
    $\sigma$ to $\sigma''$.
\end{itemize}

\newpage

Finally, rule (\ref{while_false}) describes how, if the loop condition is false,
then the desugared \ms{while}-loop leaves the state unmodified (ie. \ms s is
never executed)

\bigskip

Notice that:

\begin{itemize}
  \item while not immediately obvious from rule (\ref{desugaring_rule}), the
    store $\sigma$ is always modified by executing a \ms{for} loop, since even
    if the loop body \ms s is never executed (ie. if $\ms n_1 \geq \ms n_2$),
    then the semantics is a chagne in state from $\sigma$ to $\sigma[\ms x
    \mapsto \ms n_1]$.
  \item rule (\ref{while_true}) explicitly separates the loop body \ms s from
    the following increment of \ms x as to make the distinction that \ms x can
    also be modified in \ms s.
\end{itemize}

\Sectend
