
\newpage
\section{A3.3}

\begin{itemize}
  \item \emph{Extend the list program with a predicate \ms{longerthan/2} such
    that\\ \ms{longerthan(L1, L2)} holds if the list \ms{L1} is longer than the
    list \ms{L2}.}
\end{itemize}

The assignment text does not state what the predicate should return if either
\ms{L1} or \ms{L2} is not a well-formed list. For example, what should be the
value of the below statement:

$$
\ms{longerthan(cons(3, cons(4, foo)), cons(2, nil)).}
$$

Clearly the first argument is not a well-formed list, but a naive implementation
might simply determine that the first list has more layers of ``cons'' and thus
conclude that it must be the longer list.

\medskip

However, I will assume that the predicate should be \ms{false} when either one
or both of the lists are not well-formed lists. Thus my solution is:

\begin{minted}[linenos=false]{prolog}
% if both L1 and L2 have a tail, then L1 is longer than T2 if T1's tail is
% longer than T2's tail.
longerthan(cons(_, T1), cons(_, T2)) :- longerthan(T1, T2).

% if L1 has a tail and L2 is nil, then L1 is longer than L2. however, we must
% also assert that L1 is, in fact, a well-formed list.
longerthan(cons(_, T1), nil) :- list(T1).
\end{minted}

Where \ms{list/1} is as defined in the lecture slides.

\Sectend
