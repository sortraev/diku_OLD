
\newpage
\section{A3.5}

\begin{itemize}
  \item \emph{Extend the type rules of the functional language (as presented in
    the lectures) with rules for the new \ms{private .. within} expression.}
\end{itemize}


I extend the type rules with the below rule:

\begin{gather*}
  \frac{
    \begin{aligned}
      \tau          &\vdash e_1 \; :\ t_1\\
      \tau[x \mapsto t_1] &\vdash e_2 \; :\ t_2
    \end{aligned}
  }{\tau \vdash \ms{private }\ x = e_1\ \ms{within } \ e_2\ :\ t_2}
\end{gather*}


The rule states that:
\begin{itemize}
  \item if $e_1$ has the type $t_1$ in the type environment $\tau$, and
  \item if, after extending $\tau$ with the type binding $x : t_1$, the
    expression $e_2$ has type $t_2$,
  \item then $\ms{private }\ x = e_1\ \ms{within } \ e_2$ has type $t_2$
\end{itemize}

Interestingly, the \ms{private .. within} expression is simply a renaming of the
more well-known \ms{let .. in} expression from the ML family of languages.

\Sectend
