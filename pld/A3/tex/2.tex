\newpage
\section{A3.2}

Consider the below function written in SML

\begin{minted}[linenos=false]{sml}
  fun dingo (x, nil)   = false
    | dingo (x, y::ys) = x = y orelse dingo (x, ys);

\end{minted}

\subsection{A3.2.a}

\begin{itemize}
  \item \emph{Explain what the function does.}
\end{itemize}

Given a list \ms{ys} and an element \ms{x}, a call to \ms{dingo (x, ys)} returns
\ms{true} if \ms{x} exists in \ms{ys}; else \ms{false}.

\medskip

This is because: The first function clause states that if \ms{ys} is \ms{nil}
(the empty list), then surely \ms{x} is not a member of it, whereas the second
function clause returns \ms{true} if \ms{x} is equal to the head of the list
\ms{ys}; else a recursive call to examine the tail of the list is made.

\sectend


\subsection{A3.2.b}

\begin{itemize}
  \item \emph{Use the Hindley-Milner algorithm to infer the type of
    \emph{\ms{dingo}}.}
\end{itemize}

%
% I start by rewriting the function definition to have just one clause with a
% \ms{case} expression matching on the second parameter:
%
% \begin{minted}{sml}
% fun dingo (x, ys) =
%   case ys of
%     nil    => false
%   | y::ys' => x = y orelse dingo (x, ys')
% \end{minted}
%
% and argue that this definition of the function is equivalent to the original.

\medskip

\noindent I first initialize a type environment $\rho_0$ (corresponding to the
first function clause) with type variables for the first function clause of
\ms{dingo} with parameters \ms{x} and \ms{nil}, as well as an empty set of
global type bindings $\sigma$:

\begin{align*}
  \rho_0 &= [\ms{dingo} \mapsto ((A * B) \to C), x \mapsto A, nil \mapsto
  B]\\[4pt]
  \sigma &= [\ ]
\end{align*}

Now, I wish to infer the type of the function body in the new type environment
$\rho_0$.

\medskip

I first wish to unify $B$ with the actual type of \ms{nil}. Since \ms{nil} is a
null list literal, I know its type to be $\text{list } D$ for some new type
variable $D$, and I must then compute $\text{unify}(B,\ \text{list } D)$.

\smallskip

$B$ is an unbound type variable, so the unification is successful if $\text{list
} D$ does not contain $B$. The occurs check passes since $D$ does \emph{not}
contain $B$; thus unification is successful, and I add the binding $B
\mapsto \text{list } D$ to $\sigma$:

$$
\sigma := \sigma[B \mapsto \text{list } D]
$$

I also infer this function clause to have type \ms{bool}, since the return value
\ms{false} is a boolean literal. Since in $\rho_0$, the return type of \ms{dingo}
is the type variable $C$, I now need to compute $\text{unify}(C,\ \ms{bool})$.

\smallskip

Since $C$ is an unbound type variable, the unification is successful if $C$ does
not contain $\ms{bool}$, but this is trivially true since $\ms{bool}$ is a type
constructor with zero parameters. Thus I add the binding $(C \mapsto \ms{bool})$
to $\sigma$:

\begin{align*}
  \sigma := \sigma[C \mapsto \ms{bool}]
\end{align*}


Now, I want to infer the type of the second function clause given the bindings
acquired in the first. I continue with the type environment $\rho_0$, and at this
point, the set of global bindings $\sigma$ is:

$$
\sigma = [B \mapsto \text{list } D,\ C \mapsto \ms{bool}]
$$

Consider now the second function clause. Since the second parameter is matched
against a head/tail list pattern, I should add \ms{y} to $\rho_0$, creating a
new type environment $\rho_1$ given by:

\begin{align*}
  \rho_1 := \rho_0[\ms{y} \mapsto D]\\[4pt]
\end{align*}

Since \ms{ys} maps to $\text{list } D$.

\medskip

I now want to infer the type of function clause body. Since it consists simply
of a logical OR expression, I know that the type of the body is \ms{bool}.
Unifying $C$ (the return type of \ms{dingo}) with \ms{bool} is successful since
there exists a mapping from $C$ to \ms{bool} in $\sigma$.

\medskip

However, to successfully infer the type of the function body, the logical OR
expression must also be well-formed - it is well-formed if both of its operands
are of \ms{bool} type.

\medskip

I start with the left-hand operand \ms{x = y}. This equality is well-typed if
its operands are of the same type. To find this out, I compute the unification
of the types of \ms{x} and \ms{y} which I have given in the local type
environment:

\begin{align*}
  \text{unify}(\rho(\ms{x}),\ \rho(\ms{y})) &= \text{unify}(A,\ D)\\[4pt]
\end{align*}

The unification is determined using unification rule 4 (as given in the lecture
notes) - this holds because $A$ is an unbound type variable and $D$ does not
contain $A$. $D$ is shown to not contain $A$ by the fact that $D$ is itself an
unbound type variable different from $A$.

\medskip

As such $\text{unify}(A,\ D)$ holds - a new type variable $E$ is added to the
set of global bindings $\sigma$:

$$
\sigma := \sigma[D \mapsto A]
$$

This also updates the mapping of $B$ to $\text{list } A$.

\bigskip

Since unification of \ms{x} and \ms{y} was successful, we only need the second
operand of the logical OR expression to be well-typed. This is the case if it is
also a \ms{bool} type. Since it is a recursive call to \ms{dingo}, which we know
from the local type environment to have return type \ms{bool}, this also holds.
As such unification of the two function clauses is successful.

\bigskip

The entire function has been type inferred, and the resulting set of type
bindings is:

$$
\sigma = [B \mapsto \text{list } A, C \mapsto \ms{bool}, D \mapsto A]
$$

The type of the function is then:

$$
\ms{dingo}\ :\ (A * \text{list } A) \to \ms{bool}
$$


By replacing unbound type variables (there is only $A$) with type parameters,
the type is generalized and its type schema is then:


$$
\ms{dingo}\ :\ \forall \alpha\ .\ (\alpha * \text{list } \alpha) \to \ms{bool}
$$



\Sectend
