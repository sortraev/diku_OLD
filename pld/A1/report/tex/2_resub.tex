\newpage
\section{A1.2) Rosetta 2 \textcolor{red}{- resubmission}}

\subsection{A.1.2.b) \textcolor{red}{- resubmission}}

\begin{itemize}
    \item \emph{Discuss advantages and disadvantages of this approach over
      recompiling source code to run on ARM.}
\end{itemize}

\textcolor{red}{In my feedback, I was asked about the overall running time as
compared to simply executing ARM. The newly added answers are in red.}


\paragraph{Advantages}~\smallskip

The main advantage is of course portability. With Rosetta 2, users can run
programs written for x86 on their ARM machines. This means that users can
purchase a new Mac machine without having to let go of their old software
written for x86.

\medskip

This is of course also a huge benefit for the developers, since they do not have
to target two different architectures when they want their software to run on
both new and older machines, but instead simply target x86 and then let users
with newer machines run through Rosetta.

\paragraph{Disadvantages}~\smallskip

the JIT compiler is prone to making bad decisions. It might AOT-compile code
which is not run sufficiently many times to amortize the cost of compilation, or
it might neglect to AOT-compile parts of the code which \emph{is} in fact run
often, potentially producing a bottleneck where there shouldn't be one.

\medskip

In addition, there is some memory overhead in the JIT-compiler, since to
function it essentially needs to keep the interpreter running, whilst
occasionally invoking the compiler. This overhead can be significant if the input
program is small.

\paragraph{\textcolor{red}{Performance implications}}~\smallskip

\textcolor{red}{Running a program through Rosetta obviously has an effect on the
overall running time of that program. In the \emph{very best} case, using Rosetta
can improve running time - but this only happens when the most demanding parts
of the code are AOT-compiled early and when the less used parts of the code is
never AOT-compiled.}

\smallskip

\textcolor{red}{In the worst case, Rosetta will first execute the program
through the interpreter a number of times before eventually having compiled the
entire program - in this case, the running time will be roughly as large as if
the program had originally been compiled to ARM and run, \textbf{plus} whatever
time was spent interpreting the program in the first place.}

\smallskip

\textcolor{red}{Compared to simply running ARM and ignoring the time it takes to
AOT-compile to ARM, the running time is significantly larger.}

\Sectend
