\section{A1.1) PLD Lisp}

\subsection{A.1.1.a) \texttt{merge} and \texttt{add}}
\label{merge_add}

\begin{itemize}
    \item \emph{Implement \texttt{merge} and \texttt{add} as specified in the
        assignment text. Present the implementation in the report.}
\end{itemize}

Below listing contains my entire implementation of \ms{assignment1.le}, except
for testing expressions (as will be discussed later).

\begin{minted}[frame=none]{text}
(define merge
  (lambda
    (xs ()) xs
    (() ys) ys
    ((x . xs) (y . ys)) (cons x (cons y (merge xs ys)))
  )
)

(define add
  (lambda
    ((x . xs)) (+ (add x) (add xs))
    (x)        (if (number? x) x 0)
  )
)
\end{minted}

\sectend

\newpage
\subsection{A.1.1.b) sample usage}

As mentioned, my \ms{assignment1.le} contains a number of test S-expressions.
Instead of copy+pasting a shell session, I instead include my test cases in the
below snippet.

\begin{minted}{text}
(define assertEqual
  (lambda (a a) 'testSuccess
          (a b) 'testFailure
  )
)

'(merge - both operands empty)
(assertEqual (merge '() '())
             '())

'(merge - first operand empty)
(assertEqual (merge '() '(4 5 6 7))
             '(4 5 6 7))

'(merge - second operand empty)
(assertEqual (merge '(0 1 4 2) '())
             '(0 1 4 2))

'(merge - both operands non-empty)
(assertEqual (merge '(1 2 3) '(4 5 6 7))
             '(1 4 2 5 3 6 7))

\; add tests
'(add - empty sum)
(assertEqual (add '())
             0)

'(add - 1D list of numbers)
(assertEqual (add '(1 -42 4 +100 7 57 0))
             127)

'(add - list of only non-numbers)
(assertEqual (add '(cons foo '(() 1be2)))
             0)

'(add - multiple layers of nesting)
(assertEqual (add '(4 () '(5 '(-6 0 '(1 NULL -1)) 7 cons) 3 foo))
             13)
\end{minted}

All of my test cases are successful. To reproduce, run \ms{make} from within my
code hand-in.

\sectend


\newpage
\subsection{A.1.1.c) implementation details and reflection}

\begin{itemize}
  \item \emph{Reflect on PLD-LISP. What was easy and what was hard? Did it take
    long to learn the syntax, and was it difficult to express yourself in the
    language?}
\end{itemize}

It did not take long for me to learn the syntax. The main reason for this is
that I have prior experience in other languages with similar syntax and
programming models, such as Haskell and Prolog. \smallskip

Second, PLD-LISP came with very good documentation (in the provided
\ms{PLD-LISP.pdf}), which answered \emph{almost} all of my questions.

\bigskip

The grammar of PLD-LISP, I would argue, is very clean - but this can probably be
attributed to the fact that the language is very restrictive. As far as I was
able to discern, PLD-LISP does for example not allow local variable bindings,
and control flow is limited to pattern matching.

\bigskip

However, I did find the pattern matching of PLD-LISP to be quite useful in
conjunction with the dynamic type system.

\smallskip

To illustrate why, let's look at implementing \ms{add} in a different language
which has a similar programming model, but which is statically typed: Haskell.
Since Haskell is statically typed we cannot express arbitrarily nested lists,
and so the below code does not compile:

\begin{minted}[frame=none]{haskell}
add (x:xs) = add x + add xs
add []     = 0
add x      = x
\end{minted}

The compiler complains that it cannot construct the infinite type for \ms{xs}.
To solve this problem, we have to introduce a recursive data type:

\begin{minted}[frame=none]{haskell}
data Element = Number Integer | Symbol String | List [Element]

add :: Element -> Integer
add (List (x:xs)) = add x + add (List xs)
add (Number x)    = x
add _             = 0
\end{minted}

As we saw in task A.1.1.a), this is more concisely expressed in PLD-LISP.

\sectend

\newpage
\subsubsection{Problems in the language}

As stated, implementation was mainly a breeze. However, I did have one
significant problem with the language. It occured during implementation of
\ms{merge}, and it had to do with list construction using \ms{cons}.

\smallskip

In PLD-LISP, a list can be head/tail pattern matched with the syntax \ms{(head .
tail)}, similar to Haskell's \ms{(head : tail)}. However, whereas in Haskell
this exact same syntax can also be used outside of pattern matching contexts to
prepend elements to lists, in PLD-LISP the syntax \ms{(x . xs)} cannot be used
to used to construct the list whose head is \ms{x} and whose tail is \ms{xs}.

\medskip

Instead, the \ms{cons} function is used to prepend elements to lists, and as a
consequence, the recursive case of the \ms{merge} function in PLD-LISP is
somewhat awkward:

$$
\ms{lambda (((x . xs) (y . ys)) (cons x (cons y (merge xs ys))))}
$$

\noindent as opposed to Haskell:

$$
\ms{merge (x:xs) (y:ys) = x : y : merge xs ys}
$$

I estimate that over half of the imlementation time for the entire coding task
was spent trying to figure out why \ms{(x . (y . (merge xs ys)))} produced
invalid expression errors - the interpreter was not very helpful with error
messages either, stating simply for some general input \ms{(x . xs)} that \ms{x
can not be applied as a function}.

\Sectend
