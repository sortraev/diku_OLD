\newpage
\section{A1.3) Postfix expressions using queues}

\subsection{A.1.3.a)}

\begin{itemize}
    \item \emph{Describe pseudocode for translating fully parenthesized infix
      expressions to queue postfix expressions.}
\end{itemize}

I take the hint given in the assignment, of processing infix expressions one at
a time, outputting numbers and operators, while re-queueing subexpressions.

\begin{minted}{python}
infix_to_queue_expression(exp):

  out  = new empty stack # will hold the output queue expression.

  exps = new empty queue # will hold unprocessed (sub-)expressions,
  exps.enqueue(exp)      # starting with the entire infix expression.

  while exps is non-empty:
    case exps.dequeue() of
      (number x) ->      # if next in queue is a number, simply push to out.
        out.push(x)

      # if next in queue is a binop, push its operator
      # to out and enqueue its two subexpression operands.
      (subexp_1, operator, subexp_2) ->
        out.push(operator)

        exps.enqueue(subexp_2)
        exps.enqueue(subexp_1)

  return out
\end{minted}


\newpage

\subsection{A.1.3.b)}
\begin{itemize}
  \item \emph{Show the queue and (partial) output during translation of \ms{((1
    + 3) - (5 * 7))} to \ms{1 3 5 7 + * -}.}
\end{itemize}
\begin{minted}[linenos=false]{text}
step | exps queue       | dequeuing     | output  | comment
0    | [((1+3)-(5*7))]  |               |         | init exps queue

1    | []               | ((1+3)-(5*7)) |         | dequeue expression
2    | [(1+3), (5*7)]   |               | -       | output op; queue subexps
3    | [(1+3)]          | (5*7)         | -       | dequeue expression
4    | [5, 7, (1+3)]    |               | *-      | output op; queue subexps
5    | [5, 7]           | (1+3)         | *-      | dequeue expression
6    | [1, 3, 5, 7]     |               | +*-     | output op; queue subexps

7    | [1, 3, 5]        | 7             | +*-     | dequeue expression
8    | [1, 3, 5]        |               | 7+*-    | output value
9    | [1, 3]           | 5             | 7+*-    | dequeue expression
10   | [1, 3]           |               | 57+*-   | output value
11   | [1]              | 3             | 57+*-   | dequeue expression
12   | [1]              |               | 357+*-  | output value
13   | []               | 1             | 357+*-  | dequeue expression
14   | []               |               | 1357+*- | output value

15   | []               |               | 1357+*- | queue empty; terminate
\end{minted}

