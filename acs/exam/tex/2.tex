\newpage
\section{Performance}

\subsection{Bottlenecks}

\begin{itemize}
  \item \textit{What are the performance bottlenecks in the two scenarios?}
\end{itemize}

One bottleneck that the first scenario exhibits is the case where the
aggregation stage of the computation is as large as or larger than the
computation stage. Consider for example squaring a list of numbers and then
summing the results; the squaring would be assigned to the workers, while the
summation would necessarily be done by the aggregator. Here, the longest path of
execution is $O(n)$ long, since that is how long it takes the aggregator to sum
$n$ numbers, and then the benefits of parallelization are possibly lost in the
overhead of communication.
\medskip

The given example only holds for scenario 1, but the same problem is exhibited
by scenario 2. A solution to this problem is have multiple aggregation steps,
which can potentially reduce the time of aggregation from being \emph{linear} in
the size of input to being \emph{logarithmic} in the size of the input. This is
known as a reduction tree, and is largely what the architecture behind MapReduce
is based on, so using MapReduce would also be a good solution.
\bigskip

Another bottleneck that impedes both systems is data skew. Consider the
histogram example given in the text - if each worker is eg. given one file of
text each, then one file may be 3 words long while another may be a million
words long. The performance of such a system is bounded by the largest file,
which may overshadow even the aggregation step.
\smallskip

This type of problem is actually the fault of the scheduler, and the solution to
this is to have more clever data partitioning, such that all workers are given
equal workload.
\sectend

\subsection{Performance Evaluation}

\begin{itemize}
  \item  \textit{How would you evaluate the performance of the above
    architecture in both scenarios?}
\end{itemize}

First off I would want to compare the execution time with that of a single
worker thread, in order to examine the overhead in parallelization. In this
regard it would also be beneficial to evaluate the aggregation step in isolation
to figure out where the bottleneck lies in the system.
\medskip

Wrt. performance metrics, I would first and foremost consider throughput. For
most operations this is expected to increase as the number of workers increases,
while for others the overhead of communication as well as the aggregation step
might be dominant.
\medskip

Then, I would measure the amount of extra resources required for the
parallelization and weigh it against the improve in throughput to determine
whether parallelization is worth the cost.

\begin{itemize}
  \item \textit{Would you setup the experiments differently for the two
    scenarios?}
\end{itemize}

No. I would perform the same experiments on both scenarios to better be able to
compare the results, and in order to better determine which scenario the
architecture handles better.

\Sectend
