\newpage
\section{Atomicity}

\begin{itemize}
  \item  \textit{Give three flavors of atomicity.}
\end{itemize}


We have discussed many flavors, but three of them that I can recall are
\emph{all or nothing atomicity}, \emph{before-or-after atomicity}, and
\emph{durability atomicity}.
\medskip


\textbf{All-or-nothing atomicity} aims to ensure that from the point of view of
the database, any given transaction is always either finished or not at all
initiated, which is important since half-baked results can disturb other
transactions. This can be hard to achieve since if transactions depend on each
other (eg. if they access the same memory), then ensuring the property holds for
one transaction is dependent on ensuring the property for other transaction,
especially in the case of failure. One way of ensuring this flavor of atomicity
is logging changes before the transactions commits such that they can be undone.
\medskip

\textbf{Before-or-after atomicity} is the idea of scheduling and executing
transactions such that from the point of view of the invoker, the actions occur
either completely before or completely after one another. This can be hard to
achieve since it involves carefully analyzing the access patterns of
transactions and devising a serializable schedule - this can be done with eg.
two-phase locking. However, sometimes it is simply impossible to interleave
transactions and still achieve a serializable schedule - in this case the only
way to guarantee before-or-after atomicity is to literally run them
sequentially.
\medskip

\textbf{Durability atomicity} is discussed with respect to crash recovery. It is
the notion of storing objects in permanent stable storage such that they can
always be access at any later point. This is hard to achieve since in practice
there is no such thing as permanent storage - however, with eg. redundancy
(RAID), stable storage can be \emph{approximated} (meaning it is practically
indefinite).

\Sectend
