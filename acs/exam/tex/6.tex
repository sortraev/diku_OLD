\newpage
\section{Distributed Transaction}

\subsection{Basic Concepts of Distributed Transactions}


\begin{itemize}
  \item \textbf{Basic Concepts of Distributed Transactions, 6.1.1:}
        \textit{Does the DanskeShop setup fulfill the atomicity property in case
        of node failures? Use an example to explain why or why not.}
\end{itemize}

It does \emph{not}, since it only uses a one-phase commit. Consider this
example: P1 sends a NewOrder message to both Payment and Shipment. Payment
agrees and sends back an AckOrder message to P1. Meanwhile, there is a node
failure at Shipment and so the NewOrder message is not even received at
Shipment.
\medskip

Payment does not wait for a doCommit message from Order since that is not part
of the protocol, but rather simply begins processing the request as if all is
good. However, the order is never shipped, so the customer will be paying for
nothing.

\sectend

\subsection{Two-Phase Commit}

\begin{itemize}
  \item \textbf{Two-Phase Commit, 6.2.1:}
        \textit{Describe the 2PC commit procedure of a New Order transaction
        when there is no failure. Write down the coordinator's log records.}
\end{itemize}

\begin{enumerate}
  \item Coordinator writes "Prepare order" to log.
  \item Coordinator sends \ms{canCommit?}-messages to all three modules.
  \item Coordinator receives \ms{yes} from all three modues.
  \item Coordinator writes "Commit order" to log.
  \item Coordinator sends \ms{doCommit}-messages to all three modules.
  \item Coordinator receives \ms{haveCommited} messages from all three modules.
  \item Coordinator writes "End order" to log.
\end{enumerate}


\begin{itemize}
  \item \textbf{Two-Phase Commit, 6.2.2:}
        \textit{Describe a scenario in which Shipment and Payment can proceed to
        complete the transaction.}
\end{itemize}

The exam text does not state what is meant by ``to complete the transaction''.
\textit{I will assume that a transaction is completed if it is either commited
or aborted.}
\medskip

Also, the text says that Order fails after the coordinator decides to abort and
after the abort has been logged, but \emph{before} abort messages are sent out.
\medskip

Then Shipment and Payment can proceed to complete the transaction as such:

\begin{enumerate}
  \item having already logged the abort, the coordinator sends rollback messages
    to all three modules. Payment and Shipment receive these.
  \item Payment and Shipment each rollback the commits and send back
    \ms{haveRolledBack}-messages to the coordinator, thus completing the
    transaction with an abort (as per the assumption that a transaction is
  completed when it is either committed or aborted).
\end{enumerate}

When Order comes back online, the coordinator can retry the transaction if it
wishes.


\begin{itemize}
  \item \textbf{Two-Phase Commit, 6.2.3:}
        \textit{Under the given scenario, describe what could happent to
        Shipment if it has not received the abort message, and what could happen
      to Order after it restarts from the failure.}
\end{itemize}

When Order restarts from failure, it will look for a commit message in its log.
However, it will not find one and instead assume that it must have aborted.

\Sectend
