\newpage
\section{Recovery}

\subsection{The Missing Log}

\begin{itemize}
  \item \textit{Complete the log given in the exam text.}
\end{itemize}

Since my exam number is 307, the second page table entry is \ms{P307} and its
\ms{recLSN} is 6. Below is the log:

\begin{minted}[linenos=false]{text}
LSN | PREV_LSN | XACT_ID | TYPE       | PAGE_ID | UNDONEXTLSN
----+----------+---------+------------+---------+------------
1   | -        | -       | begin CKPT | -       | -
2   | -        | -       | end CKPT   | -       | -
3   | NULL     | T3      | update     | P42     | -
4   | 3        | T3      | commit     | -       | 3
5   | NULL     | T1      | update     | P99     | -
6   | NULL     | T2      | update     | P307    | -
7   | 6        | T2      | abort      | -       | 6
8   | 5        | T1      | update     | P99     | 5
+++ CRASH +++ CRASH +++ CRASH +++ CRASH +++ CRASH +++ CRASH +++
\end{minted}

\subsection{The Recovery}

\begin{itemize}
  \item \textit{Based on the log, complete the recovery procedure. Show the 5
    items mentioned in the exam text.}
\end{itemize}

\paragraph{Set of winners and losers}~\smallskip

The set of winners is $\{T3\}$, since T3 was the only transaction to commit
before the crash. The set of losers is thus $\{T2, T1\}$.

\paragraph{LSNs for start of redo and end of undo}~\smallskip

The redo phase starts at LSN 3, since this is the smallest recLSN in the DPT.
\smallskip

Since initially, $\text{toUndo} = \{7, 8\}$, we will undo the transactions in the
order 8, 7, 6, 5. The undo phase thus ends at LSN = 5.

\paragraph{Set of log records that may cause pages to be rewritten during redo}~\smallskip

The set of log records that may cause pages to be rewritten during redo is
$\{3\, , 6\, , 8\}$. \textit{Notice that this set \textbf{does not} contain the update
with LSN = 5; this is because its LSN is less than the recLSN for P99 (which is
8).}


\paragraph{Set of log records undone during undo}~\smallskip

The set of log records undone during undo is $\{8, 7, 6, 5\}$. This is because
intially, we have $\text{toUndo} = \{7, 8\}$, and during the undo phase we will
we will first undo 8, which adds 6 to toUndo; next, we undo 7, which adds 6 to
toUndo; then, we undo 6 and 5, neither of which add anything to toUndo, since
their PREV\_LSN are NULL.

\paragraph{Contents of the log after recovery completes}~\smallskip

\begin{minted}{text}
LSN | PREV_LSN | XACT_ID | TYPE       | PAGE_ID | UNDONEXTLSN
----+----------+---------+------------+---------+------------
1   | -        | -       | begin CKPT | -       | -
2   | -        | -       | end CKPT   | -       | -
3   | NULL     | T3      | update     | P42     | -
4   | 3        | T3      | commit     | -       | 3
5   | NULL     | T1      | update     | P99     | -
6   | NULL     | T2      | update     | P307    | -
7   | 6        | T2      | abort      | -       | 6
8   | 5        | T1      | update     | P99     | 5
+++ CRASH +++ CRASH +++ CRASH +++ CRASH +++ CRASH +++ CRASH +++
9   | 4        | T3      | end        | -       | -
10  | 8        | T1      | CLR        | P99     | 5
11  | 7        | T2      | abort      | -       | 6
12  | 6        | T2      | CLR        | P307    | NULL
13  | 12       | T2      | end        | -       | -
14  | 5        | T1      | CLR        | P99     | NULL
15  | 14       | T1      | end        | -       | -
\end{minted}

First, notice that since T3 had committed before the crash, an end record is
written for T3 during analysis.
\smallskip

Then, undo commences, undoing log records in the order 8, 7, 6, and finally 5.
Since T2 aborted at LSN = 7, this is re-recorded. Log records for T2 and T1 are
written after they have been fully undone.

\Sectend
