
\newpage
\section{Reliabilty}

\subsection{Lamport and Vector Clock}

\begin{itemize}
  \item \textbf{Lamport and Vector Clock, 5.1.1:}
        \textit{Write the Lamport logical clock of each event.}
\end{itemize}

\begin{minted}{text}
e1  = 1
e2  = 2
e3  = 2
e4  = 3
e5  = 3
e6  = 3
e7  = 4
e8  = 4
e9  = 5
e10 = 6
e11 = 5
\end{minted}


\begin{itemize}
  \item \textbf{Lamport and Vector Clock, 5.1.2:}
        \textit{Write the vector clock of each event.}
\end{itemize}

\begin{minted}{text}
e1  = (1, 0, 0)
e2  = (1, 1, 0)
e3  = (2, 0, 0)
e4  = (2, 0, 1)
e5  = (3, 1, 0)
e6  = (2, 2, 0)
e7  = (4, 1, 0)
e8  = (2, 0, 2)
e9  = (5, 2, 0)
e10 = (6, 2, 1)
e11 = (4, 1, 3)
\end{minted}


\begin{itemize}
  \item \textbf{Lamport and Vector Clock, 5.1.3:}
        \textit{Can P1 tell that the message received at e10 was actually sent
        before the one received at e9 using either Lamport or vector clock?}
\end{itemize}


\paragraph{Using Lamport timestamps}~\smallskip

If in real time some event $e_i$ happened before another event $e_j$, then
$\text{LC}(e_i) < \text{LC}(e_j)$, where $\text{LC}(e)$ is the Lamport clock of
some event $e$. However, the implication does not go the other way!
\smallskip

Hence, P1 cannot use the Lamport timestamps to tell that $e4$ happened before
$e6$, only that $e9$ happened before $e10$.


\paragraph{Using vector timestamps}~\smallskip

Yes! P1 can use ector timestamps to tell that $e4$ happened before $e6$.
\smallskip

This is because with vector clocks, the aforementioned implication \emph{does}
holds the other way - ie. if $\text{VC}(e_i) < \text{VC}(e_j)$, where
$\text{VC}(e)$ is the vector clock for event $e$, then $e_i$ happened before
$e_j$ in real time.



\begin{itemize}
  \item \textbf{Lamport and Vector Clock, 5.1.4:}
        \textit{What are the correct responses of Shipment to the two NewOrder
        messages, and how can Shipment enforce these responses?}
\end{itemize}

Upon receiving the two NewOrder messages for the same OrderID, Shipment should
respond that this order has already been cancelled. Shipment can enforce this
response by keeping track of cancelled orders, even if those orders have not
been seen by Shipment yet (this of course assumes that Order never repeats
OrderIDs).


\sectend

\subsection{Replication}

\begin{itemize}
  \item \textbf{Replication, 5.2.1:}
        \textit{Compare pros and cons of synchronous and asynchronous
        replication for replicating Shipment.}
\end{itemize}

\emph{In the following I will assume that the replicas are only used to backup
state and receive messages, and that only the main Shipment server sends
back messages, since the exam text does not state otherwise.} \medskip

Asynchronous replication, of course, has the advantage over synchronous
replication in that it requires significantly less bandwidth, which might be
useful if replicas reside far away from the main Shipment server. However,
synchronous replication has the immense benefit that fail-over is much faster in
case of main Shipment server failure.

\begin{itemize}
  \item \textbf{Replication, 5.2.2:}
        \textit{Would a write quorum of 3 and a read quorum of 2 be problematic
        for 5 replicas of Shipment?}
\end{itemize}

$Q_w = 3$ is ideal, since $3 > 5/2$; in other words, only one
transaction can write a time. \smallskip

However, if $Q_w = 3$, then $Q_r$ needs to also be at least 3, since otherwise
two transactions can read and write at the same time. So yes, this setup is
problematic!


\begin{itemize}
  \item \textbf{Replication, 5.2.3}
\end{itemize}

Blank.

\Sectend
