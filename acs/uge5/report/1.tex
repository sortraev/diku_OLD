\newpage
\section{Implementation}

I implement all the specifications given in the assignment text. I omit it from
the report since it is not specified to include documentation in the report, and
I don't want to go too much over the page count.
\medskip

In any case, my implementation is attached to the hand-out, and the source code
is self-documenting via appropriate code comments.

\tykstreg


\newpage
\section{Experiment setup}

\begin{itemize}
  \item \textit{Describe in detail the setup you have created for your
    experiments.}
\end{itemize}

\paragraph{Sample data generation}~
\smallskip

When generating random sets of \ms{ImmutableStockBook}s, I of course want to
generate books with different properties (title, price, number of sales misses,
and so on). \medskip

I write a function that generates a single \ms{StockBook}. The function randomly
generates each of the parameters to the \ms{ImmutableStockBook} constructor
except for the ISBN, which is passed as a parameter to the function. This is to
more easily generate sets of books with all unique ISBNs.

\begin{minted}{java}
public StockBook nextStockBook(int isbn) {
  String title  = "title_"  + isbn; // suffix books with ISBN to guarantee unique names.
  String author = "author_" + isbn;

  float price   = rand.nextFloat() * 50 + 0.1f;  // price from 0 to 50.
  int numCopies = rand.nextInt(61) + 20;         // 20 to 80 copies.


  long numSaleMisses = rand.nextInt(11);         // 0 to 10 sales misses.
  long numTimesRated = rand.nextInt(41);         // 0 to 40 times rated.
  long averageRating = rand.nextInt(6) + 1;      // average rating between 1 and 5.
  long totalRating = numTimesRated * averageRating;

  // one third of books is an editor pick.
  boolean isEditorPicked = rand.nextInt() % 3 == 0;

  return new ImmutableStockBook(isbn, title, author,
          price, numCopies, numSaleMisses,
          numTimesRated, totalRating, isEditorPicked);
}
\end{minted}

\ms{nextSetOfStockBooks()} then calls this in a loop over a pre-generated set of
unique ISBNs. \medskip

\textit{Note that the title is simply the word ''\ms{title_}'' suffixed with the ISBN.
ISBNs are always chosen in the inclusive integer range $\{1,..., 10^9\}$ (to
avoid collisions during generation), so titles are between 7 and 16 characters
long. Author fields are generated in the same fashion, which means author
strings are between 8 and 17 characters long.}

\paragraph{Workload configuration}~\smallskip

When testing with large numbers of worker thread (>100), I found that the
success rate would sometimes drop as low as 95\%. To counter this, I modified
some of the workload configuration parameters in the \ms{WorkloadConfiguration}
class. \medskip

Below snippet shows those configuration parameters that I changed:

\begin{minted}{java}
client.workloads.WorkloadConfiguration.numEditorPicksToGet: 20 -> 10
client.workloads.WorkloadConfiguration.numAddCopies:        20 -> 8
client.workloads.WorkloadConfiguration.numBooksToAdd:       10 -> 4
\end{minted}

With these changes, I never experienced failing interactions. \medskip

No other configuration parameters are changed, meaning tests are still run 500
times (following 100 warm up runs).


\paragraph{Hardware and hyperparameters}~\smallskip

All experiments (local and otherwise) are run on my local machine, featuring an
intel i7-7700hq 2.8GHz CPU. My machine also has 8GB of RAM, which might be
useful when the number of worker threads become large\footnote{To be honest, I
am not familiar with how or whether JVMs utilize 64-bit memory address spaces.}.
\medskip

To test a varying number of worker threads, I have changed various
hyperparameters in the source code to allow more simultaneous worker threads and
HTTP proxy clients. Specifically, I have made the following changes:

\begin{minted}{java}
client.BookStoreClientConstants.CLIENT_MAX_THREADSPOOL_THREADS: 250 -> 512

server.BookStoreHTTPSERVER.MIN_THREADPOOL_SIZE: 10  -> 1
server.BookStoreHTTPSERVER.MAX_THREADPOOL_SIZE: 100 -> 512

utils.BookStoreConstants.BINARY_SERIALIZATION: false -> true
\end{minted}

Notice that I have also chosen binary serialization. This decision was made in
an effort to speed up non-local tests, which quickly became excruciatingly slow
for larger worker thread counts (> 100).

\streg

\section{Experiment execution}

I want to measure throughput and latency of the \ms{CertainWorkload}. I want to
benchmark using various numbers of worker threads (hereby denoted by $N$), and
for this I have chosen $N = 2^i$ for $i \in \{0, ..., 9\}$, ie. all powers of
$2$ from $1$ to $512$. \medskip

I execute all tests both locally and non-locally. \medskip

All tests are run on the hardware and under the circumstances described above.

\section{Experiment results}

\paragraph{Throughput}~\smallskip

Plot 1 shows the measured throughput:

\pic{certainbookstore_thruput.png}

\textit{Please note that the plot is logarithmic.}\medskip

The highest throughputs are measured for $N = 256$ and 512 for the local and
non-local tests, respectively. As expected, throughput grows with $N$, but
whereas there is a peak at $N = 256$ for local tests, the maximum throughput for
non-local execution may require even larger $N$. \medskip

Not surprisingly, the local execution reaches throughput many orders of
magnitude bigger than the non-local tests - this of course due to the immense
overhead in RPC calls and the fact that here, the book store is run on an
entirely different JVM.

\paragraph{Latency}~\smallskip

Plot 2 shows the measured latency:

\pic{certainbookstore_latency.png}

\textit{Please note that the plot is logarithmic.}\medskip

Latency grows very quickly as $N$ increases for the non-local runs, whereas the
latency is relatively steady for $N \leq 32$ for the local runs, before it, too,
begings to grow linearly in $N$ (the measured latencies grow exponentially, but
so do the values of $N$). \medskip

Interestingly, but expectedly, latency is lower the smaller $N$ becomes (with
only few exceptions, which may or may not be due to errors in measurement); this
is of course due to the lessened traffic.

\streg

\newpage

\section{Reliability of results}

The experiments are not particularly exhausting of the actual performance of the
book store. Only one type of randomly generated books were used, and only one
set of workload configuration parameters were used. I used parameters that would
mostly produce rather \textit{small} interactions. For example, customers only
ever buy 3 book copies at a time, and stock managers only ever add 4 new books
at a time, and so on. Ideally, I would also have run a test with a large number
of books manipulated per interaction.

\tykstreg
