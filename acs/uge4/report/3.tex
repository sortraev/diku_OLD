\newpage
\section{Recovery Concepts}

\begin{itemize}
  \item \textit{1. In a force/steal system, is it necessary to implement
    a scheme for redo?}
\end{itemize}

If pages are forced to disk before a transaction is allowed to commit, rather
than flushing when it is convenient, then there is never a need to redo
transactions that have already committed.


\begin{itemize}
  \item \textit{What about undo?}
\end{itemize}

With \textit{steal}, transactions might push pages to disk which have not been
committed yet; if this is the case, then this must be undone.

\streg

\begin{itemize}
  \item \textit{2. What is the difference between volatile and non-volatile
    storage, and what types of failures are survived by each?}
\end{itemize}

Put shortly, volatile storage is only active when powered on, whereas
non-volatile storage retains its state when the power is on. Examples are main
memory and disk memory, respectively. \medskip

Volatile storage is lost on crashes, whereas non-volatile storage is not;
however, both are vulnerable to hardware failure (bit flipping, regular wear and
tear, etc). Especially harddisks with moving parts are vulnerable to failures
from wear and tear.

\streg

\begin{itemize}
  \item \textit{3. In WAL, which are the two situations in which the log tail is
    forced to disk?}
\end{itemize}

First, when pages are updated, a log record for that update is forced to disk before
the page is actually modified in memory.\smallskip

Second, before a given transaction is allowed to commit, all log records for
that transaction must be forced to disk.


\begin{itemize}
  \item \textit{Why are log forces necessary, and why are they sufficient for
    durability?}
\end{itemize}

One big reason they are necessary is the no-force policy, since the log may be
the only record of certain committed transactions having even taken place.

They are sufficient for durability because they record all \textit{intended}
mutations before they are carried out, and that is all that is needed for
redo/undo during recovery.

\tykstreg
