\newpage
\section{More ARIES}

\begin{itemize}
  \item \textit{1. Map the placeholders \ms{A-D} to the correct \ms{LSN}s.}
\end{itemize}

\ms{A = 5}, since this \ms{CLR} is for the \ms{update} record with \ms{LSN = 6}.
\smallskip

\ms{B = 4}, since this \ms{CLR} is for the \ms{update} record with \ms{LSN = 8}.
\smallskip

\ms{C = D = NULL}.

\streg

\begin{itemize}
  \item \textit{2. What are the states of the transaction and dirty page tables
    after the analysis phase?}
\end{itemize}


\begin{minted}{text}
/-----------------------------\
|      Transaction Table      |
|-----------------------------|
| xact id | status | last_LSN |
|---------+--------+----------|
| T1      | U      | 15       |
| T2      | U      | 16       |
\-----------------------------/

/--------------------\
|  Dirty Page Table  |
|--------------------|
| page id | rec_LSN  |
|---------|----------|
| P3      | 3        |
| P1      | 4        |
| P2      | 6        |
| P4      | 8        |
\--------------------/
\end{minted}

\ms{T3} is not in the transaction table after the analysis phase, since there
exists an \ms{end} record for it. \ms{T1} and \ms{T2} are in the transaction
table since, and their \ms{lastLSN}s are the last \ms{CLR} corresponding to their
respective abort recoveries; their statuses are set to \ms{U} since they have
not committed. \smallskip

The dirty page table contains all four pages and their earliest associated
\ms{LSN}s after the last checkpoint.

\streg

\begin{itemize}
  \item \textit{3. Show the contents of the log after the crash recovery
    completes.}
\end{itemize}

\begin{minted}{text}
LSN | PREV_LSN | XACT_ID | TYPE       | PAGE_ID | UNDONEXTLSN | toUndo after step
----+----------+---------+------------+---------+-------------+
1   | -        | -       | begin CKPT | -       | -           |
2   | -        | -       | end CKPT   | -       | -           |
3   | NULL     | T1      | update     | P3      | -           |
4   | 3        | T1      | update     | P1      | -           |
5   | NULL     | T2      | update     | P1      | -           |
6   | 5        | T2      | update     | P2      | -           |
7   | NULL     | T3      | update     | P2      | -           |
8   | 4        | T1      | update     | P4      | -           |
9   | 7        | T3      | commit     | -       | -           |
10  | 8        | T1      | abort      | -       | -           |
11  | 6        | T2      | abort      | -       | -           |
12  | 11       | T2      | CLR        | P2      | 5           |
13  | 9        | T3      | end        | -       | -           |
14  | 10       | T1      | CLR        | P4      | 4           |
15  | 14       | T1      | CLR        | P1      | NULL        |
16  | 12       | T2      | CLR        | P1      | NULL        |
----------------------CRASH AND RESTART-----------------------|
17  | 16       | T2      | CLR        | P1      | 12          |
18  | 17       | T1      | CLR        | P1      | 14          |
19  | 18       | T1      | CLR        | P4      | 10          |
19  | 18       | T1      | CLR        | P1      | 11          |
20  | 19       | T1      | CLR        | -       | 6           |
21  | 20       | T1      | CLR        | -       | 8           |
22  | 21       | T1      | CLR        | P4      | 4           |
23  | 22       | T2      | CLR        | P2      | 5           |
24  | 23       | T2      | CLR        | P1      | NULL        |
25  | 24       | T1      | CLR        | P1      | 3           |
26  | 25       | T1      | CLR        | P3      | NULL        |
\end{minted}

I'm pretty sure there is something wrong with my solution. Some tips or pointers
would be much appreciated! I think there is something wrong with my understanding of how to backtrace
\ms{CLR}s.


\begin{itemize}
  \item \textit{4a and 4b.}
\end{itemize}

I unfortunately did not have time to finish these.



\tykstreg
