\section{Questions for discussion on architecture}

\subsection{1. Short description of implementation and tests}

\begin{itemize}
    \item \textit{(a) How does your implementation and tests address the all-or-nothing semantics?}
\end{itemize}

\paragraph{All-or-nothing semantics in implementation}~\smallskip

\noindent Of those methods that I have implemented for the purposes of this assignment,
only\\ \ms{CertainBookStore.rateBooks()} needs explicit handling of all-or-nothing
semantics. However, this is simply covered by iterating the entire input
\ms{bookRatings} set, asserting that each \ms{BookRating} is valid, failing
immediately when a rating isn't, and \textit{only then} registering the book
ratings. \medskip

In the below snippet, lines 11-19 show how each \ms{BookRating} is validated
before ratings are registered in lines 22-26:


\begin{minted}[highlightlines={11-19, 22-26}]{java}
@Override
public synchronized void rateBooks(Set<BookRating> bookRatings) throws BookStoreException {

  // validate all ratings in bookRatings.
  for (BookRating bookRating : bookRatings) {
    int isbn   = bookRating.getISBN();
    int rating = bookRating.getRating();

    // a BookRating is valid if its isbn is valid and corresponds to an existing book,
    // and if its rating is a number in the range 0..5.
    if (BookStoreUtility.isInvalidISBN(isbn)) {
      throw new BookStoreException(BookStoreConstants.ISBN + isbn + BookStoreConstants.INVALID);
    }
    else if (!bookMap.containsKey(isbn)) {
      throw new BookStoreException(BookStoreConstants.ISBN + isbn + BookStoreConstants.NOT_AVAILABLE);
    }
    else if (BookStoreUtility.isInvalidRating(rating)) {
      throw new BookStoreException(BookStoreConstants.RATING + rating + BookStoreConstants.INVALID);
    }
  }

  for (BookRating bookRating : bookRatings) {

    BookStoreBook book = bookMap.get(bookRating.getISBN()); // the actual existence of a book with
    book.addRating(bookRating.getRating());                 // the given ISBN was checked earlier.

  }
}
\end{minted}

\newpage
\paragraph{All-or-nothing semantics in testing}~\smallskip

\noindent Below snippet is of the test case asserting all-or-nothing semantics of
\ms{CertainBookStore}\\\ms{.rateBooks()}. An invalid \ms{BookRating} is created in line
16 and registered in line 20, and in line 26 I assert that no book has received
any rating.

\begin{minted}[highlightlines={16, 20, 26}]{java}
@Test
public void testRateBooksAllOrNothing() throws BookStoreException {

  // add a couple more default books.
  initializeMoreBooks();

  // assert that no book has been rated yet.
  storeManager.getBooks().forEach(book -> assertEquals(0, book.getTotalRating()));

  Set<BookRating> ratings = new HashSet<>();

  // create one valid and one invalid BookRating.
  int valid_rating   = 3;
  int invalid_rating = -42;
  ratings.add(new BookRating(TEST_ISBN,  valid_rating));
  ratings.add(new BookRating(TEST_ISBN2, invalid_rating));

  // register the new ratings. this fails, since the second rating is invalid.
  try {
    client.rateBooks(ratings);
  }
    catch (BookStoreException ignored) {
  }

  // assert all-or-nothing semantics by verifying that all books still have rating 0.
  storeManager.getBooks().forEach(book -> assertEquals(0, book.getTotalRating()));
}
\end{minted}

\begin{itemize}
  \item \textit{(b) How did you test whether the service behaves according to the
      interface regardless of use of RPCs or local calls?}
\end{itemize}

\noindent I simply executed my tests both locally and non-locally by setting the
\ms{localTest} private field of each test class to \ms{true} and \ms{false},
respectively, before running my test suites.

\streg


\newpage

\subsection{2. Strong Modularity in the Architecture}

\begin{itemize}
  \item \textit{(a) In which sense is the architecture strongly modular?}
\end{itemize}

\noindent The architecture is for example modular in the sense that it allows different
types of clients (and stock managers), as long as these adhere to the BookStore
interface. In addition, the fact that the bookstore is implemented as a simple
HTTP server means that clients do not even have to be Java threads, as long as
they adhere to the RPC message format.

\begin{itemize}
  \item \textit{(b) What kind of isolation and protection does the architecture
    provide between the two types of clients and the bookstore service?}
\end{itemize}

Blank.

% The architecture places proxies (or rather, proxy interfaces) between clients
% and the bookstore server. The bookstore expects RPC communication only, and the
% client classes adhere to this protocol by using proxies to send RPCs over HTTP.

\begin{itemize}
  \item \textit{(c) How is enforced modularity affected when we run clients and
    services locally in the same JVM, as possible through our test cases?}
\end{itemize}

\noindent When \ms{localTest} is set to \ms{true} in either test class, then the private
fields \ms{storeManager} and \ms{client} are each set to the same instance of
the \ms{CertainBookStore} class. If either the client or store manager fails,
then the other fails aswell.

\streg

\newpage

\subsection{3. Naming service in the architecture}

\begin{itemize}
  \item \textit{(a) Is there a naming service in the architecture? If so, what is its
  functionality?}
\end{itemize}

\noindent I would say no - if I'm wrong, then either I can't find it or I'm unsure what
is meant by ''naming service''.

\begin{itemize}
  \item \textit{(b) Describe the naming mechanism that allows clients to discover and
  communicate with services.}
\end{itemize}

\noindent Blank :)

% \textit{This question is a little ambiguous, as it can refer to both the
% CertainBookStore clients/services, as well as clients and services in general. I
% assume that the question pertains to the general case.}
% \smallskip


% UUUHH.

\streg


\subsection{4. RPC semantics in the architecture}

\noindent \textit{Note: I should first say that I am not familiar with the behind-the-scenes of the
various \ms{jetty} HTTP libraries we use.} \medskip

I would argue that the architecture implements \textit{at-most-once} RPC
semantics, since HTTP requests are never retried. This is a good idea for the
book store, since a number of RPC's modify the book store in some way - we
wouldn't want to accidentally too many copies of the same book, or to add
duplicates of the same book to the store front.

\streg


\subsection{5. Web proxies}

\begin{itemize}
  \item \textit{(a) Is it safe to use web proxy servers with the architecture in
  Figure 1?}
\end{itemize}

\noindent Yes.

\begin{itemize}
  \item \textit{(b) If so, explain why this is safe and describe in between which
  components these proxy servers should be deployed. If not, why not?}
\end{itemize}

\noindent Proxy servers could be placed between clients and the book store server. This could
alleviate incoming traffic to the book store server via eg. caching and traffic
control. \smallskip

I would argue that it is safe so long as the system is coherent. This would be a
complex system to fault-secure, but would ultimately be a more secure, tolerant
system than letting a single server receive all incoming requests in real-time.

\streg

\subsection{6. Bottlenecks in the architecture}
\begin{itemize}
  \item \textit{(a) Is/are there any scalability bottleneck/s in this architecture
  with respect to the number of clients?}
\end{itemize}

\noindent Yes!

\begin{itemize}
  \item \textit{If so, where is/are the bottleneck/s? If not, why can we infinitely scale the
number of clients accessing this service?}
\end{itemize}

\noindent There is a bottleneck in that for the moment, only a single BookStoreHTTPServer
handles all clients. Infinitely scaling the number of clients would mean huge
queues of incoming HTTP requests and very plausibly dropped requests.

\streg

\subsection{7. In the Event of CertainBookStore crashing}
\begin{itemize}
  \item \textit{(a) Would clients experience failures differently if web proxies were used in the
architecture?}
\end{itemize}

\noindent Definitely. The result of all RPCs which have no side effects can be stored in
these proxies, such that they can be served to users even when the main
bookstore server is temporarily down.

\begin{itemize}
  \item \textit{(b) Could caching at the web proxies be employed as a way to mask failures from
          clients?}
\end{itemize}

\noindent This depends on whether the proxy is able to cache requests and re-attempt
forwarding to the CertainBookStore once it restarts. If we assume this is not
possible, then caching would only be able to mask failure for those requests
which only \textit{read} from the book store, such as getting the selection of
books or the editor picks, but would not be able to serve eg. purchase requests
before the CertainBookStore was back up and running.

\begin{itemize}
  \item \textit{(c) How would the use of web caching affect the semantics offered by
  the bookstore service?}
\end{itemize}

\noindent Blank.

\tykstreg

