\section{Question 3: Serializability and locking}

In the following, I will use \ms{S(V)} and \ms{X(V)} to denote shared and
exclusive locking of a variable \ms{V}, respectively. Since in the S2PL a commit
incurs an implicit phase 2-transition and unlocking of all locks, I will omit
phase 2-transitions and unlocks.

\begin{itemize}
\item \textit{Draw the precedence graph for each schedule. Are the schedules
conflict-serializable?}

\paragraph{Schedule 1}

\begin{minted}{text}
T1: W(Y)             R(X) R(Z) C
T2:      W(Z) C
T3:             R(X)             R(Z) W(Y) C
\end{minted}

T1 and T3 share a WR conflict on Y, producing a dependency (T1, T3).
\smallskip

T2 and T1 share a WR conflict on Z, producing a dependency (T2, T1).
\smallskip

T2 and T3 share a WR conflict on Z, producing a dependency (T2, T3).

\pic{schedule_1_dependency_graph.png}

The dependency graph has no cycles, so schedule 1 is conflict serializable.

\paragraph{Schedule 2}

\begin{minted}{text}
T1: W(Z)      W(X) C
T2:                  R(Y) R(Z) C
T3:      R(X)                    W(Y) C
\end{minted}

T1 and T2 share a WR conflict on Z => T1 -> T2
T3 and T1 share a RW conflict on X => T3 -> T1
T2 and T3 share a RW conflict on Y => T2 -> T3

\pic{schedule_2_dependency_graph.png}

The dependency graph has a cycle in T1 -> T2 -> T3 -> T1, so schedule 2 is
\textit{not} serializable.



\paragraph{Schedule 1}~\smallskip

Yes, schedule 1 could have been produced by an S2PL protocol, for example with
the locking seen in the snippet below:

\begin{minted}{text}
    |time 0  |time 1 |time 2      |time 3    |time 4
T1: |P1 X(Y) |W(Y)   |            |          |S(X) S(Z) R(X) R(Z) C
T2: |P1      |       |X(Z) W(Z) C |          |
T3: |P1      |       |            |S(X) R(X) |

    |time 5
T1: |
T2: |
T3: |S(Z) X(Y) R(Z) W(Y) C
\end{minted}

Explanation:

\begin{itemize}
\item[time 0] T1, T2, and T3 each enter P1. T1 X-locks Y.
\item[time 1] T1 writes Y.
\item[time 2] T2 X-locks Z; writes Z; commits and releases X-lock of Z.
\item[time 3] T3 S-locks X; reads X.
\item[time 4] T1 S-locks X and Z; reads X and Z; commits and releases S-locks
  of X and Z.
\item[time 5] T3 S-locks Z and X-locks Y; reads Z and writes Y; commits and
  releases S-lock of Z and X-lock of Y.
\end{itemize}


\paragraph{Schedule 2}~\smallskip

No, schedule 2 could \textit{not} have been produced by an S2PL protocol. As
seen in the snippet below, there is an unavoidable deadlock:

\begin{minted}{text}
    |time 0       |time 1    |time 2
T1: |P1 X(Z) W(Z) |          |DEADLOCK W(X) C
T2: |P1           |          |                R(Y) R(Z) C
T3: |P1           |S(X) R(X) |                            W(Y) C
\end{minted}


Time-step explanation:
\begin{itemize}
\item[time 0] T1, T2, and T3 each enter P1. T1 exclusive-locks Z and writes Z.
\item[time 1] T3 share-locks X and reads X.
\item[time 2] T1 wants to write X and thus needs an exclusive lock of X.
              T3 still holds its shared lock of X and cannot release it until
              it has commited. Deadlock!
\end{itemize}
