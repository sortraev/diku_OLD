\newpage
\section{Question 4: Optimistic Concurrency Control}

\subsection{Notes on my answers}

\subsubsection{Test formulation}

For each of the three scenarios, at least one of the three tests (as described
in eg. the lecture 4 slides) must hold. Since in each schedule we consider three
transactions, and are only concerned with validating the third transactions, I
use the following specialized formulations of the validation tests:

\paragraph{Test 1}~\smallskip
For $i \in \{1, 2\}$, $T_i$ must complete before $T_3$ starts.

\paragraph{Test 2}~\smallskip
For $i \in \{1, 2\}$, $T_i$ must complete before $T_3$ begins its write phase
\textbf{and} the intersection between WriteSet($T_i$) and ReadSet($T_j$) must be
empty.

\paragraph{Test 3}~\smallskip
For each $i \in \{1, 2\}$, $T_i$ must complete its read phase before $T_3$
completes its own \textbf{and} the intersection of WriteSet($T_i$) and
ReadSet($T_3$) must be empty \textbf{and} the intersection of WriteSet($T_i$)
and WriteSet($T_j$) must be empty.

\subsubsection{Ambiguities in the given schedules}

The assignment text is very ambiguous - in particular with respect to statements
such as ``$T_i$ completes before $T_j$ begins with its write phase``. In this particular
case, it is for example not possible to argue whether $T_i$ finishes its read
phase before $T_j$ does.

Whenever there is such ambiguity, I will \textit{pessimistically} assume that
$T_i$ finishes as late as possible.


\subsection{Schedule 1}
\begin{minted}{text}
  T1: RS(T1) = {}, WS(T1) = {3},
    T1 completes before T3 starts.

  T2: RS(T2) = {2, 3, 4, 6}, WS(T2) = {4, 5},
    T2 completes before T3 begins with its write phase.

  T3: RS(T3) = {3, 4, 6}, WS(T3) = {3}
\end{minted}

$T_1$ completes before $T_3$ starts, but $T_2$ completes \textit{after} $T_3$
starts, and so \textbf{test 1 fails}. \medskip

\textbf{Test 2 fails} because the intersection of WriteSet($T_1$) and ReadSet($T_3$)
is the non-empty set $\{3\}$. \medskip

\textbf{Test 3 fails} for the same reason that test 2 failed (the write set of $T_1$
overlaps with the read set of $T_3$). \bigskip

All three tests failed, and so $T_3$ must be rolled back.

\subsection{Schedule 2}

\begin{minted}{text}
  T1: RS(T1) = {5, 6, 7}, WS(T1) = {8},
    T1 completes read phase before T3 does.

  T2: RS(T2) = {1, 2, 3}, WS(T2) = {5},
    T2 completes before T3 begins with its write phase.

  T3: RS(T3) = {3, 4, 5, 6}, WS(T3) = {3}
\end{minted}

\textbf{Test 1 fails} since both $T_1$ and $T_2$ end \textit{after} $T_3$ starts.
\medskip

$T_2$ completes before $T_3$ begins its write phase, but it is not stated
whether $T_1$ does. I assume it does not, and thus \textbf{test 2 fails}.
\medskip

$T_1$ completes its read phase before $T_3$ does, but it is \textit{not} stated
whether $T_2$ also does. I assume it does not, and thus \textbf{test 3 also
fails}.

Again, all three tests fail, so $T_3$ must be rolled back.

\subsection{Schedule 3}

\begin{minted}{text}
T1: RS(T1) = {5, 6, 7}, WS(T1) = {8},
  T1 completes read phase before T3 does.

T2: RS(T2) = {1, 2, 3}, WS(T2) = {5},
  T2 completes before T3 begins with its write phase.

T3: RS(T3) = {3, 4, 5, 6}, WS(T3) = {3}
\end{minted}

\textbf{Test 1 fails} since both $T_1$ and $T_2$ end \textit{after} $T_3$ starts.
\medskip

\textbf{Test 2 fails} since under the specifications, $T_1$ does not necessarily finish
before $T_3$ begins its write phase. \medskip

\textbf{Test 3 fails} since under the specifications, $T_2$ does not necessarily finish
its read phase before $T_3$. \medskip

Once again, all three tests fail, so $T_3$ must be rolled back.

\tykstreg
