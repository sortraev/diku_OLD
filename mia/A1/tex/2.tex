\newpage
\newgeometry{top=4.5cm}
% \vspace{-2cm}
\section{Image reconstruction via backprojection}

% \subsection{Reconstruction using the inverse Radon transform}

As mentioned, Radon also described a method for reconstructing an object
function based on its Radon transformation, and gave his formular for the
inverse Radon transform. In image analysis the process of reconstructing images
from projections using the (multiple invocations of) inverse Radon transform is
aptly called \textit{backprojection}.

\medskip

The backprojection is performed by ``shooting'' projections back along the
angles at which they were originally acquired during the forward transform, and
is perhaps most easily expressed in terms of a 2D inverse Fourier transform of
$F\{\mbf R_\theta\{f(u, v)\}\}$, the Fourier transform of the Radon transform of
the object function:

\begin{alignat}{3}
  \hat f(x, y) &= \infint \infint F\{\mbf
R_\theta\{f(u, v)\}\} e\, ^{i\tau(xu + vy)}\
  du\,dv,\label{eq:bp}\\
   \text{with}\  (u, v) &= (\omega \cos \theta, \omega \sin \theta)\nonumber,
\end{alignat}
and where $\omega$ is the Fourier frequency component; $\hat f(x, y)$ is then the
reconstructed object function.

\medskip

Below \cref{fig:no-filter} shows Radon transformations and subsequent
backprojection reconstructions of the famous Shepp-Logan phantom, at relatvely
low and high number of projection angles, respectively, using the \texttt{radon()}
and \texttt{iradon()} functions from the \texttt{skimage.transform} Python library.

\begin{figure}[H]
  \includegraphics[width=\textwidth]{figures/no_filter.png}
  \vspace{-0.8cm}
  \caption{\small \textit{Radon transform and backprojection of the Shepp-Logan phantom at
  various projection and backprojection angles (no filtering).}}
\label{fig:no-filter}
\end{figure}

\subsection{Filtered backprojection}

The reconstructions in above \cref{fig:no-filter} are obviously \textit{blurry} in
comparison to the original image. One common issue with backprojection for image
reconstruction is blurriness, which arises from the fact that in 2D frequency space,
low frequencies reside closer to the origin while high frequencies lie further
away, creating an overrepresentation of lower frequencies in the backprojection,
ultimately resulting in a blur in the image domain, since the high frequencies
in frequency space correspond to the finer details in the image
domain\cite{toft}.

One remedy is to apply a high pass filter to the frequency space (eg. the
Fourier transform of the Radon transform, hence \cref{eq:bp}), dampening low
frequencies and, consequently, accentuating high frequencies before projecting
back (ie. applying the 2D inverse Fourier, as in \cref{eq:bp}) -- the
combination of filtering and backprojection is called \textit{filtered
backprojection} (FBP for short).

\medskip

Below \cref{fig:filter} shows the same reconstructions as before, but
this time performed using FBP with a simple ramp
filter\footnote{\texttt{skimage.transform.iradon} offers a number of filters,
but all of them are different types of high pass filters and thus may produce visually similar
results; hence only the ``ramp'' filter is demonstrated for this example, but
note that other filters may better handle high frequency noise.}.

\begin{figure}[H]
  \centering
  \includegraphics[width=0.8\textwidth]{figures/filter_vs_no_filter.png}
  \vspace{-0.2cm}
  \caption{\small \textit{Radon transform and FBP of the Shepp-Logan phantom at
  various projection and backprojection angles using a ramp filter.}}
\label{fig:filter}
\end{figure}


Like the forward Radon transform, FBP is also a linear transformation,
\textit{so long as the chosen filtering function is linear}. I shall not attempt
to prove this fact, but rather illustrate it using
\texttt{skimage.transform.iradon()} in below \cref{fig:fbp-linearity}:

\begin{figure}[H]
  \centering
  \includegraphics[width=\textwidth]{figures/fbp_linearity.png}
  \vspace{-0.2cm}
  \caption{\small \textit{Comparing sum of two partial reconstructions with a
  reference reconstruction (using ramp filter). Note the very small MSE of
  $9.67\text{e}-33$ in the difference plot -- because of its small magnitude, I
  argue that this error is due to float
  imprecisions.}}
\label{fig:fbp-linearity}
\end{figure}

% Note the very small MSE of $9.67\text{e}-33$ in the difference image between the
% reference reconstruction and the sum of partial reconstructions -- I argue that
% because the error is this small, it most likely stems from float imprecisions.

\sectend
