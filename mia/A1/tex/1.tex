\section{The Radon Transform}

In broad and sparse terms, the Radon transform $\mbf R\{f(\mbf x)\}$ of some
object function $f(\mbf x)$ for $\mbf x \in \mathbb R^n$ is a transformation
from the spatial domain $\mathbb R^n$ onto some chosen projection space in
$\mathbb R^{n - 1}$ -- the transform is then given by the set of hyperplane
integrals over all hyperplanes in the spatial domain perpendicular to the
projection space. By changing the angle of projection, the $(n-1)$-dimensional
volume of the object function $f$ can be ``viewed'' from different
angles.

However, for the purposes of image de- and reconstruction, we shall restrict our
view to object functions of two dimensions, ie. images. 

Radon proved in 1917\cite{radon} that given infinitely many projections, the
original image can be accurately reconstructed. Since we are working with
computers, we are limited to a finite number of projection angles, and thus to
an \textit{approximation} of the original object function, the accuracy of
which, however, increases with the number of projection angles.

\subsection{Mathematical formular for the Radon transform}

Using Dhawan's\cite{dhawan} definition and notation, the 2D Radon transform
$\mbf R\{f(x, y)\}$ of some object function $f(x, y)$ is the projection $P(p,
\theta)$ in the polar coordinate system given by the line integral:
\begin{alignat*}{2}
  P(p, \theta) = \mbf R_\theta\{f(x, y)\} = \int_{L_\theta} f(x, y) \ dl
\end{alignat*}
where $x \cos \theta + y \sin \theta = p$ and $L_\theta \subset \mathbb R^2$ is
the set of lines with slope $\theta$. In other words, the Radon transform
computes the infinite number of line integrals over the parallel
paths whose slope is $\theta$, and thus $\text{img}(\mbf R_\theta\{f(x,
y)\})$ is two-dimensional.

\medskip

We might wish to express the transform in terms of a rectangular coordinate
system in the frequency domain. To do so, we rotate $(x, y)$ to $(p, q)$ by
choosing:
\begin{alignat*}{3}
  p &=  &x \cos \theta + y \sin \theta~\\
  q &= -&x \sin \theta + y \cos \theta,
\end{alignat*}
and then re-express the transform as:

\begin{alignat}{2}
  \mbf R_\theta\{f(x, y)\} =\int\limits_{-\infty}^\infty f(p \cos \theta - q \sin \theta, p \sin \theta + q
  \cos \theta) \ dq. \label{eq:R}
\end{alignat}


\subsection{Proof: Linearity of the Radon transform}
% I will only show linearity of the 2D Radon transform.
\begin{proof}
To prove linearity of the Radon transform $\mbf R_\theta$, it suffices to show:
\begin{align}
  \mbf R_\theta\{a\,f(x, y) + b\,g(x, y)\} = a \mbf R_\theta\{f(x, y)\} + b \mbf R_\theta\{g(x, y)\}
  \label{eq:proof_end}
\end{align}
for object functions (eg. images) $f(x, y)$ and $g(x, y)$, and
constants $a, b$.

\noindent The proof relies on the sum and constant coefficient rules for integration,
  which I shall not reiterate here.

Starting from the LHS and using \cref{eq:R}:
\begin{alignat*}{3}
  \mbf R_\theta\{a\,f(x, y) + b\,g(x, y)\} &=
    \int &a\,f(p \cos \theta - q \sin \theta,\ &p \sin \theta + q \cos
    \theta)\\[-8pt]
    &&+\, b\,g(p \cos \theta - q \sin \theta,\ &p \sin \theta + q \cos \theta)\
    dq.
\end{alignat*}
\noindent Using the sum rule and constant coefficient rules for integration:
\begin{align*}
  \mbf R_\theta\{a\,f(x, y) + b\,g(x, y)\} =
    &\quad \, \int a\,f(p \cos \theta - q \sin \theta,\ p \sin \theta + q \cos
    \theta)\ dq\\[-6pt]
     &+\, \int b\,g(p \cos \theta - q \sin \theta,\ p \sin \theta + q \cos \theta)\
    dq\\[4pt]
  =
    &\quad \, \,  a \int f(p \cos \theta - q \sin \theta,\ p \sin \theta + q \cos
    \theta)\ dq\\[-6pt]
     &+\ b \int g(p \cos \theta - q \sin \theta,\ p \sin \theta + q \cos \theta)\
    dq.
\end{align*}
\noindent Finally, using \cref{eq:R} in reverse on the summand integrals, I get:
$$
  \mbf R_\theta\{a\,f(x, y) + b\,g(x, y)\} =\ a \mbf R_\theta\{f(x, y)\} +\ b \mbf R_\theta\{g(x, y)\},
$$
  which is equivalent to \cref{eq:proof_end}, thus concluding the proof.
\end{proof}


\sectend
