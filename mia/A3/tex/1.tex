\section{Image registration in medical image analysis}

In medical image analysis we often find the need to compare different images,
whether it be images of the same subject but of different modalities, or images
of different subjects where we expect some sort of similarity (ie. images of the
same body part/s from different patients). In many cases, such a comparison is
not immediately applicable, either due to internal or external factors: Internal
factors are the anatomical differences from patient to patient, while the
external factors are things such as modality, exposure angle/duration, the
choice of tracer (in nuclear imaging), imaging noise, and so on.

The medical image analyst's tool of choice in these cases is \emph{image
registration}. Given \emph{fixed} and \emph{moving} images, registration is
a method which can be used to geometrically transform the moving image into
(something structurally resembling) the fixed image, using proper rigid
transformations (those which preserve object shape and orientation, ie.
translation and rotation, but not reflection), affine transformation (any
transformation which can be expressed as an affine function, which includes
scaling, shearing, and the rigid transformations), and non-rigid transformations
(which allow local, and even arbitrary deformations) of the coordinate space.

Each of the three types of transformations have their uses. For example, if the
same patient is imaged multiple times then we might expect low impact from
internal factors, but different cameras or operators might might result in a
different rotation or position of the subject relative to the camera, and this
requires only a rigid transformation. On the other hand, if we need to compare
images of body parts of organs of two different patients, we may need to account
for different shapes altogether, and here, we can use non-rigid transformation
to squeeze and stretch shapes in specific parts of the image -- however, in many
of these cases, an affine transformation may suffice, eg. if one patient simply
has a larger head than the other.

\sectend
