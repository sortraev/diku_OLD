% \documentclass[a4paper,12pt]{article}
%
% %% various math packages
\usepackage{amsthm}
\usepackage{amssymb}
\usepackage{amsmath}
\usepackage{amsfonts}
\usepackage{mathtools}
\usepackage{xifthen}
\usepackage{xparse}
\usepackage{dsfont}
\everymath{\displaystyle} % force display style for inline math
\newcommand\numberthis{\addtocounter{equation}{1}\tag{\theequation}} % equation numbering for align*

%% misc packages.
\usepackage{xfrac}
\usepackage{parskip}
\usepackage[utf8]{inputenc}
\usepackage{hyperref}
\usepackage{cleveref}
\usepackage{graphicx}
\usepackage{float}
\usepackage{subcaption}
\usepackage{enumitem}
\usepackage[outputdir=out]{minted}
\usemintedstyle{trac}
\setminted{%frame=lines,
    linenos=true,
    fontsize=\footnotesize
}
\usepackage[table,xcdraw]{xcolor}


%% math commands.
\DeclareMathOperator*{\argmin}{argmin} % argmin.
\newcommand{\bbar}[1]{\overline{#1}}   % a wider bar than \bar.
\newcommand{\eps}{\varepsilon}         % prettier epsilon.
\newcommand{\mbf}[1]{\mathbf{#1}}      % shorthand.
\newcommand{\R}{\mathbb R}             % set of real numbers.
\newcommand{\N}{\mathbb N}             % set of natural numbers.
\newcommand{\Z}{\mathbb Z}             % set of integers.
\let\emptyset\varnothing               % better emptyset symbol. 
\newcommand{\pr}[1]{\text{Pr}\left[#1\right]} % pretty probabilities.
\DeclareMathSymbol{*}{\mathbin}{symbols}{"01} % map asterisks to \cdot. use
                                              % \ast for asterisk in math mode. 

% bold and red TODO's.
\newcommand{\TODO}[1]{\textcolor{red}{TODO: #1}}


%
% \newcommand{\concat}{\ensuremath{+\!\!\!\!+\,}}
% \newcommand{\rpm}{\raisebox{.2ex}{$\scriptstyle\pm$}}
% \newcommand{\resub}[1]{\textcolor{red}{#1}}
% \newcommand{\blue}[1]{\textcolor{blue}{#1}}
% \newcommand{\ltt}{\sqsubseteq}
% \newcommand{\twodots}{\mathinner {\ldotp \ldotp}}
%
% \title{\large MIA -- Hand-in 3}
% \author{{\footnotesize Anders Lietzen Holst (wlc376)}\\ {\footnotesize \today}}
%
% \date{}

% % (L3C) Describe regularization.

% % (L3E) Implement a non-rigid registration in python, including a
% %         transformation model and a similarity measure.
%
% % (L3B) Describe the role of similarity measure.
% % (L3A) Describe what the registration problem is.
% % (L3D) Knowledge of the different types of transformations.


\iffalse{
  >> disposition

  section 1:
  * describe registration -- geometrically transform an image into (something
      geometrically resembling) another. useful tool to be able to compare
      different images of the same thing (or the same thing in different
      bodies). For example, if two images are created using the same modality
      (x-ray, MRI, etc) and are aligned wrt. the subject, determining the
      correlation may not be a problem at all, but sometimes we may eg. want to
      compare an x-ray to an MRI, or to compare an image of one body to the
      image of another, perhaps taken with no knowledge of the first image.

  * when is it used and why
  * supervised algorithm; needs training data


  section 2:
  steps of the algorithm -- in particular, describe similarity:

    pixelwise similarities (eg. MSE) and (normalized) cross correlation match
    pixel intensities, but we often want to match \textit{patterns}: hence
    mutual image information, based on joint entropy.

    The mutual information $MI(I, J)$ measures the amount of information one can
    obtain about an image $I$ from the image $J$, and vice versa, and is given
    by the double sum over image pixels:


\begin{align*}
  MI(\mathcal I, \mathcal J) &= \sum_{i \in \mathcal I} \sum_{j \in \mathcal J}
  p(i, j) \log{3}\TODO{this}
\end{align*}

  and an image registration is finished when the mutual information is
  \textit{maximized}.


  section 3:
  * different types of transformations: rigid, non-rigid

  * regularization?

}\fi



\documentclass{llncs}

%% various math packages
\usepackage{amsthm}
\usepackage{amssymb}
\usepackage{amsmath}
\usepackage{amsfonts}
\usepackage{mathtools}
\usepackage{xifthen}
\usepackage{xparse}
\usepackage{dsfont}
\everymath{\displaystyle} % force display style for inline math
\newcommand\numberthis{\addtocounter{equation}{1}\tag{\theequation}} % equation numbering for align*

%% misc packages.
\usepackage{xfrac}
\usepackage{parskip}
\usepackage[utf8]{inputenc}
\usepackage{hyperref}
\usepackage{cleveref}
\usepackage{graphicx}
\usepackage{float}
\usepackage{subcaption}
\usepackage{enumitem}
\usepackage[outputdir=out]{minted}
\usemintedstyle{trac}
\setminted{%frame=lines,
    linenos=true,
    fontsize=\footnotesize
}
\usepackage[table,xcdraw]{xcolor}


%% math commands.
\DeclareMathOperator*{\argmin}{argmin} % argmin.
\newcommand{\bbar}[1]{\overline{#1}}   % a wider bar than \bar.
\newcommand{\eps}{\varepsilon}         % prettier epsilon.
\newcommand{\mbf}[1]{\mathbf{#1}}      % shorthand.
\newcommand{\R}{\mathbb R}             % set of real numbers.
\newcommand{\N}{\mathbb N}             % set of natural numbers.
\newcommand{\Z}{\mathbb Z}             % set of integers.
\let\emptyset\varnothing               % better emptyset symbol. 
\newcommand{\pr}[1]{\text{Pr}\left[#1\right]} % pretty probabilities.
\DeclareMathSymbol{*}{\mathbin}{symbols}{"01} % map asterisks to \cdot. use
                                              % \ast for asterisk in math mode. 

% bold and red TODO's.
\newcommand{\TODO}[1]{\textcolor{red}{TODO: #1}}



\title{Mia Assignment 3}
\author{Anders Holst (wlc376)}
\institute{}

\begin{document}

\pagenumbering{arabic}
\maketitle

\section{Make Your Own (10 points)}

\begin{enumerate}

  \item \textit{How do you define $\mathcal X$?}

    For each student I would collect current GPA, class attendance, and
    average hand-in page count. Assuming the Danish 7 point grading scale, that
    class attendance is described by a number between 0 and 1, and that page
    count is a natural number, we have $\mathcal X = [-3, 12] \times [0, 1]
    \times \mathbb N$.


  % \item for each prior course taken by the particular student, I would collect
  %   data on the student's lecture/class attendance, assignment point grading
  %
  %   \TODO{what?}. Thus the sample space would be \TODO{$\mathcal X = ...$}.

  \item \textit{How do you define $\mathcal Y$?}

    The label space $\mathcal Y$ would be the set of possible grades --
    assuming the Danish 7 point grading scale, we would have $\mathcal Y = \{-3,
    00, 02, 4, 7, 10, 12\}$.

  \item \textit{How do you define $\ell(y', y)$?}

    I would want incorrect predictions to be punished harder the greater the
    difference between $y'$ and $y$ -- however, to better reduce the number of
    incorrect predictions in the extremes, I would use a square loss function:
    $\ell(y', y) = (y' - y)^2$.

  \item \textit{How do you define $d(x, x')$?}


    Since the three dimensions in $\mathcal X$ have very different ranges, I
    can't use a simple Euclidean distance for $d$, since for one this would weigh
    average page count much higher than class attendance percentage. Instead, in
    order to weigh the differences in all three dimensions equally, I use
    a symmetric relative difference function, and $d$ is defined as:

    $$d(x, x') = \sqrt{\sum_{\substack{i \in \{GPA,\\ attendance,\\ pages\}}}
    \left(\frac{|x_i - x'_i|}{x_i + x'_i}\right)^2}$$

  \item \textit{How would you evaluate your algorithm in terms of $\ell(y', y)$}?

    I am not \textit{entirely} sure that I interpret the question correctly.
    However, I would probably evaluate model accuracy using MSE such that bad
    predictions weigh heavier. Since $\ell$ is a square loss function, the MSE
    is:

    $$
    \frac{1}{n}\sum_{i = 0}^n \ell(y'_i, y_i)
    $$


  \item \textit{Do you expect any issues after deployment?}

    The biggest problem I can imagine is the issue of new students for whom
    there exists no prior data. To alleviate this, I would generate random data
    based on other students (with added random noise, of course).

\end{enumerate}

\sectend

\newpage
\section{Digits Classification with K Nearest Neighbors (45 points)}

\subsection{Task \#1}

\paragraph{How does $n$ affect fluctuation in validation error?}~\smallskip

\begin{figure}[H]
  \includegraphics[width=\textwidth]{figures/fig2_2.png}
\caption{\footnotesize \it Fluctuation in validation error for varying i as a function of k, for
  each n.}
\label{fig:2-2}
\end{figure}

While $i$ determines which subset of the data is used for validation, $n$
determines the \textit{size} of each of these validation subsets.

There is a very clear trend in the relationship the value of $n$ and fluctuation
in validation error across all values of $k$, but it is best illustrated by
examining the right end of the plot.

For high values of $k$ (roughly 30 and greater), there is an undeniable
indication that the higher the value of $n$, the smaller the fluctuation in
validation error across values of $i$. This is because as $n$ increases, the
differences between two given validation subsets are smoothened out -- in other
words, the smaller the $n$, the more the result is influenced by outliers in the
validation set.


\newpage
\paragraph{How does $K$ affect prediction accuracy?}~\smallskip

\begin{figure}[H]
  \includegraphics[width=\textwidth]{figures/fig2_1.png}
  \caption{\footnotesize \it Mean zero/one prediction error of KNN for varous i, n, k.}
\label{fig:2-1}
\end{figure}

Interestingly, the overall best $K$ seems to be $K^* = 1$, with almost zero
validation error for most values of $i$ across all $n$.

For all four values of $n$, prediction error seems to increase with $K$. This is
likely because our training set is only 100 digits, and as $K$ increases towards
half of the training set, the model moves closer towards simply estimating the
sample mean.


\subsection{Task \#2}

\begin{figure}[H]
  \includegraphics[width=\textwidth]{figures/fig2_3.png}
\caption{\footnotesize \it Mean zero/one prediction error of KNN for various degrees
         of data corruption; for n = 80 and varying i, k.}
\label{fig:2-3}
\end{figure}

\paragraph{How does corruption magnitude influence prediction accuracy and the optimal
value of K?}~\smallskip

Not surprising, there is a correlation between corruption magnitude and
prediction error -- for the uncorrupted set, the prediction error lies roughly
in the range $[0, 0.16]$, while prediction errors lie roughly in the ranges
$[0.02, 0.18]$, $[0.1, 0.3]$, and $[0.2, 0.4]$ for the lightly, moderately, and
heavily corrupted sets, respectively. For the heavily corrupted set, prediction
error reaches as high as roughly 40\% for $i = 5$.

As far as the optimal $K$, based on the plots we may still be inclined to simply
choosing small values of $K^*$ for the three corrupted sets, but the indication
is not as strong based on the plots alone, since the trend in the curves seem
flatter than that of the uncorrupted set -- this perhaps applies less so for the
lightly corrupted set.

Most interestingly, however, is perhaps the fluctuation in validation error
across values of $i$. This has not been examined explicitly, but for the heavily
corrupted set it appears as though the five curves stick closer together than
for the uncorrupted set.

\sectend



\begin{thebibliography}{8}
  \bibitem{wu}
    Wu, Zufeng; Lan, Tian; Wang, Jiang; Ding, Yi; Qin, Zhiguang. \textit{Medical
    Image Registration Using B-Spline Transform}. School of Information and
    Software Engineering, University of Electronic Science and Technology of
    China.

    \url{https://ijssst.info/Vol-17/No-48/paper1.pdf}.

\end{thebibliography}

\end{document}
