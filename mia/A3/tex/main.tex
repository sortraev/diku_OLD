% \documentclass[a4paper,12pt]{article}
%
% %% FROM THE PROVIDED math_symbols.tex
\usepackage{amsthm}
\usepackage{amssymb}
\usepackage{amsmath}
\usepackage{mathtools}
\usepackage{xifthen}
\usepackage{xparse}
\usepackage{dsfont}


% Left-right bracket
\newcommand{\lr}[1]{\left (#1\right)}

% Left-right square bracket
\newcommand{\lrs}[1]{\left [#1 \right]}

% Left-right curly bracket
\newcommand{\lrc}[1]{\left \{#1\right\}}

% Left-right absolute value
\newcommand{\lra}[1]{\left |#1\right|}

% Left-right upper value
\newcommand{\lru}[1]{\left \lceil#1\right\rceil}

% Scalar product
\newcommand{\vp}[2]{\left \langle #1 , #2 \right \rangle}

% The real numbers
\newcommand{\R}{\mathbb R}

% The natural numbers
\newcommand{\N}{\mathbb N}

% Expectation symbol with an optional argument
\NewDocumentCommand{\E}{o}{\mathbb E\IfValueT{#1}{\lrs{#1}}}

% Indicator function with an optional argument
\NewDocumentCommand{\1}{o}{\mathds 1{\IfValueT{#1}{\lr{#1}}}}

% Probability function
\let\P\undefined
\NewDocumentCommand{\P}{o}{\mathbb P{\IfValueT{#1}{\lr{#1}}}}

% A hypothesis space
\newcommand{\HH}{\mathcal H}

% A sample space
\newcommand{\XX}{\mathcal{X}}

% A label space
\newcommand{\YY}{\mathcal{Y}}

% A nicer emptyset symbol
\let\emptyset\varnothing

% Sign operator
\DeclareMathOperator{\sign}{sign}
\newcommand{\sgn}[1]{\sign\lr{#1}}

% KL operator
\DeclareMathOperator{\KL}{KL}

% kl operator
\DeclareMathOperator{\kl}{kl}

% The entropy
\let\H\relax
\DeclareMathOperator{\H}{H}

% Majority vote
\DeclareMathOperator{\MV}{MV}

% Variance
\DeclareMathOperator{\V}{Var}
\NewDocumentCommand{\Var}{o}{\V\IfValueT{#1}{\lrs{#1}}}

% VC
\DeclareMathOperator{\VC}{VC}

% VC-dimension
\newcommand{\dVC}{d_{\VC}}

% FAT ...
\DeclareMathOperator{\FAT}{FAT}
\newcommand{\dfat}{d_{\FAT}}
\newcommand{\lfat}{\ell_{\FAT}}
\newcommand{\Lfat}{L_{\FAT}}
\newcommand{\hatLfat}{\hat L_{\FAT}}

% Distance
\DeclareMathOperator{\dist}{dist}

% change all asterisks to \cdots in math mode. use \textasciiasterisk for
% asterisks in math mode.
\DeclareMathSymbol{*}{\mathbin}{symbols}{"01}

% easy integrals over -infinty .. infinity
\newcommand{\infint}{\int\limits_{-\infty}^{\infty}}


%% MY PACKAGES
\usepackage[utf8]{inputenc}
\usepackage[plainpages=false]{hyperref}
\usepackage{cleveref}
\usepackage{graphicx}
\usepackage{minted}    % code snippets
\usemintedstyle{trac}
\setminted{fontsize=\scriptsize, highlightcolor=gray, linenos, frame=lines, framesep=10pt}
% \usepackage{tikz}
% \usetikzlibrary{automata, positioning}
\usepackage[]{xcolor}
\definecolor{gray}{RGB}{235, 230, 222}
\definecolor{myRed}{RGB}{173, 20, 0}
\definecolor{myGreen}{RGB}{45, 150, 0}
\newcommand{\gray}[1]{{\setlength{\fboxsep}{0pt}\colorbox{gray}{#1}}}
\everymath{\displaystyle} % force display style for inline math. why though?


%% My COMMANDS
% big, fat, red TODO's.
\newcommand{\TODO}[1]{\textcolor{red}{TODO: #1}}

\newcommand\numberthis{\addtocounter{equation}{1}\tag{\theequation}} % equation numbering for align*

%% math related commands/operators.

% neat section and subsection separators (but have to insert manually).
\newcommand{\sectend}{\smallskip\noindent\makebox[\textwidth]{\rule{\textwidth}{0.4pt}}}
\newcommand{\Sectend}{\medskip\noindent\makebox[\textwidth]{\rule{1.1\textwidth}{1pt}}}

% easy monospace text for inline code. also,
% no need to escape underscores as with \texttt.
\newcommand{\ms}[1]{\mintinline[fontsize=\normalsize]{text}{#1}}

\DeclareMathOperator*{\argmin}{argmin}

\newcommand{\mbf}[1]{\mathbf{#1}}
\usepackage{geometry}

%
% \newcommand{\concat}{\ensuremath{+\!\!\!\!+\,}}
% \newcommand{\rpm}{\raisebox{.2ex}{$\scriptstyle\pm$}}
% \newcommand{\resub}[1]{\textcolor{red}{#1}}
% \newcommand{\blue}[1]{\textcolor{blue}{#1}}
% \newcommand{\ltt}{\sqsubseteq}
% \newcommand{\twodots}{\mathinner {\ldotp \ldotp}}
%
% \title{\large MIA -- Hand-in 3}
% \author{{\footnotesize Anders Lietzen Holst (wlc376)}\\ {\footnotesize \today}}
%
% \date{}

% % (L3C) Describe regularization.

% % (L3E) Implement a non-rigid registration in python, including a
% %         transformation model and a similarity measure.
%
% % (L3B) Describe the role of similarity measure.
% % (L3A) Describe what the registration problem is.
% % (L3D) Knowledge of the different types of transformations.


\iffalse{
  >> disposition

  section 1:
  * describe registration -- geometrically transform an image into (something
      geometrically resembling) another. useful tool to be able to compare
      different images of the same thing (or the same thing in different
      bodies). For example, if two images are created using the same modality
      (x-ray, MRI, etc) and are aligned wrt. the subject, determining the
      correlation may not be a problem at all, but sometimes we may eg. want to
      compare an x-ray to an MRI, or to compare an image of one body to the
      image of another, perhaps taken with no knowledge of the first image.

  * when is it used and why
  * supervised algorithm; needs training data


  section 2:
  steps of the algorithm -- in particular, describe similarity:

    pixelwise similarities (eg. MSE) and (normalized) cross correlation match
    pixel intensities, but we often want to match \textit{patterns}: hence
    mutual image information, based on joint entropy.

    The mutual information $MI(I, J)$ measures the amount of information one can
    obtain about an image $I$ from the image $J$, and vice versa, and is given
    by the double sum over image pixels:


\begin{align*}
  MI(\mathcal I, \mathcal J) &= \sum_{i \in \mathcal I} \sum_{j \in \mathcal J}
  p(i, j) \log{3}\TODO{this}
\end{align*}

  and an image registration is finished when the mutual information is
  \textit{maximized}.


  section 3:
  * different types of transformations: rigid, non-rigid

  * regularization?

}\fi



\documentclass{llncs}

%% FROM THE PROVIDED math_symbols.tex
\usepackage{amsthm}
\usepackage{amssymb}
\usepackage{amsmath}
\usepackage{mathtools}
\usepackage{xifthen}
\usepackage{xparse}
\usepackage{dsfont}


% Left-right bracket
\newcommand{\lr}[1]{\left (#1\right)}

% Left-right square bracket
\newcommand{\lrs}[1]{\left [#1 \right]}

% Left-right curly bracket
\newcommand{\lrc}[1]{\left \{#1\right\}}

% Left-right absolute value
\newcommand{\lra}[1]{\left |#1\right|}

% Left-right upper value
\newcommand{\lru}[1]{\left \lceil#1\right\rceil}

% Scalar product
\newcommand{\vp}[2]{\left \langle #1 , #2 \right \rangle}

% The real numbers
\newcommand{\R}{\mathbb R}

% The natural numbers
\newcommand{\N}{\mathbb N}

% Expectation symbol with an optional argument
\NewDocumentCommand{\E}{o}{\mathbb E\IfValueT{#1}{\lrs{#1}}}

% Indicator function with an optional argument
\NewDocumentCommand{\1}{o}{\mathds 1{\IfValueT{#1}{\lr{#1}}}}

% Probability function
\let\P\undefined
\NewDocumentCommand{\P}{o}{\mathbb P{\IfValueT{#1}{\lr{#1}}}}

% A hypothesis space
\newcommand{\HH}{\mathcal H}

% A sample space
\newcommand{\XX}{\mathcal{X}}

% A label space
\newcommand{\YY}{\mathcal{Y}}

% A nicer emptyset symbol
\let\emptyset\varnothing

% Sign operator
\DeclareMathOperator{\sign}{sign}
\newcommand{\sgn}[1]{\sign\lr{#1}}

% KL operator
\DeclareMathOperator{\KL}{KL}

% kl operator
\DeclareMathOperator{\kl}{kl}

% The entropy
\let\H\relax
\DeclareMathOperator{\H}{H}

% Majority vote
\DeclareMathOperator{\MV}{MV}

% Variance
\DeclareMathOperator{\V}{Var}
\NewDocumentCommand{\Var}{o}{\V\IfValueT{#1}{\lrs{#1}}}

% VC
\DeclareMathOperator{\VC}{VC}

% VC-dimension
\newcommand{\dVC}{d_{\VC}}

% FAT ...
\DeclareMathOperator{\FAT}{FAT}
\newcommand{\dfat}{d_{\FAT}}
\newcommand{\lfat}{\ell_{\FAT}}
\newcommand{\Lfat}{L_{\FAT}}
\newcommand{\hatLfat}{\hat L_{\FAT}}

% Distance
\DeclareMathOperator{\dist}{dist}

% change all asterisks to \cdots in math mode. use \textasciiasterisk for
% asterisks in math mode.
\DeclareMathSymbol{*}{\mathbin}{symbols}{"01}

% easy integrals over -infinty .. infinity
\newcommand{\infint}{\int\limits_{-\infty}^{\infty}}


%% MY PACKAGES
\usepackage[utf8]{inputenc}
\usepackage[plainpages=false]{hyperref}
\usepackage{cleveref}
\usepackage{graphicx}
\usepackage{minted}    % code snippets
\usemintedstyle{trac}
\setminted{fontsize=\scriptsize, highlightcolor=gray, linenos, frame=lines, framesep=10pt}
% \usepackage{tikz}
% \usetikzlibrary{automata, positioning}
\usepackage[]{xcolor}
\definecolor{gray}{RGB}{235, 230, 222}
\definecolor{myRed}{RGB}{173, 20, 0}
\definecolor{myGreen}{RGB}{45, 150, 0}
\newcommand{\gray}[1]{{\setlength{\fboxsep}{0pt}\colorbox{gray}{#1}}}
\everymath{\displaystyle} % force display style for inline math. why though?


%% My COMMANDS
% big, fat, red TODO's.
\newcommand{\TODO}[1]{\textcolor{red}{TODO: #1}}

\newcommand\numberthis{\addtocounter{equation}{1}\tag{\theequation}} % equation numbering for align*

%% math related commands/operators.

% neat section and subsection separators (but have to insert manually).
\newcommand{\sectend}{\smallskip\noindent\makebox[\textwidth]{\rule{\textwidth}{0.4pt}}}
\newcommand{\Sectend}{\medskip\noindent\makebox[\textwidth]{\rule{1.1\textwidth}{1pt}}}

% easy monospace text for inline code. also,
% no need to escape underscores as with \texttt.
\newcommand{\ms}[1]{\mintinline[fontsize=\normalsize]{text}{#1}}

\DeclareMathOperator*{\argmin}{argmin}

\newcommand{\mbf}[1]{\mathbf{#1}}
\usepackage{geometry}


\title{Mia Assignment 3}
\author{Anders Holst (wlc376)}
\institute{}

\begin{document}

\pagenumbering{arabic}
\maketitle

\section{Make Your Own (10 points)}

\begin{enumerate}

  \item \textit{How do you define $\mathcal X$?}

    For each student I would collect current GPA, class attendance, and
    average hand-in page count. Assuming the Danish 7 point grading scale, that
    class attendance is described by a number between 0 and 1, and that page
    count is a natural number, we have $\mathcal X = [-3, 12] \times [0, 1]
    \times \mathbb N$.


  % \item for each prior course taken by the particular student, I would collect
  %   data on the student's lecture/class attendance, assignment point grading
  %
  %   \TODO{what?}. Thus the sample space would be \TODO{$\mathcal X = ...$}.

  \item \textit{How do you define $\mathcal Y$?}

    The label space $\mathcal Y$ would be the set of possible grades --
    assuming the Danish 7 point grading scale, we would have $\mathcal Y = \{-3,
    00, 02, 4, 7, 10, 12\}$.

  \item \textit{How do you define $\ell(y', y)$?}

    I would want incorrect predictions to be punished harder the greater the
    difference between $y'$ and $y$ -- however, to better reduce the number of
    incorrect predictions in the extremes, I would use a square loss function:
    $\ell(y', y) = (y' - y)^2$.

  \item \textit{How do you define $d(x, x')$?}


    Since the three dimensions in $\mathcal X$ have very different ranges, I
    can't use a simple Euclidean distance for $d$, since for one this would weigh
    average page count much higher than class attendance percentage. Instead, in
    order to weigh the differences in all three dimensions equally, I use
    a symmetric relative difference function, and $d$ is defined as:

    $$d(x, x') = \sqrt{\sum_{\substack{i \in \{GPA,\\ attendance,\\ pages\}}}
    \left(\frac{|x_i - x'_i|}{x_i + x'_i}\right)^2}$$

  \item \textit{How would you evaluate your algorithm in terms of $\ell(y', y)$}?

    I am not \textit{entirely} sure that I interpret the question correctly.
    However, I would probably evaluate model accuracy using MSE such that bad
    predictions weigh heavier. Since $\ell$ is a square loss function, the MSE
    is:

    $$
    \frac{1}{n}\sum_{i = 0}^n \ell(y'_i, y_i)
    $$


  \item \textit{Do you expect any issues after deployment?}

    The biggest problem I can imagine is the issue of new students for whom
    there exists no prior data. To alleviate this, I would generate random data
    based on other students (with added random noise, of course).

\end{enumerate}

\sectend

\section{Graph-based Image Segmentation}

If we interpret an image as a weighted graph, in which vertices and edges
represent pixels; where vertices are connected if the underlyig pixels are
neighbours in the image; and edge weights represent differences in intensities
between these pixels, then we can employ more sophisticated techniques than the
histogram-based segmentation, and since 3D images are simply graphs with higher
connectivity, the methods generalize nicely.

\subsection{Min-cut Segmentation}

One such graph-based method for segmentation is the \textit{min-cut} method. The
method relies on the property of flot networks\footnote{Note that a flow
network can be represented with a weighted (di-)graph if
the capacity function is embedded into graph edges as weights.}, that the
maximum flow through a network is equal to the total weight of edges in a
minimum cut separating the source and sink terminals\footnote{CLRS\cite{clrs}
has a nice proof of this property (pp. 723-724).}.

This is also called the max-flow/min-cut theorem, and we can exploit it for
segmentation purposes if we define the graph edge weights (ie. the capacity
function) in such a way that similar pixels share a \textit{high} capacity edge,
and edges between contrasting pixels have \textit{low} capacity.

As with threshold segmentation some user input is at this point required, in the
form of seed pixel coordinate(s) denoting source(s) and sink(s) for a particular
object. By choosing seeds inside and outside of a desired segment, respectively,
the algorithm will analyze the flows running from the object and out and produce
a min-cut along the edge of the object -- flooding can then be
used to label pixels inside the border denoted by the min-cut. 

\medskip

Ideally, that is -- if seeds are well-chosen, this will be the case. As Grady
notes in \cite{grady}, a successful segmentation is practically guaranteed given
enough user inpupt -- however, a number of different problems can arise from
incorrect seeding. The most promiment of these is the ``small cut problem'', a
consequence of under-seeding, in which the algorithm will simply select a cut
involving a very small number of edges close to a seed vertex, since cutting a
small number of high-intensity edges can in many cases be cheaper than cutting a
large number of low-intensity edges. This is rectifiable but requires
additional, or better placed, seeding \cite{grady}.


% \TODO{choose 0 or 1 of these out-commented paragraphs!}
% Another disadvantage of the method is the fact that only a single, contiguous
% object can be segmented at a time, meaning disjoint objects which we would
% identify as coherent (eg. a pair of lungs or parallel arteries) would
% have to be segmented separately.

% Another disadvantage is that since there can exist many cuts with capacities near
% the min-cut, and even multiple min-cuts, the amount and placement of seeds, as
% well as small changes in the image (eg. noise) can result in entirely different
% results from the algorithm.

\subsection{Random Walker Segmentation}

As a remedy to the disadvantages of min-cut segmentation, Leo Grady formulated
Random Walker Segmentation (henceforth, RW) as an alternative graph-based
algorithm in \cite{grady}. Rather than using seeds to partition the graph, RW
computes the probability than any given pixel be associated to any given seed,
for each seed in the image, and labels each pixel according to the seed with
which it is most likely to be associated.

These probabilities are based on the same edge capacities as before, and do not
involve (pseudo-)randomness, but it is beneficial to model the problem using
probabilities. However in some simplified terms the highest probability
terminating seed for a given random walker can be thought of as the seed
for which the mean edge capacities on the path from pixel to seed are maximized.


\medskip

Below figure \cref{fig:RW-seg} shows RW segmentation of the same example image
as before, for two different levels of seeding and two different
parameterizations of the RW motion probability function.


\begin{figure}%[H]
  \includegraphics[width=0.8\textwidth]{figures/RW_segmentation.png}
  % \vspace{-0.8cm}
  \caption{\small \textit{Comparison of RW for different amounts of
  seeding and different beta (penalization coefficient for the RW motion
  function). The sparse seeding is a random sample
  of the full seed pool. Note how segment 4 (yellow) is split in two disjoint
  objects.}}
  \label{fig:RW-seg}
\end{figure}

Obviously noise in the input image makes the segmentation less precise, but one
advantage of RW (and graph-cut, too) is that with good seeding, the output
segmentation is already de-noised as opposed to histogram segmentation.

As the plot shows RW is also very sensitive to proper seeding: The disjoint
segment 4 (yellow) is not properly segmented when its annex is not seeded
(bottom row). Even if the solution is simply more seeding, this is still a
disadvantage. Perhaps the biggest disadvantage of RW, however, is a huge
sensitivity to proper parameterization of the RW motion probability function
(ie. the probabilities attached to each edge in the graph), as is also evident
in the plot, where the segmentation is, to put it mildly, hit and miss depending
on beta (see caption).



\sectend

\iffalse{
Section 2 (L2C, L2D) -- graph cut and random walker algorithms for segmentation -- 
* graph cut: finding minimum cut via max flow.
  the max-flow/min-cut theorem states that the maximum flow through a graph
  is equal to the minimum sum cut.

  in graph cut segmentation, we denote capacity of an edge between two pixels as
  the difference in pixel intensities; as an example, the edge between two
  (neighbouring) black pixels would have higher capacity than would the edge
  between a black and a white pixel.

  By choosing a source pixel (or \textit{seed}) inside of a given object, aswell
  as a target seed \textit{outside} of that object (ie. in the background of the
  image), the border of the object will, if the seeds are chosen correctly and
  the algorithm is successful, be the min-cut of the image graph.

  the problem: works only if we have one object and a background.

* random walker

  can segment multiple objects and several backgrounds.
  ??

  % we call it a \textit{random} walker, but a good way to think about the
  % algorithm is to call it a \textit{minimizing} walker, and, in fact, there is
  % not one but \textit{many} such ``walkers''

  show example of random walker segmentation

* both random walker and graph cut assume that objects in the image form in
  contiguous clusters -- this implies that to segment multiple related objects,
  each need to be properly and individually seeded, whereas with histogram based
  segmentation we can simply choose a good threshold and capture all desired
  objects -- on the other hand, if used properly then graph based segmentation
  can more easily filter out unwanted noise without the need for CCD.

}\fi



\begin{thebibliography}{8}
  \bibitem{wu}
    Wu, Zufeng; Lan, Tian; Wang, Jiang; Ding, Yi; Qin, Zhiguang. \textit{Medical
    Image Registration Using B-Spline Transform}. School of Information and
    Software Engineering, University of Electronic Science and Technology of
    China.

    \url{https://ijssst.info/Vol-17/No-48/paper1.pdf}.

\end{thebibliography}

\end{document}
