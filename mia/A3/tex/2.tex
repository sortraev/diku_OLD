\newcommand{\I}{\mathcal I}
\newcommand{\J}{\mathcal J}

\section{Algorithm(s) for image registration}

So far we simply know that image registration computes the transformation which
turns a moving image $\I$ into a new image $\I'$ most resembling some fixed
image $\J$. In less abstract terms, we can view registration as an optimization
problem: We repeatedly morph and transform the moving image until we obtain the
$\I'$ which most resembles $\J$, at each step evaluating the similarity between
$\I'$ and $\J$.

\subsection{Basic algorithm}

The general algorithm for image registration consists of three steps:

\begin{enumerate}
  \item[1.] \textbf{initialization}: the transformation function we are looking
    for is initialized, and, optionally, a domain expert or possibly a computer
    may determine landmarks in the image, whose preservation the algorithm
    should prioritize during optimization of the transformation function
    (this is one form of regularization, which will be discussed in
    \cref{sec:regularization}).

  \item[2.] \textbf{image similarity quantification}: the current transformation
    function is applied to the moving image, and the degree of similarity
    between the transformed moving and the fixed image is measured as per a
    chosen similarity metric (see \cref{sec:similarity}).
    This can be viewed as training error.

  \item[3.] \textbf{transformation estimation}: the transformation which best maps
    the coordinate system of the moving image to the fixed image is estimated
    via optimization, wrt. the similarities computed earlier. If similarity
    is optimal, return the estimated transformation function; else go back to
    step 2.
\end{enumerate}

When the estimated transformation function is returned, it can be applied to the
moving image, but the function itself may also reveal interesting facts about
the differences between the images. In the following section, we shall discuss
similarity metrics, the optimization step, and regularization.

\subsubsection{Similarity metrics}~\smallskip
\label{sec:similarity}

In the second step of the algorithm, the current transformation is applied to
the moving image, after which the resulting image is compared to the fixed
image. This is done using a pre-chosen \textit{similarity metric}, a
function over two images returning a similarity score, which can be viewed as
negative training error and thus be used in optimization. Three commonly used
categories of similarity metrics are the intensity-based, histogram-based, and
feature-based similarities.

\paragraph{Intensity-based similarity}~

The simplest forms of image similarity are those based on actual pixel values.
The intensity-based similarity between two images $\I$ and $\J$ is proportional
to the differences in pixel intensities between the two images, whether it be
per pixel or region based. The most commonly used examples of intensity based
measures are  MSE (mean square error) and cross-correlation, which compute
per-pixel intensity similarities.

These measures have the immediate advantage that they are easy to reason about,
and some of them can be computationally cheap compared to eg. MI, but they all
require images to be of similar intensity ranges, and do not consider structural
similarities. As an example, consider two images $\I$ and $\J = -\I$. One is
the negative of the other, so they must be structurally identical, but the
intensity based similarity will be very low since the images are literally
opposites -- for these reasons, these methods can be very effective for images
which are known to be of similar intensities, eg. images created using the same
modality, but can fail tremendously in many cases.

\paragraph{Histogram-based similarty (MI)}~

If pixel value ranges in the two images $\I$ and $\J$ are not comparable, then
we might use histogram-based similarities. The mutual information $MI(\I, \J)$
measures the amount of information one can obtain about image $\I$ from image
$\J$, and vice versa, and is given by the double sum over image pixels:

\begin{align*}
  MI(\I, \J) &= \sum_{i\, \in \,\I} \sum_{j\, \in
  \,\J} p(i, j) \log\left\{\frac{p(i, j)}{p(i)p(j)}\right\},
\end{align*}

where $p(i, j)$ is the joint probability, and $p(i)$ and $p(j)$ are the
individual image probabilities. In the discrete case, $p(i, j)$ is derived from
the joint 2D histogram, while $p(i)$ and $p(j)$ are derived from the marginal
histograms (in turn derived from the joint histogram).

% Since $MI(\I, \J)$ is the amount of information shared between $\I$ and $\J$, an
% image registration is optimal when the mutual information is \textit{maximized}.

One of the primary motivations for MI is the usefulness in multi-modal
registration, as well as for use between uni-modal images from different cameras
or subjects, and since it is agnostic of individual intensities, for the
previous example of $\J = -\I$, mutual information would be maximal.

However, note that even if we don't require images to \textit{look} similar in
order to use histogram-based metrics, we still require some structural
relationship. This would eg. imply that a CT scan and an MRI of the same brain
might not be comparable, since the CT scan would mostly only capture the outline
of the skull, whereas the MRI would capture a lot of shapes inside.

\paragraph{Feature-based similarity}~

The third category of similarity metrics is useful whenever we need to compare
two images with only some contents in common, as in the brain example given
above. Here we would prefer the algorithm to transform wrt. a set of known
structures.

Compared to the other two categories, feature-based is the most involved, since
it requires user input in the form of landmark coordinates (reminiscent of
graph-based segmentation algorithms), but can be computationally cheaper since
the problem can now be expressed as a set of equations.

If feature-based similarity is used, then the landmark selection is part of step
1 (initialization) of the algorithm.

\subsubsection{Transformation fitting (optimization)}~

If the current transformation was found suboptimal, the optimization step will
update the parameters of the transformation wrt. the derivative of the
similarity metric. This optimization is usually performed using gradient descent
(GD) or a similar iterative optimization algorithm, however GD has the pitfall
of finding local minima, so it may be a good idea to use a more exhaustive
search for the first initialization of the transform.

The result of the fitting is a new transformation field. For rigid and affine
transformations, fitting the new transformation may be as simple as updating the
weight in a rotation/translation matrix, but for non-rigid transformations the
process is a little more tedious. Since a non-rigid transformation can contain
an arbitrary mapping with local deformations of the coordinate system, the lines
of the coordinate system likely do not stay straight and parallel wrt. each
other -- however, we at least prefer to preserve continuity of the coordinate
system. In general this may be difficult depending on the transformation, but we
can create an approximation with continuous lines using eg. B-spline
interpolation or Bézier curves, which, in fact, is often computationally
efficient\cite{wu}.


\subsection{Atlas-based registration}

One possible variant of the algorithm is \textit{atlas-based image
registration}, which uses a training set (or \textit{atlas}) of images to
compute the average of transformations of moving images wrt. a fixed image,
which can then be used to register new images.
\vspace{-0.8cm}

\begin{figure}[H]
  \includegraphics[width=\textwidth]{figures/atlas_registration_5_trainimgs.png}
  \caption{{\footnotesize Example atlas-based registration using 5 training
  images.}}
  \label{fig:atlasreg1}
  % \includegraphics[width=\textwidth]{figures/atlas_registration_40_trainimgs.png}
  % \caption{{\footnotesize Example atlas-based registration using 40 training
  % images.}}
  % \label{fig:atlasreg2}
\end{figure}
\vspace{-0.5cm}

The immediate benefit of this method is that it allows for more nuanced
registrations, in that the result is not fitted especially to certain features
which may be unique to a single image.

The atlas-based does have a big disadvantage, though, as it computationally
expensive, requiring one complete registration for each training image, so there
is a very big trade-off between speed and efficiency related to the size of the
training set, and the results are not necessarily impressive, since the
averaging part of the algorithm means small or outlying features are hard to
capture.

\subsubsection{Regularization}
\label{sec:regularization}

In machine learning, a common problem in predictions and model estimation is
called \textit{overfitting}, and this happens when a model becomes overly
sensitive to outliers and noise in a particular training data, to the point
where the model begins to expect similar noise in new data, ultimately
influencing the accuracy of new predictions negatively.

While cross-validation can be used as an effective data-oriented solution,
\textit{regularization} is another commonly used solution method, which involves
restricting the learning process to prevent learning overly complex models.

\smallskip

The problem of overfitting exists also in image registration, where we might
sometimes notice the registration eagerly fitting minute details of the moving
image to the fixed image, where we might rather wish for the registration to
transform the larger structures of the image and leave the finer details be.
This problem is especially prevalent in multimodal registration, where we often
have two images containing shapes sharing the same basic structural outline, but
with different insides, as in the PET/MRI example given in below
\cref{fig:overfitting}:

\begin{figure}[H]
  \includegraphics[width=\textwidth]{figures/petmri_overfitting.png}
  \caption{{\footnotesize Example of overfitting in multi-modal registration.}}
  \label{fig:overfitting}
\end{figure}
\vspace{-0.5cm}

Inspecting the deformation field, we see some rather intense deformations in the
area inside the skull, where we might have expected to see deformations along
the boundary of the skull and lines closer to being straight inside (this
example is, of course, for illustrative purposes -- note that in this novel
example, we might have achieved the desired result with simply translation and
scaling).

While in such cases we are perhaps better defining a number of landmark points
and using feature-based registration, we might also employ regularization to
penalize the high number of irregular deformations inside the skull.

\sectend
