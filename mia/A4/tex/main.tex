\iffalse{
  >> LEARNING GOALS

  % (L4A) Explain the theoretical basics behind CNNs
  % (L4B) Implement and test a specific CNN for image segmentation
  % (L4C) Define what cross-validation is
  % (L4D) Assess classifier performance with different metrics
  % (L4E) Design your own algorithm (e.g. CNN) and evaluate its performance on the skin lesion classification data.

  >> DISPOSITION

  >> 1: Convolutional Neural Networks in MIA (L4A)
  * what is a neural network? an ML model comprised of a network of connected
  \emph{neurons}, organized in so-called \emph{layers}, each with its own
  associated weights and biases -- much like a set of equations. Each neuron
  outputs a value that is fed to an activation function which indicates to the
  next layer whether or not the output of this particular neuron is important.

  section 1.1 convolutional NN
  * what is a convolutional neural network? One big problem arises when we
  attempt to use ANN's with image data: Input image data can vary greatly, and
  ideally we want to build a model that is robust regardless of the placement
  and orientation of objects in the image. This is called translation
  invariance. A CNN comes close by feeding the input image through consecutive
  convolution layers, each layer producing multiple feature maps, each map
  capturing different features of the image -- for example, in one layer we
  might have a set of neurons in a given layer which solely detect diagonal
  rounded edges, and another set which boosts and captures the contrast between
  foreground and background objects -- and then feeding the result of this
  through downsampling (or pooling) layers, which extract defining features and
  compresses the input into an exponentially smaller map of information.

  This downsampling greatly reduces computational overhead as opposed to a
  regular ANN, however, a natural consequence, of course, is that we must trade
  off translation invariance for -equivariance (ie. the output \emph{is}
  affected by translation of the input, but the translation is similar and thus
  predictable).


  section 2: U-Net and implementation (L4B + L4C?)

  describe the U-Net model for image segmentation.
  contracting and expanding path.
  idea is to extract features, then upsample the extracted features to something
  resembling the input image, but just the important features.
  example of such a network (graph, or just textual description????)

  show example segmentation (and mention why it fails for that one stupid mask)

  subsection 2.1 : training and cross-validation (L4C)

  subsection 2.2: evaluation (L4D)
  loss/accuracy graph. mention that loss seems to still be decreasing, which
  might indicate a necessity for further fitting, however, validation loss seems
  to have plateaued.
  ROC graph and AUC. explain why AUC = 0.97 is pretty good.


  section 3: skin lesion image binary classification (L4E)

  very short description of dataset: benign/malignant. We want a binary
  classification of the data cases.

  subsection 3.1:
  describe the model: same as prior U-Net model, but with a dense layer substituted
  for the entire expanding path.

  subsection 3.2: Regularization using data augmentation
  The supplied training data is heavily imbalanced with 80.8\% benign cases.
  With the data as is, the model is very likely to simply always predict
  ''benign``, for a training accuracy of $\sim 80.8\%$. In cases such as this
  one, data augmentation is a suitable regularization method, can be used to
  straighten out the imbalance. Hence, I augment the dataset with 522 malignant
  cases for a total of 695, to better match the 727 benign cases, using random
  rotation, translation, noise/blur, and brightness/contrast.

  subsection 3.x: classifier assessment

  subsubsection: assessing training process

  show loss/accuracy of training process. explain that this can be used to
  assess whether model tweaking or more fitting is needed (decreasing loss can
  indicate need for more fitting; small loss but unexpected accuracy can
  indicate bad local minimum, which can further indicate overly complex model or
  not enough training data).

  subsubsection: assessing results

  After training: run on test data.
  receiver operating characteristic curve
}\fi


\documentclass{llncs}

%% various math packages
\usepackage{amsthm}
\usepackage{amssymb}
\usepackage{amsmath}
\usepackage{amsfonts}
\usepackage{mathtools}
\usepackage{xifthen}
\usepackage{xparse}
\usepackage{dsfont}
\everymath{\displaystyle} % force display style for inline math
\newcommand\numberthis{\addtocounter{equation}{1}\tag{\theequation}} % equation numbering for align*

%% misc packages.
\usepackage{xfrac}
\usepackage{parskip}
\usepackage[utf8]{inputenc}
\usepackage{hyperref}
\usepackage{cleveref}
\usepackage{graphicx}
\usepackage{float}
\usepackage{subcaption}
\usepackage{enumitem}
\usepackage[outputdir=out]{minted}
\usemintedstyle{trac}
\setminted{%frame=lines,
    linenos=true,
    fontsize=\footnotesize
}
\usepackage[table,xcdraw]{xcolor}


%% math commands.
\DeclareMathOperator*{\argmin}{argmin} % argmin.
\newcommand{\bbar}[1]{\overline{#1}}   % a wider bar than \bar.
\newcommand{\eps}{\varepsilon}         % prettier epsilon.
\newcommand{\mbf}[1]{\mathbf{#1}}      % shorthand.
\newcommand{\R}{\mathbb R}             % set of real numbers.
\newcommand{\N}{\mathbb N}             % set of natural numbers.
\newcommand{\Z}{\mathbb Z}             % set of integers.
\let\emptyset\varnothing               % better emptyset symbol. 
\newcommand{\pr}[1]{\text{Pr}\left[#1\right]} % pretty probabilities.
\DeclareMathSymbol{*}{\mathbin}{symbols}{"01} % map asterisks to \cdot. use
                                              % \ast for asterisk in math mode. 

% bold and red TODO's.
\newcommand{\TODO}[1]{\textcolor{red}{TODO: #1}}



\title{MIA Assignment 4}
\author{Anders Holst (wlc376)}
\institute{}

\begin{document}

\pagenumbering{arabic}
\maketitle

\section{Make Your Own (10 points)}

\begin{enumerate}

  \item \textit{How do you define $\mathcal X$?}

    For each student I would collect current GPA, class attendance, and
    average hand-in page count. Assuming the Danish 7 point grading scale, that
    class attendance is described by a number between 0 and 1, and that page
    count is a natural number, we have $\mathcal X = [-3, 12] \times [0, 1]
    \times \mathbb N$.


  % \item for each prior course taken by the particular student, I would collect
  %   data on the student's lecture/class attendance, assignment point grading
  %
  %   \TODO{what?}. Thus the sample space would be \TODO{$\mathcal X = ...$}.

  \item \textit{How do you define $\mathcal Y$?}

    The label space $\mathcal Y$ would be the set of possible grades --
    assuming the Danish 7 point grading scale, we would have $\mathcal Y = \{-3,
    00, 02, 4, 7, 10, 12\}$.

  \item \textit{How do you define $\ell(y', y)$?}

    I would want incorrect predictions to be punished harder the greater the
    difference between $y'$ and $y$ -- however, to better reduce the number of
    incorrect predictions in the extremes, I would use a square loss function:
    $\ell(y', y) = (y' - y)^2$.

  \item \textit{How do you define $d(x, x')$?}


    Since the three dimensions in $\mathcal X$ have very different ranges, I
    can't use a simple Euclidean distance for $d$, since for one this would weigh
    average page count much higher than class attendance percentage. Instead, in
    order to weigh the differences in all three dimensions equally, I use
    a symmetric relative difference function, and $d$ is defined as:

    $$d(x, x') = \sqrt{\sum_{\substack{i \in \{GPA,\\ attendance,\\ pages\}}}
    \left(\frac{|x_i - x'_i|}{x_i + x'_i}\right)^2}$$

  \item \textit{How would you evaluate your algorithm in terms of $\ell(y', y)$}?

    I am not \textit{entirely} sure that I interpret the question correctly.
    However, I would probably evaluate model accuracy using MSE such that bad
    predictions weigh heavier. Since $\ell$ is a square loss function, the MSE
    is:

    $$
    \frac{1}{n}\sum_{i = 0}^n \ell(y'_i, y_i)
    $$


  \item \textit{Do you expect any issues after deployment?}

    The biggest problem I can imagine is the issue of new students for whom
    there exists no prior data. To alleviate this, I would generate random data
    based on other students (with added random noise, of course).

\end{enumerate}

\sectend

\newpage
\section{Digits Classification with K Nearest Neighbors (45 points)}

\subsection{Task \#1}

\paragraph{How does $n$ affect fluctuation in validation error?}~\smallskip

\begin{figure}[H]
  \includegraphics[width=\textwidth]{figures/fig2_2.png}
\caption{\footnotesize \it Fluctuation in validation error for varying i as a function of k, for
  each n.}
\label{fig:2-2}
\end{figure}

While $i$ determines which subset of the data is used for validation, $n$
determines the \textit{size} of each of these validation subsets.

There is a very clear trend in the relationship the value of $n$ and fluctuation
in validation error across all values of $k$, but it is best illustrated by
examining the right end of the plot.

For high values of $k$ (roughly 30 and greater), there is an undeniable
indication that the higher the value of $n$, the smaller the fluctuation in
validation error across values of $i$. This is because as $n$ increases, the
differences between two given validation subsets are smoothened out -- in other
words, the smaller the $n$, the more the result is influenced by outliers in the
validation set.


\newpage
\paragraph{How does $K$ affect prediction accuracy?}~\smallskip

\begin{figure}[H]
  \includegraphics[width=\textwidth]{figures/fig2_1.png}
  \caption{\footnotesize \it Mean zero/one prediction error of KNN for varous i, n, k.}
\label{fig:2-1}
\end{figure}

Interestingly, the overall best $K$ seems to be $K^* = 1$, with almost zero
validation error for most values of $i$ across all $n$.

For all four values of $n$, prediction error seems to increase with $K$. This is
likely because our training set is only 100 digits, and as $K$ increases towards
half of the training set, the model moves closer towards simply estimating the
sample mean.


\subsection{Task \#2}

\begin{figure}[H]
  \includegraphics[width=\textwidth]{figures/fig2_3.png}
\caption{\footnotesize \it Mean zero/one prediction error of KNN for various degrees
         of data corruption; for n = 80 and varying i, k.}
\label{fig:2-3}
\end{figure}

\paragraph{How does corruption magnitude influence prediction accuracy and the optimal
value of K?}~\smallskip

Not surprising, there is a correlation between corruption magnitude and
prediction error -- for the uncorrupted set, the prediction error lies roughly
in the range $[0, 0.16]$, while prediction errors lie roughly in the ranges
$[0.02, 0.18]$, $[0.1, 0.3]$, and $[0.2, 0.4]$ for the lightly, moderately, and
heavily corrupted sets, respectively. For the heavily corrupted set, prediction
error reaches as high as roughly 40\% for $i = 5$.

As far as the optimal $K$, based on the plots we may still be inclined to simply
choosing small values of $K^*$ for the three corrupted sets, but the indication
is not as strong based on the plots alone, since the trend in the curves seem
flatter than that of the uncorrupted set -- this perhaps applies less so for the
lightly corrupted set.

Most interestingly, however, is perhaps the fluctuation in validation error
across values of $i$. This has not been examined explicitly, but for the heavily
corrupted set it appears as though the five curves stick closer together than
for the uncorrupted set.

\sectend

\newpage
\section{A1.3) Postfix expressions using queues}

\subsection{A.1.3.a)}

\begin{itemize}
    \item \emph{Describe pseudocode for translating fully parenthesized infix
      expressions to queue postfix expressions.}
\end{itemize}

I take the hint given in the assignment, of processing infix expressions one at
a time, outputting numbers and operators, while re-queueing subexpressions.

\begin{minted}{python}
infix_to_queue_expression(exp):

  out  = new empty stack # will hold the output queue expression.

  exps = new empty queue # will hold unprocessed (sub-)expressions,
  exps.enqueue(exp)      # starting with the entire infix expression.

  while exps is non-empty:
    case exps.dequeue() of
      (number x) ->      # if next in queue is a number, simply push to out.
        out.push(x)

      # if next in queue is a binop, push its operator
      # to out and enqueue its two subexpression operands.
      (subexp_1, operator, subexp_2) ->
        out.push(operator)

        exps.enqueue(subexp_2)
        exps.enqueue(subexp_1)

  return out
\end{minted}


\newpage

\subsection{A.1.3.b)}
\begin{itemize}
  \item \emph{Show the queue and (partial) output during translation of \ms{((1
    + 3) - (5 * 7))} to \ms{1 3 5 7 + * -}.}
\end{itemize}
\begin{minted}[linenos=false]{text}
step | exps queue       | dequeuing     | output  | comment
0    | [((1+3)-(5*7))]  |               |         | init exps queue

1    | []               | ((1+3)-(5*7)) |         | dequeue expression
2    | [(1+3), (5*7)]   |               | -       | output op; queue subexps
3    | [(1+3)]          | (5*7)         | -       | dequeue expression
4    | [5, 7, (1+3)]    |               | *-      | output op; queue subexps
5    | [5, 7]           | (1+3)         | *-      | dequeue expression
6    | [1, 3, 5, 7]     |               | +*-     | output op; queue subexps

7    | [1, 3, 5]        | 7             | +*-     | dequeue expression
8    | [1, 3, 5]        |               | 7+*-    | output value
9    | [1, 3]           | 5             | 7+*-    | dequeue expression
10   | [1, 3]           |               | 57+*-   | output value
11   | [1]              | 3             | 57+*-   | dequeue expression
12   | [1]              |               | 357+*-  | output value
13   | []               | 1             | 357+*-  | dequeue expression
14   | []               |               | 1357+*- | output value

15   | []               |               | 1357+*- | queue empty; terminate
\end{minted}



\newpage

% \begin{thebibliography}{8}

  % \bibitem{ronneberger}
  %   Ronneberger, Olaf; Fischer, Philipp; Brox, Thomas. \textit{U-Net:
  %   Convolutional Networks for Biomedical Image Segmentation}. arXiv:1505.04597.
  %   \url{https://arxiv.org/abs/1505.04597}.
  % \bibitem{wu}
  %   Wu, Zufeng; Lan, Tian; Wang, Jiang; Ding, Yi; Qin, Zhiguang. \textit{Medical
  %   Image Registration Using B-Spline Transform}. School of Information and
  %   Software Engineering, University of Electronic Science and Technology of
  %   China.
  %
  %   \url{https://ijssst.info/Vol-17/No-48/paper1.pdf}.

% \end{thebibliography}

\end{document}
