\section{Morphological operations and postprocessing}
\label{sec:morphs}

In \cref{fig:hist-seg} in \cref{sec:hist-seg} we saw how the histogram-based
segmentation of a noisy image produced an almost equally noisy result. In images
we must deal with two types of noise: Noise overlaying objects, and noise
in background regions of the image, and each can come in the form of ``peppery''
dots, or larger areas both inside and outside of image objects.

An easy remedy for both are morphological openings and closings: A closing can
be used to remove noise in foreground objects or to fuse together image objects
which belong together by ``filling the gaps'', so to speak, while an opening can
be used to remove small objects from the foreground by simply suffocating them.
In \cref{fig:hist-seg}, the segments were de-noised by applying morphological
operations to each segment in isolation.

An opening/closing can be created using two kernel convolutions \footnote{So
long as the $n$ erosions and $m$ dilations are not interleaved, which they are
not in the traditional definition of the morphological operations.}, and it
follows that the operations generalize to higher dimensions (eg. 3D images).

However one should be wary when employing opening/closing: Erosion has the
disadvantage that small or thin objects, which are not actually noise, such as
tumors or blood vessels, may be lost if too many erosions are performed, while
important inter- and/or intra-object boundaries, such as brain tissue barriers
or, again, parallel blood vessels, may be obscured if too many dilations are
performed in sequence.

% \iffalse{
% Section 3 (L2B) -- postprocessing of segmented images
% * CCD, erosion, and dilation for post-processing of segmented images.
% * benefits of erosion/dilation: can be used to easily remove noise in image
%   objects, eg. organs, and to join together image objects which may have been
%   separated by noise in the image.
%
% * give example of image
% }\fi

\sectend
