\documentclass[a4paper,12pt]{article}

%% various math packages
\usepackage{amsthm}
\usepackage{amssymb}
\usepackage{amsmath}
\usepackage{amsfonts}
\usepackage{mathtools}
\usepackage{xifthen}
\usepackage{xparse}
\usepackage{dsfont}
\everymath{\displaystyle} % force display style for inline math
\newcommand\numberthis{\addtocounter{equation}{1}\tag{\theequation}} % equation numbering for align*

%% misc packages.
\usepackage{xfrac}
\usepackage{parskip}
\usepackage[utf8]{inputenc}
\usepackage{hyperref}
\usepackage{cleveref}
\usepackage{graphicx}
\usepackage{float}
\usepackage{subcaption}
\usepackage{enumitem}
\usepackage[outputdir=out]{minted}
\usemintedstyle{trac}
\setminted{%frame=lines,
    linenos=true,
    fontsize=\footnotesize
}
\usepackage[table,xcdraw]{xcolor}


%% math commands.
\DeclareMathOperator*{\argmin}{argmin} % argmin.
\newcommand{\bbar}[1]{\overline{#1}}   % a wider bar than \bar.
\newcommand{\eps}{\varepsilon}         % prettier epsilon.
\newcommand{\mbf}[1]{\mathbf{#1}}      % shorthand.
\newcommand{\R}{\mathbb R}             % set of real numbers.
\newcommand{\N}{\mathbb N}             % set of natural numbers.
\newcommand{\Z}{\mathbb Z}             % set of integers.
\let\emptyset\varnothing               % better emptyset symbol. 
\newcommand{\pr}[1]{\text{Pr}\left[#1\right]} % pretty probabilities.
\DeclareMathSymbol{*}{\mathbin}{symbols}{"01} % map asterisks to \cdot. use
                                              % \ast for asterisk in math mode. 

% bold and red TODO's.
\newcommand{\TODO}[1]{\textcolor{red}{TODO: #1}}



\newcommand{\concat}{\ensuremath{+\!\!\!\!+\,}}
\newcommand{\rpm}{\raisebox{.2ex}{$\scriptstyle\pm$}}
\newcommand{\resub}[1]{\textcolor{red}{#1}}
\newcommand{\blue}[1]{\textcolor{blue}{#1}}
\newcommand{\ltt}{\sqsubseteq}
\newcommand{\twodots}{\mathinner {\ldotp \ldotp}}

\title{\large MIA -- Hand-in 2}
\author{{\footnotesize Anders Lietzen Holst (wlc376)}\\ {\footnotesize \today}}

\date{}


\begin{document}


% (L2E) Explain how PCA can be used in medical image segmentation

% (L2A) Explain how and where medical image segmentation can be used
% (L2B) Explain the benefits and risks of usage of image dilation/erosion operations. Give examples.
% (L2C) Implement and test random walker for medical image segmentation (both 2D and 3D). 
% (L2D) What are the advantages of random walker/graph cut, what kind of problems they can solve.

\maketitle

\section{Make Your Own (10 points)}

\begin{enumerate}

  \item \textit{How do you define $\mathcal X$?}

    For each student I would collect current GPA, class attendance, and
    average hand-in page count. Assuming the Danish 7 point grading scale, that
    class attendance is described by a number between 0 and 1, and that page
    count is a natural number, we have $\mathcal X = [-3, 12] \times [0, 1]
    \times \mathbb N$.


  % \item for each prior course taken by the particular student, I would collect
  %   data on the student's lecture/class attendance, assignment point grading
  %
  %   \TODO{what?}. Thus the sample space would be \TODO{$\mathcal X = ...$}.

  \item \textit{How do you define $\mathcal Y$?}

    The label space $\mathcal Y$ would be the set of possible grades --
    assuming the Danish 7 point grading scale, we would have $\mathcal Y = \{-3,
    00, 02, 4, 7, 10, 12\}$.

  \item \textit{How do you define $\ell(y', y)$?}

    I would want incorrect predictions to be punished harder the greater the
    difference between $y'$ and $y$ -- however, to better reduce the number of
    incorrect predictions in the extremes, I would use a square loss function:
    $\ell(y', y) = (y' - y)^2$.

  \item \textit{How do you define $d(x, x')$?}


    Since the three dimensions in $\mathcal X$ have very different ranges, I
    can't use a simple Euclidean distance for $d$, since for one this would weigh
    average page count much higher than class attendance percentage. Instead, in
    order to weigh the differences in all three dimensions equally, I use
    a symmetric relative difference function, and $d$ is defined as:

    $$d(x, x') = \sqrt{\sum_{\substack{i \in \{GPA,\\ attendance,\\ pages\}}}
    \left(\frac{|x_i - x'_i|}{x_i + x'_i}\right)^2}$$

  \item \textit{How would you evaluate your algorithm in terms of $\ell(y', y)$}?

    I am not \textit{entirely} sure that I interpret the question correctly.
    However, I would probably evaluate model accuracy using MSE such that bad
    predictions weigh heavier. Since $\ell$ is a square loss function, the MSE
    is:

    $$
    \frac{1}{n}\sum_{i = 0}^n \ell(y'_i, y_i)
    $$


  \item \textit{Do you expect any issues after deployment?}

    The biggest problem I can imagine is the issue of new students for whom
    there exists no prior data. To alleviate this, I would generate random data
    based on other students (with added random noise, of course).

\end{enumerate}

\sectend

\newpage
\section{A1.3) Postfix expressions using queues}

\subsection{A.1.3.a)}

\begin{itemize}
    \item \emph{Describe pseudocode for translating fully parenthesized infix
      expressions to queue postfix expressions.}
\end{itemize}

I take the hint given in the assignment, of processing infix expressions one at
a time, outputting numbers and operators, while re-queueing subexpressions.

\begin{minted}{python}
infix_to_queue_expression(exp):

  out  = new empty stack # will hold the output queue expression.

  exps = new empty queue # will hold unprocessed (sub-)expressions,
  exps.enqueue(exp)      # starting with the entire infix expression.

  while exps is non-empty:
    case exps.dequeue() of
      (number x) ->      # if next in queue is a number, simply push to out.
        out.push(x)

      # if next in queue is a binop, push its operator
      # to out and enqueue its two subexpression operands.
      (subexp_1, operator, subexp_2) ->
        out.push(operator)

        exps.enqueue(subexp_2)
        exps.enqueue(subexp_1)

  return out
\end{minted}


\newpage

\subsection{A.1.3.b)}
\begin{itemize}
  \item \emph{Show the queue and (partial) output during translation of \ms{((1
    + 3) - (5 * 7))} to \ms{1 3 5 7 + * -}.}
\end{itemize}
\begin{minted}[linenos=false]{text}
step | exps queue       | dequeuing     | output  | comment
0    | [((1+3)-(5*7))]  |               |         | init exps queue

1    | []               | ((1+3)-(5*7)) |         | dequeue expression
2    | [(1+3), (5*7)]   |               | -       | output op; queue subexps
3    | [(1+3)]          | (5*7)         | -       | dequeue expression
4    | [5, 7, (1+3)]    |               | *-      | output op; queue subexps
5    | [5, 7]           | (1+3)         | *-      | dequeue expression
6    | [1, 3, 5, 7]     |               | +*-     | output op; queue subexps

7    | [1, 3, 5]        | 7             | +*-     | dequeue expression
8    | [1, 3, 5]        |               | 7+*-    | output value
9    | [1, 3]           | 5             | 7+*-    | dequeue expression
10   | [1, 3]           |               | 57+*-   | output value
11   | [1]              | 3             | 57+*-   | dequeue expression
12   | [1]              |               | 357+*-  | output value
13   | []               | 1             | 357+*-  | dequeue expression
14   | []               |               | 1357+*- | output value

15   | []               |               | 1357+*- | queue empty; terminate
\end{minted}


\newpage
\section{Digits Classification with K Nearest Neighbors (45 points)}

\subsection{Task \#1}

\paragraph{How does $n$ affect fluctuation in validation error?}~\smallskip

\begin{figure}[H]
  \includegraphics[width=\textwidth]{figures/fig2_2.png}
\caption{\footnotesize \it Fluctuation in validation error for varying i as a function of k, for
  each n.}
\label{fig:2-2}
\end{figure}

While $i$ determines which subset of the data is used for validation, $n$
determines the \textit{size} of each of these validation subsets.

There is a very clear trend in the relationship the value of $n$ and fluctuation
in validation error across all values of $k$, but it is best illustrated by
examining the right end of the plot.

For high values of $k$ (roughly 30 and greater), there is an undeniable
indication that the higher the value of $n$, the smaller the fluctuation in
validation error across values of $i$. This is because as $n$ increases, the
differences between two given validation subsets are smoothened out -- in other
words, the smaller the $n$, the more the result is influenced by outliers in the
validation set.


\newpage
\paragraph{How does $K$ affect prediction accuracy?}~\smallskip

\begin{figure}[H]
  \includegraphics[width=\textwidth]{figures/fig2_1.png}
  \caption{\footnotesize \it Mean zero/one prediction error of KNN for varous i, n, k.}
\label{fig:2-1}
\end{figure}

Interestingly, the overall best $K$ seems to be $K^* = 1$, with almost zero
validation error for most values of $i$ across all $n$.

For all four values of $n$, prediction error seems to increase with $K$. This is
likely because our training set is only 100 digits, and as $K$ increases towards
half of the training set, the model moves closer towards simply estimating the
sample mean.


\subsection{Task \#2}

\begin{figure}[H]
  \includegraphics[width=\textwidth]{figures/fig2_3.png}
\caption{\footnotesize \it Mean zero/one prediction error of KNN for various degrees
         of data corruption; for n = 80 and varying i, k.}
\label{fig:2-3}
\end{figure}

\paragraph{How does corruption magnitude influence prediction accuracy and the optimal
value of K?}~\smallskip

Not surprising, there is a correlation between corruption magnitude and
prediction error -- for the uncorrupted set, the prediction error lies roughly
in the range $[0, 0.16]$, while prediction errors lie roughly in the ranges
$[0.02, 0.18]$, $[0.1, 0.3]$, and $[0.2, 0.4]$ for the lightly, moderately, and
heavily corrupted sets, respectively. For the heavily corrupted set, prediction
error reaches as high as roughly 40\% for $i = 5$.

As far as the optimal $K$, based on the plots we may still be inclined to simply
choosing small values of $K^*$ for the three corrupted sets, but the indication
is not as strong based on the plots alone, since the trend in the curves seem
flatter than that of the uncorrupted set -- this perhaps applies less so for the
lightly corrupted set.

Most interestingly, however, is perhaps the fluctuation in validation error
across values of $i$. This has not been examined explicitly, but for the heavily
corrupted set it appears as though the five curves stick closer together than
for the uncorrupted set.

\sectend

\section{Preprocessing}

\subsection{Abu Mostafa exercise 9.1}

If income is measured in thousands of dollars, then the euclidean distance
between Mr. Unknown to Mr. Good and Mr. Bad are:

\begin{align*}
  \text{dist(Mr. Unknown, Mr. Good)} &= \sqrt{(21 - 47)^2 + (36 - 35)^2}\\
                                     &= 26.02,\\[4pt]
  \text{dist(Mr. Unknown, Mr. Bad)} &= \sqrt{(21 - 22)^2 + (36 - 40)^2}\\
                                    &= 4.12,
\end{align*}

and so Mr. Unknown is clearly much closer to Mr. Bad, and thus BoL should
\textit{not} give credit to Mr. Unknown.

However, if income is measured in dollars:

\begin{align*}
  \text{dist(Mr. Unknown, Mr. Good)} &= \sqrt{(21 - 47)^2 + (36000 - 35000)^2}\\
                                     &= 1000.34,\\[4pt]
  \text{dist(Mr. Unknown, Mr. Bad)} &= \sqrt{(21 - 22)^2 + (36000 - 40000)^2}\\
                                    &= 4000.00,
\end{align*}

then Mr. Unknown is closer to Mr. Good, since now income weighs much heavier on
the distance than does age, and so here BoL \textit{should} give credit to Mr.
UNknown. 

\subsection{Abu Mostafa exercise 9.2}

\newcommand{\one}{\mbf{1}_n}
\newcommand{\id}{\mbf{I}_n}

Starting from the definition of $\mbf Z$, the centering of $\mbf X$, as given in
Abu Mostafa chp. 9.1:

\begin{align*}
  \mbf Z = \mbf X - \one \left(\tfrac{1}{n}\mbf X \T \one\right)\T,
\end{align*}

where $\one$ is a vector of $n$ ones. Multiply both sides with $\mbf X^{-1}$:

\begin{align*}
  \mbf Z \mbf X^{-1} &= \id - \one \left(\tfrac{1}{n}\mbf X \T \one\right)\T \mbf
  X^{-1}.
\end{align*}

Using $(\mbf A\T \mbf B) \T = \mbf B \T \mbf A$:

\begin{align*}
  &= \id - \one \left(\tfrac{1}{n}\one\T \mbf X\right) \mbf X^{-1}\\[4pt]
  &= \id - \one \tfrac{1}{n}\one\T \mbf X \mbf X^{-1}\\[4pt]
 &= \id - \tfrac{1}{n}\one \one\T\ .
\end{align*}

Now, multiply both sides with $\mbf X$:

\begin{align*}
  \mbf Z &= \left(\id - \tfrac{1}{n}\one \one\T\right) \mbf X\ .
\end{align*}
\qed

\subsection{Abu Mostafa exercise 9.4}
\subsubsection{Part (a) -- variance and covariance}

For $x_1$ we have simply $\var{x_1} = \var{\hat x_1} = 1$. For $x_2$:

\begin{align*}
  \var{x_2} &= \var{\sqrt{1 - \eps^2}\hat x_1 + \eps\hat x_2}
\end{align*}

Using independence of $\hat x_1$ and $\hat x_2$ and $\var{\hat x_1} = \var{\hat
x_2} = 1$:

\begin{align*}
  &= \var{\sqrt{1 - \eps^2} \hat x_1} + \var{\eps \hat x_2}\\
  &= (1 - \eps^2) + \eps^2\\&= 1.\\
\end{align*}

For the covariance, we want to compute:
\begin{align}
  \cov{x_1, x_2} =\cov{\hat x_1, \sqrt{1 - \eps^2} \hat x_1 + \eps  \hat
  x_2}\label{eq:target}.
\end{align}


First, note that covariance is a bilinear operation, meaning that for
RV's $X, Y, Z$ and constants $a, b$, we have:

\begin{align*}
  \cov{X, aY + bZ} &= \cov{X, aY} + \cov{X, bZ}\nonumber\\
                    &= a * \cov{X, Y} + b * \cov{X, Z}
\end{align*}

In this case we have $(X, Y, Z) = (\hat x_1, \hat x_1, \hat x_2)$ and
$(a, b) = (\sqrt{1 - \eps^2}, \eps)$. Substituting this into \cref{eq:target}:

\begin{align*}
  \cov{\hat x_1, \sqrt{1 - \eps^2} \hat x_1 + \eps  \hat x_2} &=
  \sqrt{1 - \eps^2}\  \cov{\hat x_1, \hat x_1} + \eps\  \cov{\hat x_1, \hat
  x_2}
\end{align*}

Using $\cov{\hat x_1, \hat x_1} = \var{\hat x_1} = 1$:

\begin{align*}
  &= \sqrt{1 - \eps^2} + \eps\  \cov{\hat x_1, \hat x_2}
\end{align*}

Using independence of $\hat x_1$ and $\hat x_2$, we have $\cov{\hat
x_1, \hat x_2} = 0$, and finally:

\begin{align*}
\cov{\hat x_1, \sqrt{1 - \eps^2}  \hat x_1 + \eps  \hat x_2} =
  \sqrt{1 - \eps^2}.
\end{align*}


\subsubsection{Part (b) -- linearity in correlated inputs}

To show that $f$ is linear in its correlated inputs, I want to find $(a,\, b,\, c,\, d)$ such that:
\begin{align*}
  w_1 &= a\hat w_1 + b \hat w_2\\
  w_2 &= c\hat w_1 + d \hat w_2\ .
\end{align*}

Consider $f(\mbf x)$:
\begin{align*}
  f(\mbf x) &= w_1 x_1 + w_2 x_2\\
            &= w_1\hat x_1 + w_2 \left(\sqrt{1 - \eps^2}\hat x_1 + \eps \hat x_2  \right)\\
            &= w_1 \hat x_1 + w_2 \sqrt{1 - \eps^2}\ \hat x_1 + \eps w_2 \hat
            x_2\\
            &= \left(\sqrt{1 - \eps^2}\ w_2 + w_1\right) \hat x_1 + \eps w_2 \hat x_2
\end{align*}

We obtain $\hat w_1 = \sqrt{1 - \eps^2}\ w_2 + w_1$ and $\hat w_2 = \eps w_2$.
Thus $f$ is linear in its correlated inputs with:
\begin{align*}
  (a,\, b,\, c,\, d) \ = \ \left(1, -\sqrt{1 - \eps^2}, 0, \tfrac{1}{\eps}\right)\ .
\end{align*}

\subsubsection{Part (c) -- bounding regularization}

If we have $f(\hat{\mbf x}) = \hat x_1 + \hat x_2 = \hat w_1 \hat x_1 + \hat w_2
\hat x_2$ with $\hat w_1 = \hat w_2 = 1$, then the correlated inputs are:
\begin{align*}
  \hat w_2 = 1 &\quad \Leftrightarrow\quad w_2 = \frac{1}{\eps},\\[8pt]
\end{align*}
and:
\begin{align*}
  \hat w_1 = 1 \quad \Leftrightarrow\quad w_1 &= 1 - \sqrt{1 - \eps^2}\ w_2\\
                                              &= 1 - \frac{\sqrt{1 -
                                              \eps^2}}{\eps}\ ,
\end{align*}

and the lower bound on $C$ is:
\begin{align*}
  C \ \geq\  w_1^2 + w_2^2 &= \left(1 - \frac{\sqrt{1 - \eps^2}}{\eps}\right)^2 +
  \frac{1}{\eps^2}\\
                           &= \frac{(\eps - \sqrt{1 - \eps^2})^2}{\eps^2} +
  \frac{1}{\eps^2}\\
                           &= \frac{1 + (\eps - \sqrt{1 - \eps^2})^2}{\eps^2}\ .
\end{align*}

Using $(a - b)^2 = a^2 + b^2 - 2ab$:
\begin{align*}
  C \ &\geq \ \frac{1 + (\eps^2 + \sqrt{1 - \eps^2}^2 - 2\eps\sqrt{1 - \eps^2})^2}{\eps^2}\\[4pt]
  &= \ \frac{1 + (\eps^2 + 1 - \eps^2 - 2\eps\sqrt{1 - \eps^2})^2}{\eps^2}\\[4pt]
  &= \ \frac{2 - 2\eps\sqrt{1 - \eps^2}}{\eps^2}\ .
\end{align*}

\subsubsection{Part (d) -- lower bound on maximum regularization}

The question is essentially:
\begin{align*}
  \lim_{\eps \to 0}\ C \quad &\geq\quad  \lim_{\eps \to 0}\ \frac{2 -
  2\eps\sqrt{1 - \eps^2}}{\eps^2}\ .
\end{align*}

Since we have $2 - 2 \eps \sqrt{1 - \eps^2} < \eps^2$ for all $\eps \in (-1, 1)$
(note that given the square root, this is the range of $\eps$ so long as we
consider only real-valued solutions), we have:

\begin{align*}
  \lim_{\eps \to 0}\ \frac{2 - 2\eps\sqrt{1 - \eps^2}}{\eps^2} = \infty
\end{align*}

And thus we see that as correlation increases, the lower bound on regularization
grows infinitely, meaning we need an impossibly large amount of regularization
on the target function.

\sectend
\newpage



\begin{thebibliography}{8}

  \bibitem{clrs}
    Cormen, T.; Leiserson, C.; Rivest, R; and Stein, C. (2009).
    \textit{Introduction to algorithms}. Cambridge (Inglaterra): MIT Press.

  \bibitem{grady}

    Grady, Leo (2006). \textit{Random Walks for Image Segmentation}.
    IEEE Transactions on Pattern Analysis and Machine Intelligence, Vol. 28, no.
    11, Nov. 2006.\\
    \url{http://leogrady.net/wp-content/uploads/2017/01/grady2006random.pdf}

\end{thebibliography}

\end{document}
