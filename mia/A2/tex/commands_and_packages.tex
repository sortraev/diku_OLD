%% FROM THE PROVIDED math_symbols.tex
\usepackage{amsthm}
\usepackage{amssymb}
\usepackage{amsmath}
\usepackage{mathtools}
\usepackage{xifthen}
\usepackage{xparse}
\usepackage{dsfont}


% Left-right bracket
\newcommand{\lr}[1]{\left (#1\right)}

% Left-right square bracket
\newcommand{\lrs}[1]{\left [#1 \right]}

% Left-right curly bracket
\newcommand{\lrc}[1]{\left \{#1\right\}}

% Left-right absolute value
\newcommand{\lra}[1]{\left |#1\right|}

% Left-right upper value
\newcommand{\lru}[1]{\left \lceil#1\right\rceil}

% Scalar product
\newcommand{\vp}[2]{\left \langle #1 , #2 \right \rangle}

% The real numbers
\newcommand{\R}{\mathbb R}

% The natural numbers
\newcommand{\N}{\mathbb N}

% Expectation symbol with an optional argument
\NewDocumentCommand{\E}{o}{\mathbb E\IfValueT{#1}{\lrs{#1}}}

% Indicator function with an optional argument
\NewDocumentCommand{\1}{o}{\mathds 1{\IfValueT{#1}{\lr{#1}}}}

% Probability function
\let\P\undefined
\NewDocumentCommand{\P}{o}{\mathbb P{\IfValueT{#1}{\lr{#1}}}}

% A hypothesis space
\newcommand{\HH}{\mathcal H}

% A sample space
\newcommand{\XX}{\mathcal{X}}

% A label space
\newcommand{\YY}{\mathcal{Y}}

% A nicer emptyset symbol
\let\emptyset\varnothing

% Sign operator
\DeclareMathOperator{\sign}{sign}
\newcommand{\sgn}[1]{\sign\lr{#1}}

% KL operator
\DeclareMathOperator{\KL}{KL}

% kl operator
\DeclareMathOperator{\kl}{kl}

% The entropy
\let\H\relax
\DeclareMathOperator{\H}{H}

% Majority vote
\DeclareMathOperator{\MV}{MV}

% Variance
\DeclareMathOperator{\V}{Var}
\NewDocumentCommand{\Var}{o}{\V\IfValueT{#1}{\lrs{#1}}}

% VC
\DeclareMathOperator{\VC}{VC}

% VC-dimension
\newcommand{\dVC}{d_{\VC}}

% FAT ...
\DeclareMathOperator{\FAT}{FAT}
\newcommand{\dfat}{d_{\FAT}}
\newcommand{\lfat}{\ell_{\FAT}}
\newcommand{\Lfat}{L_{\FAT}}
\newcommand{\hatLfat}{\hat L_{\FAT}}

% Distance
\DeclareMathOperator{\dist}{dist}

% change all asterisks to \cdots in math mode. use \textasciiasterisk for
% asterisks in math mode.
\DeclareMathSymbol{*}{\mathbin}{symbols}{"01}

% easy integrals over -infinty .. infinity
\newcommand{\infint}{\int\limits_{-\infty}^{\infty}}


%% MY PACKAGES
\usepackage[utf8]{inputenc}
\usepackage[plainpages=false]{hyperref}
\usepackage{cleveref}
\usepackage{graphicx}
\usepackage{minted}    % code snippets
\usemintedstyle{trac}
\setminted{fontsize=\scriptsize, highlightcolor=gray, linenos, frame=lines, framesep=10pt}
% \usepackage{tikz}
% \usetikzlibrary{automata, positioning}
\usepackage[]{xcolor}
\definecolor{gray}{RGB}{235, 230, 222}
\definecolor{myRed}{RGB}{173, 20, 0}
\definecolor{myGreen}{RGB}{45, 150, 0}
\newcommand{\gray}[1]{{\setlength{\fboxsep}{0pt}\colorbox{gray}{#1}}}
\everymath{\displaystyle} % force display style for inline math. why though?


%% My COMMANDS
% big, fat, red TODO's.
\newcommand{\TODO}[1]{\textcolor{red}{TODO: #1}}

\newcommand\numberthis{\addtocounter{equation}{1}\tag{\theequation}} % equation numbering for align*

%% math related commands/operators.

% neat section and subsection separators (but have to insert manually).
\newcommand{\sectend}{\smallskip\noindent\makebox[\textwidth]{\rule{\textwidth}{0.4pt}}}
\newcommand{\Sectend}{\medskip\noindent\makebox[\textwidth]{\rule{1.1\textwidth}{1pt}}}

% easy monospace text for inline code. also,
% no need to escape underscores as with \texttt.
\newcommand{\ms}[1]{\mintinline[fontsize=\normalsize]{text}{#1}}

\DeclareMathOperator*{\argmin}{argmin}

\newcommand{\mbf}[1]{\mathbf{#1}}
\usepackage{geometry}
