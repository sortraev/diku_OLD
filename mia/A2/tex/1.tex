\section{Medical Image Segmentation}

Perhaps the most commonly used technique in medical image analysis is
\textit{segmentation}.
% When imaging the inside of the human body, we are
% typically interested in focusing on one or a small number of specific objects in
% the imaged element, and while we for some imaging technologies (eg. SPECT) and
% for some types of images are able to preparate the imaged subject with eg. radio
% tracers that emphasize specific targets in the resulting image, we are in
% general not very capable of isolating specific bodily objects using the raw
% images alone.
%
This is the process of identifying and specific objects in a given image, and
separating them from each other and/or from the background of the image, such
that they can be examined in isolation.

Many different methods exist for computer-assisted image segmentation -- in this
report we shall discuss histogram-based, graph-based, and, briefly, shape-based
image segmentation.


\subsection{Histogram-based Segmentation}
\label{sec:hist-seg}

The simplest form of image segmentation is histogram-based segmentation. The
idea behind is that if we can identify important (bands of) pixel intensities
corresponding to target object/s, then we can isolate those objects by removing
pixels which fall outside of those bands.

In \cref{fig:hist-seg} an example noisy image is segmented using histogram-based
segmentation. Since the image is known ahead of time to contain four significant
segments, the histogram is performed using 10 bins, and the four largest histogram
bands are extracted and used in thresholding.

\begin{figure}%[H]
  \centering
  \includegraphics[width=0.8\textwidth]{figures/histogram_segmentation.png}
  \caption{\small \textit{Histogram-based segmentation of an example image. Note
  that only thresholdings of the two largest bins are shown here. Image created
  using } \url{jspaint.app}.}
\label{fig:hist-seg}
\end{figure}


The main advantage of the histogram method is \textit{simplicity} --
however, the method can fail if object pixels are not distinct from background
pixels, or if within one object pixels from different ends of the histogram
spectrum exist, since then thresholding woulld inevitably only be able to target
part of that particular object. The example in \cref{fig:hist-seg} was
successful mainly because each object contained only one solid color
(modulo noise), and because those four colors were relatively contrasting.

Another related disadvantage is that the output segmentation might contain noise
if the input has noise in the same histogram ranges as desired objects, as was
the case in the example in \cref{fig:hist-seg}. In \cref{sec:morphs} we shall
see discuss how to reduce such noise.

\sectend
