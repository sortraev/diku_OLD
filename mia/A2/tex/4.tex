\section{Shape-based Segmentation}

Another, yet advanced range of methods are the \textit{shape-based}
segmentations. One popular such method draws inspiration from the field of
statistics and uses principal component analysis (PCA) to extract and analyze
important features of shapes in (medical) images.

By teaching a model on existing image data of, say, lung fields, the most
important features (or landmarks), defining the shape of the average lung field,
are determined, and can be efficiently applied to classify and segment objects
in new images.

One disadvantage of the method (and shape-based methods in general) is the
limited range of use. Since a shape-based model is specifically trained on
existing data, it can only truely excel at detecting and segmenting objects
which resemble the data on which it is trained -- this assumption is suitable
for segmenting eg. organs and bones, since we don't expect the shapes of these
to deviate to the point of unrecognizability, but it may be difficult to detect
objects foreign to the body, eg. cell deformations.

Another negative consequence of the supervised nature of the PCA-based method is
the amount of training data required to build a model. While histogram and graph
based segmentation each do require some sort of user input in the form of
thresholding bands or object/background seeds, these methods can still be
considered unsupervised as they do not require data independent of the input
image, whereas, as an example, PCA-based segmentation of a lung field would
require prior knowledge of the important features of the average lung field.
However, in the modern age of data abundancy, especially in the world of medical
imaging, this is not necessarily a drawback to the algorithm.

\sectend
