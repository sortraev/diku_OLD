%% various math packages
\usepackage{amsthm}
\usepackage{amssymb}
\usepackage{amsmath}
\usepackage{amsfonts}
\usepackage{mathtools}
\usepackage{xifthen}
\usepackage{xparse}
\usepackage{dsfont}
\everymath{\displaystyle} % force display style for inline math
\newcommand\numberthis{\addtocounter{equation}{1}\tag{\theequation}} % equation numbering for align*

%% misc packages.
\usepackage{xfrac}
\usepackage{parskip}
\usepackage[utf8]{inputenc}
\usepackage{hyperref}
\usepackage{cleveref}
\usepackage{graphicx}
\usepackage{float}
\usepackage{subcaption}
\usepackage{enumitem}
\usepackage[outputdir=out]{minted}
\usemintedstyle{trac}
\setminted{%frame=lines,
    linenos=true,
    fontsize=\footnotesize
}
\usepackage[table,xcdraw]{xcolor}


%% math commands.
\DeclareMathOperator*{\argmin}{argmin} % argmin.
\newcommand{\bbar}[1]{\overline{#1}}   % a wider bar than \bar.
\newcommand{\eps}{\varepsilon}         % prettier epsilon.
\newcommand{\mbf}[1]{\mathbf{#1}}      % shorthand.
\newcommand{\R}{\mathbb R}             % set of real numbers.
\newcommand{\N}{\mathbb N}             % set of natural numbers.
\newcommand{\Z}{\mathbb Z}             % set of integers.
\let\emptyset\varnothing               % better emptyset symbol. 
\newcommand{\pr}[1]{\text{Pr}\left[#1\right]} % pretty probabilities.
\DeclareMathSymbol{*}{\mathbin}{symbols}{"01} % map asterisks to \cdot. use
                                              % \ast for asterisk in math mode. 

% bold and red TODO's.
\newcommand{\TODO}[1]{\textcolor{red}{TODO: #1}}

