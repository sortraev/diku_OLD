\newcommand{\infint}{\int\limits_{-\infty}^\infty}
\section{Image filtering}

\subsection{Finite diffference kernels}

\emph{Note: In the following, I assume that $x$-coordinates grow left to right,
and that $y$-coordinates grow bottom up, as in a regular coordinate system.}

To compute the finite difference approximation using correlation, the kernels
should be:
\begin{align*}
  y\text{-derivative correlation kernel: }&\quad
  \frac 1 2
  \begin{bmatrix}
    1\\
    0\\
   -1
  \end{bmatrix},\\
  x\text{-derivative correlation kernel: }&\quad
  \frac 1 2
  \begin{bmatrix}
    -1 & 0 & 1
  \end{bmatrix}.
\end{align*}

If we want to use convolution rather than correlation, the kernels should each
be flipped since it is not symmetric, ie.:
\begin{align*}
  y\text{-derivative convolution kernel: }&\quad
  \frac 1 2
  \begin{bmatrix}
    -1\\
     0\\
     1
  \end{bmatrix},\\
  x\text{-derivative convolution kernel: }&\quad
  \frac 1 2
  \begin{bmatrix}
    1 & 0 & -1
  \end{bmatrix}.
\end{align*}

For all kernels, the center pixel is the second element of the kernel, ie. the 0
in each kernel. For symmetric kernels such as the Gaussian or mean filter
kernels, correlation and convolution produce the same result, since
flipping/reflecting a symmetric function/kernel yields the same kernel.

\newpage
\subsection{Linear separability of Prewitt and Sobel}

\emph{The assignment text does not state whether to write the correlation or
convolution filters, so I choose convolution. Also, I again assume that x grows
left to right, and that y grows upwards.}

The two separated $y$-derivative \textbf{convolution} filters are:
\begin{align*}
  \text{Prewitt }y\text{-derv. convolution kernel}: &\quad
  \begin{bmatrix}
    1 & 1 & 1\\
    0 & 0 & 0\\
    -1 & -1 & -1 
  \end{bmatrix} = 
  \begin{bmatrix}
    1\\0\\-1
  \end{bmatrix} * 
  \begin{bmatrix}
    1 & 1 & 1
  \end{bmatrix},\\[12pt]
  \text{Sobel }y\text{-derv. convolution kernel}: &\quad
  \begin{bmatrix}
    1 & 2 & 1\\
    0 & 0 & 0\\
    -1 & -2 & -1 
  \end{bmatrix} = 
  \begin{bmatrix}
    1\\0\\-1
  \end{bmatrix} * 
  \begin{bmatrix}
    1 & 2 & 1
  \end{bmatrix}.
\end{align*}

The Prewitt and Sobel operators are more likely to be robust to noise than the
simple finite difference approximation for the following reason. An image edge
exists where there is a sudden change in pixel intensity between two
neighbouring (groups of) pixels. However, we typically think of an edge as a
sequence of pairwisely neighbouring pixels for which a smooth curve can be drawn
in the cartesian space which visits each pixel in order, such that each
neighbouring pair of pixels share similar pixel intensities, and such that for
each pixel in the sequence, its neighbouring pixel in the direction
\emph{perpendicular} to the edge sequence have significantly different pixel
intensities to that of the edge pixel.

The Prewitt and Sobel operators each respect this notion of edges by not only
considering the center pixel, but also the pixels above and below \emph{or} to
the left and right of the center pixel, for the $x$- and $y$-derivatives,
respectively, such that, for example, a single pixel that contrasts with its
neighbours in the $x$-direction is not considered part of an edge unless it also
contrasts, to some degree, with its diagonal neighbours in the $x$-direction,
whereas the simple finite diff. approximation filters would regard a single
noisy pixel as an edge on its own, regardless of the direction.
